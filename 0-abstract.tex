
\begin{abstract}
This paper introduces Yerbamate, an open science framework and software engineering convention for Python-based artificial intelligence (AI) projects. Yerbamate prioritizes modularity and separation of concerns to promote reproducibility, reusability, shareability, maintainability, and code quality. The convention and framework can be applied to a variety of Python-based tasks, including experimentation with machine learning, deep learning, genetic algorithms, optimizations, and simulations and analysis. Yerbamate simplifies the implementation of the convention with straightforward commands for installing, experimenting, and running code. The framework is compatible with all Python libraries and AI/ML frameworks, and it fosters open science by promoting the sharing of code modules among researchers and practitioners. The adoption of this convention and framework can help address the reproducibility crisis in AI and enhance the efficiency and effectiveness of AI development while promoting responsible and socially beneficial use of AI technology.
\end{abstract}





% This paper presents a software engineering convention, accompanied by a framework named Yerbamate, for Python-based artificial intelligence (AI) projects. The convention emphasizes modularity and separation of concerns, with the goal of promoting reproducibility, reusability, shareability, maintainability, and code quality. The use of modular code structures enables flexibility and scalability, facilitating the reuse of code and its components. The Yerbamate framework simplifies the implementation of the convention by providing a set of simple and consistent commands for installing, experimenting, and running code. The convention and framework are compatible with all Python libraries, AI and ML frameworks, and can be used for a variety of Python-based experimentation, analysis, ML, AI, genetic algorithms, particle swarm optimization, and simulations. The use of this convention and framework can address the reproducibility crisis in AI and promote open science by enabling the sharing of code modules among researchers and practitioners



%   In recent years, deep learning has emerged as a powerful tool for solving complex problems in various fields, such as image and speech recognition, natural language processing, and computer vision (\cite{lecun2015deep}). 
% Deep learning projects, whilst still considered software engineering, differ from traditional software engineering in that they revolve around the creation of models that can extract knowledge from data and make predictions or decisions, as opposed to relying on a pre-defined set of instructions (\cite{lecun2015deep,amershi2019software,wan2019does,se4dl}). These disparities make some phases such as the testing phase of deep learning projects distinct from traditional software (\cite{wan2019does}). Still, principles and design patterns commonly used in software engineering can still aid in developing and maintaining deep learning software (\cite{amershi2019software,wan2019does,se4dl}). 

% Artificial intelligence (AI) and machine learning (ML) have become increasingly powerful in recent years, with applications in fields ranging from natural language, art, music, healthcare, finance to education. While these technologies have the potential to revolutionize many areas of human life, their successful development and deployment depend heavily on quality code and sound software engineering practices. Good software engineering practices can increase the reliability, maintainability, and scalability of AI and ML systems, enabling their widespread adoption and use.