
\section{Results}


\subsection{Python Modularity Convention}
The convention is designed to be simple, effective, and widely applicable solution for organizing AI projects. It prioritizes modularity, separation of concerns, and consistent naming conventions to promote the maintainability and reusability of code. In line with this, the convention places a high priority on creating independent modules, which are standalone, reusable components that can be used across different project.

\subsection{Python Independent Module Convention}
Python independent modules only depend on Python dependencies (such as NumPy, PyTorch, TensorFlow, or Hugging Face), and the code inside the module uses relative imports. This design allows the reuse of code and its components under different projects. 



\subsection{No-Loop Python Experiment Configuration Convention}

This convention adopts an experiment configuration of the Python language that only disallows the use of loops as a convention. The experiment configuration design choice is intended to maintain a hyperparameter-focused experiment format while promoting separation of concerns. The resulting approach provides several benefits, including increased flexibility and customization, as the configuration format can be adapted to various AI tasks, custom use cases, frameworks, and libraries. 



\subsection{AI Project Convention}
The convention for structuring AI/ML projects involves four distinct modules: "models," "experiments," "trainers," and "data." This modular approach leads to the creation of three standalone modules for models, trainers, and data, each with the ability to operate independently. The fourth module, experiments, serves to harmonize the three modules and provide a unified and comprehensive experiment. The modular structure of independence enables the sharing of models, trainers, and data loaders across various projects and experiments, promoting the reusability of code.




\begin{figure}
\centering
\framebox[\0.45\textwidth]{%
\begin{minipage}{0.45\textwidth}
\dirtree{%
.1 /.
.2 models.
.3 \texttt{\_\_init\_\_.py}.
.2 experiments.
.3 \texttt{\_\_init\_\_.py}.
.2 trainers.
.3 \texttt{\_\_init\_\_.py}.
.2 data.
.3 \texttt{\_\_init\_\_.py}.
.2 \texttt{\_\_init\_\_.py}.
}
\end{minipage}
}
\caption{The modular architecture of the convention is characterized by the structured folder system, separating the key components into four distinct modules: models, experiments, trainers, and data. This enhances code legibility and comprehensibility, enabling other researchers to quickly access the desired code components.}
\end{figure}




% \subsubsection{}

% The implementation of software engineering principles of as separation of concerns (SoC) and modularity in deep learning projects results in a well-structured illustrated below. 

% \vspace{0.7em}


% The modular architecture of a deep learning project, characterized by the structured folder system, separates the key components into four distinct modules: models, experiments, trainers, and data. This enhances code legibility and comprehensibility, enabling other researchers to quickly access the desired code components. This approach leads to the formation of three standalone modules for models, trainers, and data, each possessing the ability to operate independently. Meanwhile, the fourth module, experiments, functions to harmonize the three modules and provide a unified and comprehensive experiment. The modular structure of independence further enables the sharing of models, trainers, and data loaders across various projects and experiments, thereby enhancing the code's reusability and scalability. 


\begin{figure}
\centering
\framebox[\0.45\textwidth]{%
\begin{minipage}{0.45\textwidth}


\dirtree{%
.1 /.
.2 data.
.3 cifar10.
.4 \texttt{data\_loader.py}.
.4 \texttt{\_\_init\_\_.py}.
.3 cifar100.
.4 \texttt{data\_loader.py}.
.4 \texttt{\_\_init\_\_.py}.
.3 \texttt{\_\_init\_\_.py}.
.2 models.
.3 resnet.
.4 \texttt{fine\_tune.py}.
.4 \texttt{resnet.py}.
.4 \texttt{\_\_init\_\_.py}.
.3 efficientnet.
.4 \texttt{efficientnet.py}.
.4 \texttt{\_\_init\_\_.py}.
.3 \texttt{\_\_init\_\_.py}.
.2 trainers.
.3 classification.
.4 \texttt{pl\_classification\_module.py}.
.4 \texttt{\_\_init\_\_.py}.
.3 \texttt{\_\_init\_\_.py}.
.2 experiments.
.3 torch.
.4 \texttt{resnet\_cifar10.py}.
.4 \texttt{resnet\_cifar100.py}.
.4 \texttt{fine\_tune\_resnet\_cifar.py}.
.4 \texttt{\_\_init\_\_.py}.
.3 keras.
.4 \texttt{keras\_efficient\_net\_cifar.py}.
.4 \texttt{\_\_init\_\_.py}.
.3 \texttt{\_\_init\_\_.py}.
.2 \texttt{\_\_init\_\_.py}.
}
\end{minipage}
}
\caption{This figure illustration a sample folder structure for a deep learning vision project\footnote{ \url{https://github.com/oalee/deep-vision}}. Each folder represents a modular code component with a specific concern, which is demonstrated by the presence of \texttt{\_\_init\_\_.py} within the folder.}
\end{figure}



% \subsection{Example Project Structure}


% \vspace{2em}
% Each top-level module can be divided into multiple sub-modules, allowing for the separation of individual components within each module. For instance, the models module can comprise sub-modules such as ResNet (\cite{resnet}), ViT (\cite{dosovitskiy2020vit}), CvT (\cite{wu2021cvt}), and others, implemented as independent components. Likewise, the data module can encompass sub-modules such as data loaders, preprocessing pipelines, and post-processing routines. This modular design promotes increased organization and manageability of the codebase, contributing to the efficiency and effectiveness of the deep learning project.


\subsection{Yerbamaté: An Open Science Python Framework}
% Yerbamaté\footnote{\url{https://github.com/oalee/yerbamate}} is an open-source, open-science Python framework that aims to streamline and simplify the development and management of AI projects. It promotes modular design principles and encourages quality coding, collaboration, sharing of models, trainers, data loaders, and knowledge, while also prioritizing reproducibility, customization, and flexibility. The framework provides a command-line interface (CLI) toolkit that conforms to the software engineering convention of modularity and independence, allowing the creation and injection of dependencies for greater reproducibility and sharing capabilities. Moreover, Yerbamate is compatible with Linux systems and Jupyter notebooks, providing researchers with the ability to run experiments on Colab. By adhering to the Yerbamate convention, any module in a project is automatically sharable.
% is a result of the study on reproducibility and the importance of modularity in AI and machine learning projects. Yerbamaté is an open source open science framework designed to streamline and simplify the development and management of artificial intelligence and machine learning projects. Built around the principles of modularity and separation of concerns, Yerbamaté provides a convenient and efficient means of adding source code and dependencies to projects.

% Modularity allows for a clean and organized codebase, making it easier to maintain, scale, and reuse the code in the future. In addition, Yerbamaté provides an easy-to-use interface for sharing the source code of models, trainers, data loaders, and experiments between modular projects, enabling greater flexibility and collaboration. Furthermore, adopting the Yerbamaté framework ensures that all projects adhering to its principles are readily compatible with Colab\footnote{\url{https://colab.research.google.com/github/oalee/yerbamate/blob/main/deep_learning.ipynb}}, a cloud-based platform widely used for machine learning experimentation and development.

% Yerbamaté also provides support for full customizability and reproducibility of results through the inclusion of dependencies in your project. The tool supports pip and conda for dependency management, making it easy to manage and install the necessary dependencies for your project.
% Additionally, Yerbamaté is fully compatible with python, and can be used with popular libraries such as PyTorch/Lightning, TensorFlow/Keras, JAX/Flax, Huggingface/transformers. Another feature of Yerbamaté is its convenient environment management through the Yerbamaté Environment API. This API allows for the creation and management of virtual environments, ensuring that each experiment is run in an isolated environment with the necessary dependencies. 
% For a comprehensive understanding of the Yerbamaté, see the documentation\footnote{\url{https://oalee.github.io/yerbamate/}}.


Yerbamate is an open-source open-science Python framework designed to streamline and simplify the development and management of AI projects using modular design principles and open source. The framework encourages quality coding, collaboration, and sharing of models, trainers, data loaders, and knowledge, while also promoting reproducibility, customization, and flexibility. Yerbamate is designed as a command-line interface (CLI) toolkit that works around the software engineering convention of modularity and independence. The CLI can be used to create and inject dependencies, providing greater reproducibility and sharing capabilities. Additionally, Yerbamate is compatible with Linux systems and Jupyter notebooks, enabling researchers to run their experiments on Colab out of the box. The framework also offers an Environment API, which, along with the CLI, is presented in the appendix and subject to updates. Users can refer to the documentation for the most up-to-date material. Any module in a project adhering to the Yerbamate convention is automatically sharable out of the box, leveraging the power of open source to enhance collaboration and knowledge sharing in the AI community.


\subsubsection{Flexibility and Customization}
One of the primary challenges in the development of Yerbamaté was ensuring flexibility and customization.
To offer increased flexibility, Yerbamaté does not impose any restrictions on additional module names, allowing researchers to utilize their preferred module names for custom tasks. For instance, researchers can use module names such as "simulations", "analyzers" for their use cases.
% framework-agnostic, meaning it can work out of the box with any machine learning framework or library.  Moreover, Yerbamaté offers compatibility with Python and Python can be used for running experiments. 




