

\section{Methodology}

The methodology of this study involved the development of a software engineering convention for AI projects that emphasizes modularity and separation of concerns. The convention was designed to be simple and effective, and involved the separation of code into independent modules whenever possible. 
% For AI and ML projects, this involved the creation of three independent modules for models, trainers, and data, along with one non-independent module for experimentation. The models module contains code for defining and constructing machine learning models. The trainers module contains code for training models and optimizing their parameters. The data module contains code for loading and preprocessing data. The experimentation module contains code for running experiments and evaluating the performance of the trained models.

\subsection{Python Independent Module Convention}
One of the key aspects of the convention is the use of Python independent modules. These modules only depend on Python dependencies (such as NumPy, PyTorch, TensorFlow, or Hugging Face), and the code inside the module uses relative imports. This design enables the reuse of code and its components under different parent modules.



\subsection{No-Loop Python Experiment Configuration Convention}

The experiment configuration in this study adopts a restricted version of the Python language that only disallows the use of loops as a convention. The experiment configuration design choice is intended to maintain a hyperparameter-focused experiment format while promoting separation of concerns. The resulting approach provides several benefits, including increased flexibility and customization, as the configuration format can be adapted to various AI tasks, custom use cases, frameworks, and libraries. Additionally, by incorporating well-documented Python code, the readability of the configuration file is improved, making it more accessible to researchers and practitioners alike. Furthermore, this format is executable directly with Python, which makes it Turing complete. This property is particularly useful because it means that the format can include arbitrary computation and is capable of expressing any algorithm, enhancing its flexibility and power.



% \section{Methodology}

% The widespread adoption of the Python programming language in AI programming has been noted in recent studies (\cite{mihajlovic2020use,raschka2020machine}). As such, this project focuses on the examination of software engineering practices in the context of Python and its machine learning frameworks. Many interdisciplinary researchers may lack the software engineering expertise necessary to manage their code bases to ensure correctness, understandability, extendability, and reusability (\cite{amershi2019software,scully-debt-ml,leakage-recrisis,accountabilityInAi}). To address these challenges, we investigated the application of software engineering principles of separation of concerns (SoC) and modular design in Python to AI projects. The goal of this study was to improve the reproducibility and reusability of python AI components through the adoption of best software engineering practices.

% \subsection{Software Engineering for Python AI Projects}
%  In the context of AI, the implementation of modularity and SoC principles results in the separation of models, data-related components, and trainers. 
% This approach facilitates the creation of a modular design, where the code is organized into distinct, reusable components (\cite{sanner1999python,pressman2010software}). This decoupling has the potential to greatly enhance the reusability of code greatly, enabling easier experimentation with models and datasets.


% \subsection{A Different Approach to Deep Learning Framework}

% An alternative approach to the compatibility and re-usability issue in deep learning frameworks is to enforce principles of separation of concerns (SoC) and modularity in Python. 
% The implementation of modularity and SoC principles results in the separation of models, data-related components, and trainers. 
% This approach facilitates the creation of a modular design, where the code is organized into distinct, reusable components (\cite{sanner1999python,pressman2010software}). This separation enhances the reusability of the code, making it easier to experiment with different models and datasets within the same framework. This approach has the advantage of limiting the learning overhead to the principles of software engineering, rather than requiring a transition to a unified deep learning framework.



% A limitation of this approach is that it requires manual translation of AI components, such as data loaders and models, into different frameworks when necessary. While this approach may require additional effort, it allows for greater flexibility and customizability in implementing deep learning models. Additionally, it promotes a deeper understanding of the underlying software engineering principles and their application in the field of AI.


