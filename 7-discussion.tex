\section{Discussion}

Yerbamaté is an open-source, open-science Python framework designed to promote software engineering principle of modularity, reproducibility, and openness in AI research. The framework offers a promising approach to addressing the reproducibility crisis in AI, by standardizing the development and management of AI projects and facilitating the sharing of research findings. Yerbamaté's software engineering conventions and accompanying toolkit provide researchers with the tools to create and inject dependencies, leading to the reuse of code and components, and ultimately increasing reproducibility and the sharing of research findings.

% The modular design principles employed in Yerbamaté can enhance transparency, reproducibility, and the sharing of knowledge and research findings. However, it is worth noting that the adoption of these principles may require additional time and effort for researchers who are not familiar with software engineering principles. Additionally, compatibility with pure Python may pose some challenges to researchers who prefer other programming languages.

The flexibility and customization offered by Yerbamaté facilitate the adaptation of the framework to meet researchers' diverse needs, contributing to the development of more accessible and efficient AI research environments. Researchers with varying technical backgrounds can share coding under software engineering guidelines, fostering collaboration and interdisciplinary learning.

The impact of Yerbamaté on the reproducibility and trustworthiness of AI research is a crucial area for future research. The usability of the framework and its potential limitations should also be further investigated. Future studies can also explore the impact of Yerbamaté on facilitating interdisciplinary collaborations and knowledge sharing in the field of AI.

%  Another area for future work could be the creation of naming conventions for modules and functions, which could further enhance the modularity and readability of the code. Additionally, comprehensive evaluation and testing of the framework could help to identify any potential issues or areas for improvement, ensuring the continued effectiveness and usefulness of the Yerbamaté framework.
 