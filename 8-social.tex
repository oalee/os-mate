
\section{Social Impact}
Machine learning models and artificial intelligence systems have the potential to impact society, both positively and negatively significantly (\cite{mittelstadt2019principles, jobin2019global,arrieta2020explainable, floridi2018ai4people}). Consequently, AI's ethical and societal implications have been the subject of much discussion and research in recent years (\cite{floridi2018ai4people, goodman2017european, floridi2019establishing, mittelstadt2016ethics}).
Open science can significantly impact society by promoting collaboration, transparency, and accessibility in research, enabling broader participation in the scientific process and sharing of knowledge and tools, thus accelerating safer progress and (\cite{kocak2022transparency, wachter2017transparent, coro2020open, braun2018open, paton2019open, goodman2017european}). 

In artificial intelligence, open science can help democratize access to machine learning and facilitate the development of ethical and accountable AI systems (\cite{goodman2017european, batarseh2020data}). By promoting the principles of open science, researchers and practitioners can work together to build more robust, trustworthy, and fair AI solutions (\cite{accountabilityInAi, kocak2022transparency,wachter2017transparent, coro2020open, braun2018open, hicks2021open, goodman2017european}.

The development of open source, open science, accessible and user-friendly tools for training machine learning models has the potential to democratize access to artificial intelligence and facilitate more involvement in research and innovation (\cite{goodman2017european}). The Yerbamaté toolkit contributes in this regard, providing a simple and effective means for researchers and practitioners to share, develop and evaluate machine learning models. Moreover, it makes training AI accessible to a broader audience as anyone can run an experiment on accessible science tools such as Colab\footnote{\url{https://colab.research.google.com/github/oalee/yerbamate/blob/main/deep_learning.ipynb}} and reproduce a scientific experiment and conduct their own experiments.


The wide accessibility of AI through Yerbamaté and other similar open science tools have the potential to accelerate research in various fields (\cite{olson2018system, wolf2020designing,morris2020ai,ong2021guide, li2018can}) including healthcare (\cite{haristiani2020combining}), law (\cite{ashley2017legal}), and education (\cite{goel2020ai}). For example, machine learning models can improve patient outcomes (\cite{hamet2017medicine}), detect fraud in financial transactions (\cite{bao2022fraudartificial}, and enhance personalized learning in education (\cite{haristiani2020combining, goel2020ai}). At the same time, it is essential to ensure that the development and application of AI adhere to ethical principles, including inclusion, transparency, accountability, and fairness (\cite{ accountabilityInAi,  jobin2019global, floridi2018ai4people, wachter2017transparent, arrieta2020explainable, mittelstadt2019principles, goodman2017european, mittelstadt2016ethics, floridi2019establishing, o2017weapons}).

