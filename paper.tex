\documentclass{IEEEtran}
\usepackage[utf8]{inputenc}
\usepackage[sorting = none]{biblatex}
\usepackage{amsmath,amssymb,amsfonts}
\usepackage{algorithm}
\usepackage{algpseudocode}
\usepackage{graphicx}
\usepackage{textcomp}
\usepackage[table]{xcolor}
\usepackage{array}
\usepackage{hyperref}           % page numbers and '\ref's become clickable
\usepackage{todonotes}
\usepackage{float}          % for forcing images to stay where they should
\usepackage{multirow}       % for using multirow in tables
\usepackage{tikz}
\usepackage{tabularx}
\usepackage{adjustbox}
\usepackage{longtable}
\usepackage[edges]{forest}
\usepackage[framemethod=TikZ]{mdframed}
\usepackage{amsmath}
\usepackage{caption}
\usepackage{subcaption}
\usepackage{MnSymbol}
\usepackage{tipa}
\renewcommand{\algorithmicrequire}{\textbf{Input:}}
\renewcommand{\algorithmicensure}{\textbf{Output:}}
\usepackage{dirtree}

% \usepackage{caption}


\mdfdefinestyle{Frame}{
    linecolor=black,
    outerlinewidth=0.3pt,
    roundcorner=2pt,
    innertopmargin=\baselineskip,
    innerbottommargin=\baselineskip,
    leftmargin =1cm,
    rightmargin =1cm,
    backgroundcolor=white}
    \mdfdefinestyle{Frame_NoMargin}{%
    linecolor=black,
    outerlinewidth=0.3pt,
    roundcorner=2pt,
    innertopmargin=\baselineskip,
    innerbottommargin=\baselineskip,
    backgroundcolor=white}
    \mdfdefinestyle{InnerFrame_NoMargin}{%
    linecolor=black,
    outerlinewidth=0.1pt,
    roundcorner=0.5pt,
    % innertopmargin=\baselineskip,
    % innerbottommargin=\baselineskip,
    leftmargin =-0.1cm,
    innerleftmargin = 0.05cm,
    rightmargin =0cm,
    backgroundcolor=white}  
    \mdfdefinestyle{InnerFrameBig_NoMargin}{%
    linecolor=black,
    outerlinewidth=0.1pt,
    roundcorner=0.5pt,
    % innertopmargin=\baselineskip,
    % innerbottommargin=\baselineskip,
    leftmargin =-0.22cm,
    innerleftmargin = 0.05cm,
    rightmargin =-0.22cm,
    backgroundcolor=white}
      \mdfdefinestyle{Frame_NoInnerMargin}{%
    linecolor=black,
    outerlinewidth=0.3pt,
    roundcorner=2pt,
    innerleftmargin = 0.02cm,
    innertopmargin=\baselineskip,
    innerbottommargin=\baselineskip,
    backgroundcolor=white}
      \mdfdefinestyle{Frame_Blank}{%
    linecolor=white,
    outerlinewidth=0.0pt,
    roundcorner=0pt,
    innerleftmargin = 0.3cm,
        innertopmargin=0cm,
    innerbottommargin=0.2cm,
    backgroundcolor=white}    

%%%%%%%%%%%%%%%%%%%%%%%
% Some commands       %
%%%%%%%%%%%%%%%%%%%%%%%
% Notes
\newcommand{\note}[1]{\todo[inline]{#1}}
\newcommand{\urgent}[1]{\todo[inline, color=red]{#1}}
\newcommand{\revision}[1]{\todo[inline, color=green]{#1}}
\newcommand{\fix}[1]{\todo[inline, color=yellow]{#1}}
% Tables
\newcolumntype{L}[1]{>{\raggedright\let\newline\\\arraybackslash\hspace{0pt}}m{#1}}
\newcolumntype{C}[1]{>{\centering\let\newline\\\arraybackslash\hspace{0pt}}m{#1}}
\newcolumntype{R}[1]{>{\raggedleft\let\newline\\\arraybackslash\hspace{0pt}}m{#1}}
\newcommand{\adaptcell}[1]{\parbox{0.95\columnwidth}{\vspace{0.5em}#1\vspace{0.5em}}}
% Math
\newcommand{\rarrow}{$\rightarrow\ $}
\newcommand{\larrow}{$\leftarrow\ $}
\newcommand{\lif}{\rightarrow}
\newcommand{\liff}{\leftrightarrow}
\newcommand{\la}{\forall}
\renewcommand{\le}{\exists}
\newcommand{\lxor}{\dot\vee}
\renewcommand{\ll}{\llbracket}
\renewcommand{\phi}{\varphi}
\newcommand{\rr}{\rrbracket}
\newcommand{\lk}{\square}
\newcommand{\lp}{\lozenge}
\newcommand{\argmin}[1]{\underset{#1}{\mathrm{argmin}}}
\newcommand{\argmax}[1]{\underset{#1}{\mathrm{argmax}}}
\newcommand{\Sum}[1]{\underset{#1}{\sum}}

% Others
\newcommand{\newcolumn}{\vfill\pagebreak}
\renewcommand{\b}[1]{\textbf{#1}}
\renewcommand{\it}[1]{\textit{#1}}
\renewcommand{\u}[1]{\underline{#1}}

%%%%%%%%%%% STUFF TO PRINT SEMANTIC TABLEAUX %%%%%%%%%%
\usepackage{tikz}
\usetikzlibrary{positioning,arrows,calc, trees, automata}

\usepackage{minted}
% \usepackage[T1]{fontenc}

\usepackage[zerostyle=b,scaled=.75]{newtxtt}
\usemintedstyle{colorful}

\setminted[python]{breaklines, framesep=2mm, fontsize=\footnotesize, numbersep=5pt}

% \usepackage{xcolor}
% \definecolor{Text}{HTML}{000000}
% \AtBeginEnvironment{minted}{\color{Text}}


\tikzset{
	modal/.style={>=stealth',shorten >=1pt,shorten <=1pt,auto,node distance=1.5cm,semithick},world/.style={circle,draw,minimum size=0.5cm,fill=gray!15},point/.style={circle,draw,inner sep=0.5mm,fill=black},reflexive above/.style={->,loop,looseness=7,in=120,out=60},reflexive below/.style={->,loop,looseness=7,in=240,out=300},reflexive left/.style={->,loop,looseness=7,in=150,out=210},reflexive right/.style={->,loop,looseness=7,in=30,out=330},none/.style={}
}

\forestset{%
	declare toks={T}{},
	declare toks={F}{},
	my label/.style={%
		tikz+={%
			\path[late options={%
				name=\forestoption{name},label={#1}}
			];
		}
	},
	tableaux/.style={%
		forked edges,
		for tree={
			math content,
			parent anchor=children,
			child anchor=parent,
		},
		where level=0{%
			for children={no edge},
			phantom,
		}{%
			before typesetting nodes={%
				content/.wrap value={\circ},
			},
			delay={%
				my label/.wrap pgfmath arg={{[inner sep=0pt, xshift=-3.5pt, yshift=3.5pt, anchor=north west, font=\scriptsize]-45:$##1$}}{content()},
				insert before/.wrap pgfmath arg={%
					[{##1}, no edge, math content, before drawing tree={x'+=7.5pt}]
				}{T()},
				insert after/.wrap pgfmath arg={%
					[{##1}, no edge, math content, before drawing tree={x'-=7.5pt}]
				}{F()},
			},
			if={n_children("!u")==1}{%
				before packing={calign with current edge},
			}{}
		},
	}
}

\forestset{
  smullyan tableaux/.style={
    for tree={
      math content
    },
    where n children=1{
      !1.before computing xy={l=\baselineskip},
      !1.no edge
    }{},
    closed/.style={
      label=below:$\times$
    },
  },
}
%%%%%%%%%%%%%%%%%%%%%%%%%%%%%%%%%%%%%%%%%%%%%%%%%%%%%%%



%%%%%%%%%%%%%%%%%%%%%%%%%%%%%%%%%%%%

\addbibresource{ref.bib}
\def\BibTeX{{\rm B\kern-.05em{\sc i\kern-.025em b}\kern-.08em
    T\kern-.1667em\lower.7ex\hbox{E}\kern-.125emX}}





% \title{ \Huge \textbf{Yerbamaté: A Modular and Open Science Convention for Python-based AI Projects} \\[0.5cm]}

\title{ \Huge \textbf{Yerbamaté: An Open Science Python Framework} \\[0.5cm]}

\author{
\begin{tabular}{ll}
Ali Rahimi
\end{tabular} \bigskip \\

\textit{Maastricht University} \\
\textit{Department of Advanced Computing Sciences}\\
\textit{Maastricht, The Netherlands}\\

}
\date{January 2023}
\addbibresource{ref.bib}


\begin{document}

\maketitle



\begin{abstract}
% This paper introduces Yerbamate\footnote{\url{https://github.com/oalee/yerbamate}}, an open science\footnote{To promote the principles of open science, the progress history of this paper is available on GitHub at the following URL: \url{https://github.com/oalee/os-yerbamate}} framework and software engineering convention for Python-based AI research projects. Yerbamate prioritizes modularity and separation of concerns to promote reproducibility, reusability, shareability, maintainability, and code quality. The convention and framework can be applied to a variety of Python-based tasks, including experimentation with machine learning, deep learning, genetic algorithms, optimizations, and simulations and analysis. Yerbamate simplifies the implementation of the convention with straightforward commands for installing, experimenting, and running code. The framework is compatible with all Python libraries and AI/ML frameworks, and it fosters open science by this convention enabling sharing of code modules such as models and trainers between practitioners. The adoption of this convention and framework can enhance collaboration, help address the reproducibility crisis in AI and enhance the developer experience while promoting open science and accessible AI.

This reports presents Yerbamate\footnote{\url{https://github.com/oalee/yerbamate}}, an open science\footnote{
As an open science work, Yerbamate strives to promote the principles of transparency and collaboration. To this end, the history of the LaTeX files for work are available on GitHub: \url{https://github.com/oalee/os-yerbamate}. These open science repositories are open to collaboration and encourage participation from the community to enhance the validity, reproducibility, accessibility, and quality of this work.
} framework designed for Python-based projects including Artificial Intelligence (AI). Yerbamate emphasizes modularity and separation of concerns to promote reproducibility, reusability, shareability, maintainability, and code quality. The framework is applicable to a range of Python-based tasks, including experimentation with machine learning, deep learning, genetic algorithms, optimizations, and simulations and analysis. Yerbamate simplifies the experimentation process with a user-friendly installation procedure, intuitive commands for running experiments, and streamlined execution of code. The framework is compatible with all Python libraries and frameworks, and it fosters open science by enabling the sharing of code modules such as models and trainers between practitioners. The adoption of Yerbamate can facilitate collaboration, enhance the developer experience, and promote open science while contributing to addressing the reproducibility crisis in the field of AI.
\end{abstract}





% This paper presents a software engineering convention, accompanied by a framework named Yerbamate, for Python-based artificial intelligence (AI) projects. The convention emphasizes modularity and separation of concerns, with the goal of promoting reproducibility, reusability, shareability, maintainability, and code quality. The use of modular code structures enables flexibility and scalability, facilitating the reuse of code and its components. The Yerbamate framework simplifies the implementation of the convention by providing a set of simple and consistent commands for installing, experimenting, and running code. The convention and framework are compatible with all Python libraries, AI and ML frameworks, and can be used for a variety of Python-based experimentation, analysis, ML, AI, genetic algorithms, particle swarm optimization, and simulations. The use of this convention and framework can address the reproducibility crisis in AI and promote open science by enabling the sharing of code modules among researchers and practitioners



%   In recent years, deep learning has emerged as a powerful tool for solving complex problems in various fields, such as image and speech recognition, natural language processing, and computer vision (\cite{lecun2015deep}). 
% Deep learning projects, whilst still considered software engineering, differ from traditional software engineering in that they revolve around the creation of models that can extract knowledge from data and make predictions or decisions, as opposed to relying on a pre-defined set of instructions (\cite{lecun2015deep,amershi2019software,wan2019does,se4dl}). These disparities make some phases such as the testing phase of deep learning projects distinct from traditional software (\cite{wan2019does}). Still, principles and design patterns commonly used in software engineering can still aid in developing and maintaining deep learning software (\cite{amershi2019software,wan2019does,se4dl}). 

% Artificial intelligence (AI) and machine learning (ML) have become increasingly powerful in recent years, with applications in fields ranging from natural language, art, music, healthcare, finance to education. While these technologies have the potential to revolutionize many areas of human life, their successful development and deployment depend heavily on quality code and sound software engineering practices. Good software engineering practices can increase the reliability, maintainability, and scalability of AI and ML systems, enabling their widespread adoption and use.

\section{Introduction}


 In recent years, AI has made remarkable progress, finding numerous applications across various fields such as natural language (\cite{gpt,gato}), art (\cite{diffusion}), music (\cite{musiclm}), healthcare (\cite{aihealthcare}), finance (\cite{bao2022fraudartificial}), education (\cite{aieducation}), and weather forecasting (\cite{weather}). The potential benefits of AI are immense (\cite{beneficialai,potencialaibenefit}), their development and deployment require quality data (\cite{lecun2015deep}), code, and sound software engineering practices (\cite{se4dl,amershi2019software}). Poor software engineering practices can lead to a reproducibility crisis, undermining the reliability and credibility of AI systems (\cite{leakage-recrisis}). On the other hand, good software engineering practices can enhance the developer experience, reliability, maintainability, and replicability of AI systems (\cite{se4dl,amershi2019software, wan2019does}). In order to realize the full potential of AI, researchers and practitioners require tools and frameworks that are user-friendly, customizable, and promote reproducibility and openness in the development of AI projects (\cite{lu2022softwareAIReponse,li2018can,wolf2020designing,olson2018system,ong2021guide,gundersen2018reproducible}). This is particularly important for ensuring that the benefits of AI are accessible to all and that the development of AI aligns with open science and ethical principles, enabling the sharing of research findings and supporting the progress and implementation of diverse AI applications for societal benefits (\cite{coro2020open,braun2018open, mittelstadt2016ethics,floridi2018ai4people,ong2021guide}).

% In recent years, AI has made remarkable progress, finding numerous applications across a wide range of fields, including natural language processing, art, music, healthcare, finance, education and weather forecasting. These technologies have the potential to revolutionize many areas of human life, leading to significant societal and economic benefits. However, their successful development and deployment depend heavily on quality data, code, and sound software engineering practices. Poor software engineering practices can result in the reproducibility crisis in AI research, negatively affecting the credibility and reliability of AI systems. Good software engineering practices, on the other hand, can increase the reliability, maintainability, and replicability of AI systems, enhancing the developer experience. Developers and researchers in the field of AI require tools and frameworks that are user-friendly, customizable, and provide a seamless development experience. These tools and frameworks should promote, reproducibility, and openness in the development of AI projects, facilitating the sharing of research findings and supporting the implementation of diverse AI applications.

% In recent years, artificial intelligence (AI) and machine learning (ML) have made remarkable progress and found numerous applications across a wide range of fields, including natural language processing, art, music, healthcare, finance, and education. 
% While these technologies have the potential to revolutionize many areas of human life, their successful development and deployment depend heavily on quality data, code and sound software engineering practices. Good software engineering practices can increase the reliability, maintainability, and replicability of AI systems, and enhance the developer experience. 

% In this paper, we propose a software engineering convention for AI and ML that emphasizes modularity, and separation of concerns, promoting reproducibility, maintainability, and code quality. This convention enhances open science by enabling the sharing of standalone code modules, such as models, trainers, and data loaders, among researchers and practitioners. By promoting modular design, this convention allows for increased flexibility and reusability of code, making it easier to reproduce and build upon previous research.

% In this paper, we present a software engineering convention of modularity that enhances reproduciblility, maintainability, code quality, and enhances open science through the sharing of standalone code modules such as models, trainers, and data loaders between researchers and practitioners. 

% The proposed convention addresses the challenges associated with the current state of software engineering practices in the AI and ML fields and has the potential to facilitate the growth and adoption of new technologies.
% However, the current state of software engineering in the AI and ML fields leaves much to be desired. Many AI and ML systems are developed in an ad hoc manner, with limited attention to testing, documentation, version control, and other fundamental software engineering practices. This lack of attention to good software engineering can result in buggy, hard-to-maintain systems that invalidate research and impede progress. Furthermore, the lack of standardization in the AI and ML fields can make it difficult for researchers to collaborate and share their work effectively.


\section{Motivation}

Despite the significant progress in AI in recent years, the lack of standardized software engineering practices for these fields remains a significant challenge. The absence of a convention and lack of a software engineering skills for AI  has led to several issues, including the presence of bugs and invalidation of research (\cite{leakage-recrisis,epskamp2019reproducibilitybug, seAIsurvey, martinez2022softwareAI}), making maintenance difficult (\cite{mainatiblity}), and hindering the wider adoption of AI systems. Open source research projects typically create their command line tools to experiment with different models and hyperparameters, which can lead to comprehensive options but often result in limitations. This re-implementation of hyperparameter selection and experiment execution leads to reinvention of the wheel and a potential learning curve.

This heterogeneity in software engineering practices poses a significant obstacle to code reuse and collaboration among researchers (\cite{davis2011understandingmodularity}). In some cases, AI codes are of low quality, resembling spaghetti code that is hard to maintain (\cite{seAIsurvey,martinez2022softwareAI,amershi2019software,mainatiblity,leakage-recrisis,gezici2022systematicsoftware}). In contrast, others exhibit a more modular design, which can enhance code quality and maintainability (\cite{seAIsurvey,martinez2022softwareAI,wan2019does}). Moreover, the problem arises when attempting to use another researcher's model, data augmentation, or a specific approach, as it requires learning a new framework for hyperparameter configuration and execution. The lack of standardized software engineering practices increases the difficulty of reproducing and building upon previous research.
% To address these issues, we present a software engineering convention of modularity that enhances maintainability, code quality, and enables the sharing of models, trainers, data loaders, and code components between researchers and practitioners.

% Our research suggests that establishing standard software engineering practices for AI  can enhance collaboration and sharing between researchers and practitioners. This will make it easier to replicate and build upon previous work, leading to faster innovation and progress in the field. By adopting good software engineering practices, AI  systems can become more reliable, maintainable, and scalable, enabling their widespread adoption and use.


% The widespread adoption of the Python programming language in AI programming has been noted in recent studies (\cite{mihajlovic2020use,raschka2020machine}). As such, this project focuses on the examination of software engineering practices in the context of Python and its machine learning frameworks. Many interdisciplinary researchers may lack the software engineering expertise necessary to manage their code bases to ensure correctness, understandability, extendability, and reusability (\cite{amershi2019software,scully-debt-ml,leakage-recrisis,accountabilityInAi}).
% To address these challenges, we investigated the application of software engineering principles of separation of concerns (SoC) and modular design in Python to AI projects. The goal of this study was to improve the reproducibility and reusability of python AI components through the adoption of best software engineering practices. Software engineering is the process of designing, developing, and maintaining software systems efficiently and reliably (\cite{pressman2010software}).
% In this internship, I explored software engineering practices that could help researchers share and collaborate AI code.


\section{Background Information}


\subsection{Crisis of Reproducibility in AI}
The crisis of reproducibility in AI refers to the difficulty in reproducing the results of AI research (\cite{gundersen2018reproducible}). The lack of transparency in data collection and research has greatly contributed to the crisis of reproducibility in AI (\cite{gundersen2018reproducible,hutson2018artificial,leakage-recrisis}). Many AI models are developed close sourced using proprietary data and methods, making it difficult for others to replicate the research and understand the inner workings of the models (\cite{gundersen2018reproducible,accountabilityInAi}). Additionally, the lack of transparency in the research process can lead to issues such as unreliable or biased methods data, which can further undermine the credibility and reproducibility of the research, and it can decrease the trust in the field as the results of the research are not independently verifiable  (\cite{accountabilityInAi,leakage-recrisis,scully-debt-ml}). The pressure to publish results and the lack of incentives to share data and code can discourage researchers from making their work easily reproducible. (\cite{psychology-reproducibility-crisis, friesike2015open,kwon2021incentive, ali2017motivating,o2017evaluation})



\subsection{Open Science}

Open science is a research methodology that prioritizes transparency, collaboration, and reproducibility (\cite{nielsen2011reinventing}). The promotion of open science in the field of AI has garnered considerable attention in recent years (\cite{accountabilityInAi,gundersen2018reproducible,leakage-recrisis,scully-debt-ml,stodden-towardreprodicibleresearch,coro2020open,braun2018open,hicks2021open,burgelman2019open}). In the field of AI, open science practices can help to address concerns about biased or unreliable data, as well as provide a way for researchers to collaborate and build reproducible research and enhance accountability in AI (\cite{accountabilityInAi,stodden-towardreprodicibleresearch}).  Open science encourages researchers to share their knowledge, data, code, and detailed documentation of their methods (\cite{hutson2018artificial,accountabilityInAi}). 
Researchers can also use open-source frameworks and standard evaluation metrics (\cite{gundersen2018reproducible}) to facilitate reproducibility. Furthermore, the scientific community can encourage reproducibility by valuing it in the peer-review process (\cite{scully-debt-ml}), and by giving credit to researchers who share their data and code (\cite{scully-debt-ml,credit-datasharing,stodden-towardreprodicibleresearch}).




\subsection{Fair Comparison of Models in AI}

Data is a crucial factor in the success of deep learning models (\cite{lecun2015deep}). The quality, pre-processing, and augmentations applied to the data can significantly impact the model's ability to extract knowledge and make accurate predictions (\cite{shorten2019survey}). Therefore, it is essential for researchers to consistently use the same data split, pre-processing and augmentations when comparing models to ensure fair comparisons (\cite{caton2020fairness,mehrabi2021survey, leakage-recrisis}).


\subsection{The FAIR Guiding Principles for scientific data management and stewardship }
The FAIR Guiding Principles for scientific data management and stewardship were introduced in 2016 as a guideline for improving the findability, accessibility, interoperability, and reusability of digital assets, with a focus on machine-actionability to enable computational systems to find, access, interoperate, and reuse data with minimal human intervention \cite{wilkinson2016fair}. The principles have been endorsed by a diverse set of stakeholders, including academia, industry, funding agencies, and scholarly publishers. The principles emphasize the importance of metadata in making data and digital assets easy to find, with machine-readable metadata being a crucial component of the FAIRification process \cite{wilkinson2016fair}. In addition to findability, the principles also address the accessibility of data, including the need for authentication and authorization for accessing digital assets \cite{wilkinson2016fair}. Interoperability is another critical aspect of the principles, with data needing to be integrated with other data and interoperable with applications or workflows for analysis, storage, and processing \cite{wilkinson2016fair}. The ultimate goal of the FAIR principles is to optimize the reuse of data, and well-described metadata and data are essential for data replication and reuse in different settings \cite{wilkinson2016fair}. The FAIR principles refer to three types of entities, including data or any digital object, metadata, which is information about the digital object, and infrastructure \cite{wilkinson2016fair}.

\subsection{The Significance of Open Science for AI}

 
Open science represents a crucial component in the pursuit of responsible and trustworthy AI (\cite{floridi2019establishing,coro2020open,braun2018open,hicks2021open}). By prioritizing transparency and reproducibility, researchers in the field can advance its development in a safer and trustworthy manner (\cite{coro2020open,floridi2018ai4people,kocak2022transparency,stodden-towardreprodicibleresearch}).
Open science practices serve to mitigate the risks associated with closed-source AI and big data bias, which have raised significant concerns among stakeholders (\cite{batarseh2020data, o2017weapons}). Adoption of open science and open source AI can increase the fairness and impartiality of AI models (\cite{stodden-towardreprodicibleresearch,accountabilityInAi,gundersen2018reproducible}) and enhance the credibility and trustworthiness of their outputs among stakeholders and decision-makers (\cite{goodman2017european,hsiao2018vtaiwan,praprotnikevaluation}). % 

\subsection{Open Source}
Open source is a development methodology that emphasizes collaboration, transparency, and community involvement. It involves the sharing of software code and the ability to view and modify that code, as opposed to proprietary or closed-source software where the code is not available for viewing or modification. Open source has been a driving force behind many technological advancements, with the internet itself being built on open source software such as Linux..
In the field of artificial intelligence, open source development has also played a significant role. Many popular machine learning and deep learning frameworks and libraries, such as Torch, Keras, Hugging Face, and Jax, are open source, allowing for a wider range of use cases, contributions from the community, and access to cutting-edge research. The open-source nature of these frameworks also enables a more transparent and collaborative development process.


\subsection{Software Engineering}
Software engineering is a well-established discipline that encompasses the process of designing, developing, testing, and maintaining software systems with a focus on quality, reliability, and efficiency \cite{pressman2010software}. While the specific activities and methodologies involved in software engineering can vary depending on the type of software, the principles of good software engineering practices are generally applicable across all types of software (\cite{pressman2010software}), including those in the field of artificial intelligence (\cite{se4dl,wan2019does,martinez2022softwareAI,davis2011understandingmodularity}). The software engineering practices employed in the development of AI systems include, but are not limited to, testing, debugging, documentation, version control, and code review. Additionally, given the complex and evolving nature of AI systems, specific attention must be given to software requirements and their evolution over time (\cite{heyn2021requirement,belani2019requirements}). However, unlike traditional software engineering, the requirements of AI systems may not always be well-defined, and software engineering practices may need to be adapted to the rapidly changing needs of these systems (\cite{heyn2021requirement,belani2019requirements}). 


\subsubsection{Separation of Concerns}
 The separation of concerns (SoC) is a software engineering principle that suggests that different aspects of a system should be separated into distinct components, allowing for increased clarity, maintainability, and scalability of code (\cite{pressman2010software, de2002importance}). 
 In the context of AI, this principle can be applied by separating different concerns of a AI into reusable components. By adhering to SoC, researchers can improve the clarity of their code and reduce the risk of introducing bugs and errors (\cite{mo2016decoupling,mo2016decoupling,pressman2010software, de2002importance}).

\subsubsection{Modularity}
Modularity, or the practice of creating reusable components, is a fundamental aspect of software engineering (\cite{pressman2010software}). By breaking down complex systems into smaller, reusable components, researchers can improve the understandability and extendability of their code. Additionally, modular code is more easily testable and maintainable, leading to increased reproducibility and reliability of results (\cite{amershi2019software,pressman2010software}). 
\subsubsection{Modualirty in Python}
In Python, modular design can be achieved through the use of functions, modules, and libraries (\cite{sanner1999python}). 
A python module contains definitions, functions, classes, and variables (\cite{raschka2015python}). By convention, modules are stored in separate directories, and a directory containing one or more modules is called a package. The presence of \verb|__init__.py| file in a package directory indicates that it is a package, and all files in the directory are considered modules of that package. In other words, the \verb|__init__.py| file makes the directory it's in a Python package, and any code in that file is executed when the package is imported.



\subsection{Developer Experience}
The concept of developer experience (DX) is a multidimensional construct that refers to developers' perceptions of various aspects of the development process, including the usability and effectiveness of the tools, frameworks, and platforms used (\cite{fagerholm2012developer}). DX is a critical factor in software development as it has the potential to impact productivity, motivation, and satisfaction (\cite{fagerholm2012developer}).



\subsection{Decoupling Concerns in Writing AI Code}


Decoupling concerns, also known as SoC, is a crucial design principle in the field of artificial intelligence and machine learning (\cite{mo2016decoupling,qian2006decoupling, pressman2010software}. For instance, in a typical deep learning experiment, the trainer component is responsible for training the model. The trainer component can be designed to receive either a string representing the dataset/model names or the actual dataset/model objects, with the latter approach providing greater flexibility and customization. Another example of separation of concerns in AI is the data loading and augmentation process, which can be hardcoded into the data loading module or passed as an object to a function. Similarly, a deep learning model can be implemented as a monolithic block of code or as a series of modular components, such as the encoder, decoder, and attention mechanism. The latter approach allows for greater flexibility and customization of the model, as each component can be modified or replaced without affecting the other components.



\section{Related Work}

In recent years, there has been an increasing demand for project structure templates to provide guidance and organization to data science projects. This trend has been towards modularization and separation of concerns to make projects more manageable and easier to maintain. This has been exemplified by frameworks like Cookiecutter, which separates the codebase into four main concerns, and Towards Data Science, which provides a comprehensive structure for data science projects. 

\subsection{Coockiecutter}
% Cookiecutter is a popular tool that helps in setting up project directory structures and boilerplate code for Python-based projects. It provides a predefined directory structure, which includes nootbooks, reports, figures, references, license, and others, to help developers with setting up a consistent project directory structure. In contrast to Yerbamate, Cookiecutter separates the src directory into four concerns and modules: data, models, notebooks, and src, while Yerbamate creates an arbitrary number of independent modules. The predefined structure of Cookiecutter helps with separating out the different aspects of a project, such as data, models, and code, into their own directories. This can help make a project more organized and easier to navigate. However, the modularity provided by Yerbamate allows for more flexibility in how a project is structured, as it allows for the creation of independent modules without being restricted to a predefined structure.
Cookiecutter is a popular tool for setting up project directory structures and boilerplate code for Python-based projects. It provides a predefined directory structure that includes notebooks, reports, figures, references, license, and others to help developers set up a consistent project directory structure. The Cookiecutter project directory structure separates the src directory into four predefined modules: data, features, models, and visualization, providing a clear separation of concerns. In contrast, Yerbamate allows for the creation of an arbitrary number of independent modules, making it more flexible and easier to manage and reuse code in different projects. Yerbamate can be integrated into the Cookiecutter project structure, managing the src module of the project, and merging the design of license, docs, notebooks, and references. 



\subsection{Software Engineering Practices For Accountable AI}

Recent research has highlighted the significance of software engineering practices for developing transparent and accountable AI systems. "Towards Accountability for Machine Learning Datasets: Practices from Software Engineering and Infrastructure" proposes a comprehensive guide for documenting the process of dataset development to ensure accountability and mitigate dataset risks. The dataset development lifecycle includes five stages: requirements analysis, design, implementation, testing, and maintenance. The authors recommend creating a comprehensive set of documentation artifacts at each stage, with clearly designated owners and responsibilities to ensure accountability. The documentation aims to create a clear and transparent understanding of the dataset, its development, and its intended uses. The authors also provide a flexible template for documenting dataset requirements that covers a broad range of factors. The proposed documentation guideline can be integrated into the Yerbamate framework, which emphasizes transparency, reproducibility, reusability, and collaboration in the research community.

% Recent research has highlighted the significance of software engineering practices for developing transparent and accountable AI systems. The paper "Towards Accountability for Machine Learning Datasets: Practices from Software Engineering and Infrastructure" proposes a comprehensive guide for documenting the process of dataset development to ensure accountability and mitigate dataset risks. The dataset development lifecycle proposed in this work includes five stages: requirements analysis, design, implementation, testing, and maintenance. To ensure accountability, the authors recommend creating a comprehensive set of documentation artifacts at each stage, with clearly designated owners and responsibilities. The primary purpose of the documentation is to create a clear and transparent understanding of the dataset, its development, and its intended uses. The authors also provide a flexible template for documenting dataset requirements that covers a broad range of qualitative and quantitative factors. The resulting documentation guidline can get integrated with the yerbamate framework.

% Recent works have emphasized the importance of software engineering practices for developing accountable and transparent AI systems. In this regard, the paper "Towards Accountability for Machine Learning Datasets: Practices from Software Engineering and Infrastructure" presented a comprehensive guide for documenting the process of dataset development to mitigate dataset risks and ensure accountability.
%  presents a dataset development lifecycle consisting of five stages: requirements analysis, design, implementation, testing, and maintenance. To ensure accountability, the authors propose a comprehensive set of documentation artifacts to be created at each stage, with clearly designated owners and responsibilities. The primary purpose of the documentation is to create a clear and transparent understanding of the dataset, its development, and its intended uses. The authors also suggest a flexible template for documenting dataset requirements that covers a broad range of qualitative and quantitative factors. This framework offers a new approach to addressing the increasing demand for accountability in machine learning, allowing for greater transparency and responsible use of data.
%  The results of the Software Engineering Practices for Accountable AI paper can be integrated with the Yerbamate framework, which emphasizes reproducibility, reusability, transparency, and collaboration in the research community, and enhances developer experience.
% As AI systems become increasingly integrated into critical systems, ensuring their accountability and transparency becomes a pressing issue. The paper "Towards Accountability for Machine Learning Datasets: Practices from Software Engineering and Infrastructure" proposes a set of best practices to mitigate dataset risks and improve the accountability and transparency of the development process. The paper suggests adopting a deliberate and intentional methodology throughout the dataset development lifecycle, emphasizing documentation practices, diverse oversight processes, and robust maintenance mechanisms. The proposed practices facilitate reviews and audits, providing bounds on accountability both in dataset creation and in the ML systems that depend on those datasets as infrastructure. The authors emphasize the importance of establishing a non-linear cycle of dataset development, with analysis, design, and evaluation central to the process. The proposed practices replace abductive and posteriori reasoning about dataset provenance with careful documentation and understanding of the limitations of the dataset. The paper's emphasis on accountability and transparency aligns with current efforts to improve AI's social and ethical implications, promoting a more responsible and effective use of AI technology.

% "Towards Accountability for Machine Learning Datasets: Practices from Software Engineering and Infrastructure" is a research paper that explores the importance of accountability in machine learning datasets, as well as the practices and strategies that can be adopted from software engineering and infrastructure to achieve it. The paper highlights the critical role that datasets play in machine learning, serving as infrastructure and knowledge construction, and emphasizes the need for intentional and deliberate methodology in their development to avoid biases and other issues. The authors propose a set of documentation practices throughout the dataset development lifecycle and identify diverse oversight processes and maintenance mechanisms for ensuring the dataset's ongoing quality and relevance. By drawing on the lessons and experiences from software engineering, the authors aim to provide a framework for promoting accountability in machine learning datasets and improving the overall fairness and trustworthiness of AI systems.

% O
% There have been several efforts to establish best practices for software engineering in AI, including the "Towards Accountability for Machine Learning Datasets: Practices from Software Engineering and Infrastructure" publication which emphasizes the importance of open science principles in the development of AI datasets and the documentation process for researcher collaboration (\cite{accountabilityInAi}). These efforts aim to establish a more standardized and transparent approach to AI development, ultimately leading to more trustworthy and reliable AI systems.


\section{Research Questions}

% The objective of this internship at the start was to address the following research questions.

% \subsection{
% What should be the recommended conventions for software engineering in AI and machine learning to promote reproducibility, reusability, shareability, maintainability, and code quality?
% }
\subsection{
    What are the most effective software engineering practices for promoting reproducibility and open science in AI research?}
    Effective software engineering practices for promoting reproducibility and open science in AI and ML research include modularity, separation of concerns, documentation, version control, and testing. Modularity enables the reuse of code and components, while separation of concerns separates code into distinct modules based on their functionality, making it easier to maintain and modify. Documentation helps other researchers understand how to use and reproduce the code, while version control enables tracking of changes to the code over time. Testing helps ensure that the code works as intended and can be used by others. These practices can help promote open science by enabling the sharing of code and promoting transparency in research.
    
    
    
\subsection{
How can software engineering principles in AI/ML be designed to improve the developer experience?
}

Software engineering best practices in AI/ML can be designed to improve the developer experience by prioritizing modularity, separation of concerns, and standardized naming conventions for independent and interchangeable modules such as models, data, and trainers. This approach allows for a standardized approach to project organization, code structure, and modularity, which can facilitate collaboration and code sharing among researchers and practitioners. Additionally, providing clear documentation and support resources can help users understand and adopt the convention, leading to an improved developer experience. Furthermore, the use of open-source technologies and practices can promote collaboration and sharing of code and models within the scientific community, leading to the development of more efficient and effective AI/ML research practices.
% Software engineering conventions in AI/ML can be designed to improve the developer experience and promote open science practices by prioritizing modularity, separation of concerns, and consistent naming conventions for independent and interchangeable modules such as models, data, and trainers. To ensure compatibility and customizability, the convention should be flexible enough to accommodate a wide range of use cases and project types. Clear documentation, training materials, and support resources can facilitate adoption of the convention, while open-source technologies and practices can foster collaboration and sharing of code and models within the scientific community. Additionally, the ability to upgrade the convention with input from the community can ensure that the convention remains relevant and adaptable to new technologies and emerging research practices. By adopting a well-designed software engineering convention, AI/ML developers can improve the efficiency and effectiveness of their research, while promoting open science practices and facilitating collaboration within the research community.


\subsection{What are the key tools and technologies needed to support open science, and how can they be made more accessible to researchers and the wider scientific community?}
The key tools and technologies needed to support open science include open source frameworks, open-access journals, repositories, data sharing platforms, collaborative tools for writing, documentation and version control, and open source software and hardware. These tools can help to promote transparency, reproducibility, and collaboration in scientific research, and can contribute to the development of a more open and accessible research environment. To make these tools more accessible to researchers and the wider scientific community, it is important to provide clear documentation, training materials, and support resources, as well as promoting interdisciplinary collaborations and knowledge sharing. Additionally, adopting open science practices can require a cultural shift in the scientific community, and it is important to engage stakeholders and the public in discussions around the benefits and challenges of open science. By promoting open science and making key tools and technologies more accessible, we can help to foster a more collaborative and innovative research environment.
    
% \subsection{
% In what ways does the Yerbamate framework align with the principles of open science, how does it support collaboration, and how can its features contribute to the development of more transparent and reproducible AI and ML research?
% }
% The Yerbamate framework aligns with the principles of open science in several ways. It promotes transparency and reproducibility by providing a set of consistent tools and practices that allow for the creation, sharing, and experimentation of machine learning and artificial intelligence models. The framework's modular design allows for easy customization and extension, making it simple for researchers to build upon existing work and collaborate with others. Additionally, the framework's collaboration features, such as out-of-the-box sharing of models via GitHub URLs, facilitate easy sharing and reuse of code, enabling researchers to build on each other's work and contribute to a more open and collaborative scientific community. By combining these features, the Yerbamate framework provides a powerful tool for advancing open science practices in the field of AI and ML research.
% \subsection{
%     How can we establish a set of conventions for software engineering in AI and ML that are widely accepted and easy to adopt?}
%     Establishing widely accepted and easy to adopt software engineering conventions for AI and ML is a complex task that requires input and collaboration from multiple stakeholders, including researchers, practitioners, and software engineers. One approach is to focus on the principles of modularity, separation of concerns, and documentation, which can help to promote flexibility, reusability, and maintainability of code. The use of independent modules that only depend on python dependencies and the code inside the module uses relative imports is a simple and effective way to promote modularity. This convention can be supplemented with best practices for documentation, including standardized naming conventions and clear documentation of code functionality, assumptions, and limitations.
    
    % \subsection{
    % How can modularity be utilized to promote open science practices in the field of artificial intelligence and machine learning?
    % }
    
    % \subsection{
    % Can conventions in AI and ML improve the learning experience for developers and enhance the efficiency and effectiveness of AI development?
    % }
    % Establishing conventions for AI and ML development can lead to an improved developer experience, resulting in greater efficiency and effectiveness in the development process. Conventions that prioritize modularity, separation of concerns, and consistent naming conventions can help to make the code more maintainable and reusable. This can save developers time in the long run, as they can easily incorporate existing code and components into new projects without having to start from scratch. Additionally, established conventions can make it easier for developers to collaborate and share code with others, ultimately contributing to a more open and collaborative scientific community. Overall, the adoption of conventions in AI and ML can lead to a more efficient and effective development process, ultimately enhancing the developer experience.
    
    % \subsection{
    % How can the adoption of software engineering conventions in AI and ML improve the developer experience and enhance the efficiency and effectiveness of AI development?
    % }
    % The establishment of conventions for software engineering in AI and ML can improve the developer experience by providing a standardized approach to project organization, code structure, and modularity. Such conventions can facilitate collaboration and enable code sharing among researchers, practitioners, and the wider scientific community. 
    % In the case of the Yerbamate framework, the use of independent modules and a non-independent experimentation module provides a clear separation of concerns and enables ease of use for experimentation, training, and validation of models. The Yerbamate framework provides a set of simple and consistent commands for installing, experimenting, and running code, which can lead to an improved developer experience for AI. Additionally, the framework includes an out-of-the-box sharing feature that allows users to share their models via GitHub URLs. By making it easy for users to share and reuse existing code and models, the framework promotes collaboration and open science practices in AI development.
    
%     \subsection{
%   What are the benefits and challenges of adopting a software engineering convention for AI and ML research, and how can we ensure that the benefits outweigh the costs while promoting open science and fostering a culture of collaboration and sharing within the research community?
%     }
%     Adopting a software engineering convention for AI and ML research can lead to significant benefits, including improved code quality, better maintainability, and enhanced reproducibility. However, there can be a learning curve associated with adopting a new convention, particularly for researchers who may not have extensive software engineering backgrounds. The challenge is to develop a convention that is easy to learn and widely accepted. Modularity can be an effective approach, as it allows for the gradual refactoring of existing code into independent, reusable modules, which can lead to an improved developer experience over time. To ensure that the benefits of adopting a convention outweigh the costs, it is essential to provide clear documentation, training materials, and support resources to facilitate the learning process. Additionally, it is important to foster a culture of collaboration and sharing within the research community, which can help to promote the widespread adoption of best practices and ensure that the benefits of improved software engineering practices are shared across the field.



% \subsection{How can a software engineering convention for AI/ML be designed to ensure compatibility and customizability, and what are the benefits of such a convention for open science and collaborative research?}
% The convention should prioritize modularity, separation of concerns, and consistent naming conventions for independent and interchangeable modules such as models, data, and trainers. The convention should also be flexible enough to accommodate a wide range of use cases and project types. Clear documentation, training materials, and support resources can help users understand and adopt the convention, and open-source technologies and practices can facilitate collaboration and sharing of code and models within the scientific community.

% \subsection{How can modular software engineering conventions promote the reusability and scalability of AI and ML systems?}

% \subsection{Can a standard format be designed for data loaders and preprocessing pipelines?}
% {
% The development of a universal standard format for data loaders and preprocessing pipelines in deep learning is challenging due to the framework-specific nature of the task. Preprocessing pipeline design is highly dependent on the specific problem and its data, making it difficult to create a general solution. However, Separation of concerns (SoC) and modularity principles can enhance code reusability by creating the data loading as an independent python module, and separating preprocessing pipeline from data loading. This project focused on making data loading and code components, such as data loaders and the preprocessing pipelines, sharable within a python machine learning framework.
% }

% \subsection{Can this be put in a high-quality, easily accessible database for local access?} {
% The creation of a high-quality, easily accessible database storing preprocessed datasets and related components, such as data loaders and preprocessing pipelines, is complex and involves challenges in terms of data processing, storage, and retrieval, which were beyond the scope of the project. Limitations such as credibility, scalability, compatibility with different deep learning frameworks, privacy, and storage pose additional challenges. This project focused on designing a standard for sharing code-related components, instead of making data available, for feasibility of the task.
% }

% \subsection{Can the database be made easily expandable?} {
% Expanding the database of deep learning code components, such as data loaders, relies on adopting software engineering principles of separation of concerns (SoC) and modularity. Adhering to these principles makes code components shareable across projects within the same framework, which can get added to a database and scale in time.
% }



\section{Methodology}

% The widespread adoption of the Python programming language in AI programming has been noted in recent studies (\cite{mihajlovic2020use,raschka2020machine}). As such, this study focused on the examination of python projects and 
This study aims to address the absence of software engineering knowledge in AI projects by developing a framework that prioritizes open science and improves the developer experience. 
% The methodology involved analyzing open source deep learning projects and reviewing literature on software engineering practices, modularity, separation of concerns, and open science in AI. The resulting convention is simple, effective, and applicable to a wide range of AI projects.


The framework has been designed with the aims of promoting modularity, separation of concerns, consistent naming conventions to improve the maintainability and reusability of code in mind. The methodology for developing this framework involved analyzing existing open-source AI projects, reviewing the literature on software engineering practices, and studying modularity, separation of concerns, and open science in AI. The resulting framework prioritizes the creation of independent modules that are standalone and reusable components, which can be used across different projects.
% The convention is designed to prioritize modularity, separation of concerns, with consistent naming conventions to promote maintainability and reusability of code. It places a high priority on creating independent modules that are standalone and reusable components, which can be used across different projects. 



To evaluate the framework, it was tested with a variety of open-source AI projects across different frameworks and libraries, including Jax, Flax, Pytorch, Lightning, Tensorflow, Keras, Darts, and Huggingface. To further demonstrate the flexibility, the framework was tested in various use cases, particle swarm optimization, computer vision experiment with ransac.
% In some cases, only the "experiment" module was necessary.

A case study was conducted to evaluate the framework's effectiveness in practice. The official implementation of Big Transfer (BiT)  (\cite{transferlearning}) was refactored to conform to modularity. The resulting code was evaluated for shareability, extendability and overall quality, with findings presented in Appendix A.
% This study was conducted to develop a software engineering convention for AI and ML projects that promotes open science practices. The development of the convention involved a review of existing literature on software engineering best practices and the identification of key principles that could be applied to AI and ML projects.

% The convention was designed to be simple, effective, and widely applicable to a range of AI and ML projects. The convention prioritized modularity, separation of concerns, and consistent naming conventions to promote the maintainability and reusability of code. The convention also included four distinct modules, namely "models", "data", "trainers", and "experiments". The "models" and "data" modules were designed to be independent and modular, while the "trainers" module was dependent on both "models" and "data". The "experiments" module was designed to combine the other modules and facilitate experimentation with different combinations of models, data, and trainers.

% To evaluate the effectiveness of the convention, it was tested with various AI and ML frameworks including Jax, Flax, PyTorch, TensorFlow, Keras, DARTS, and Hugging Face. The convention was also evaluated by looking at various open-source AI and ML projects and comparing their structure and organization to the proposed convention.

% \section{Methodology}

% This study was conducted to develop a software engineering convention for AI and ML projects that promotes open science practices. The development of the convention involved a review of existing literature on software engineering best practices and the identification of key principles that could be applied to AI and ML projects.


% \section{Methodology}

% The widespread adoption of the Python programming language in AI programming has been noted in recent studies (\cite{mihajlovic2020use,raschka2020machine}). As such, this project focuses on the examination of software engineering practices in the context of Python and its machine learning frameworks. Many interdisciplinary researchers may lack the software engineering expertise necessary to manage their code bases to ensure correctness, understandability, extendability, and reusability (\cite{amershi2019software,scully-debt-ml,leakage-recrisis,accountabilityInAi}). To address these challenges, we investigated the application of software engineering principles of separation of concerns (SoC) and modular design in Python to AI projects. The goal of this study was to improve the reproducibility and reusability of python AI components through the adoption of best software engineering practices.

% \subsection{Software Engineering for Python AI Projects}
%  In the context of AI, the implementation of modularity and SoC principles results in the separation of models, data-related components, and trainers. 
% This approach facilitates the creation of a modular design, where the code is organized into distinct, reusable components (\cite{sanner1999python,pressman2010software}). This decoupling has the potential to greatly enhance the reusability of code greatly, enabling easier experimentation with models and datasets.


% \subsection{A Different Approach to Deep Learning Framework}

% An alternative approach to the compatibility and re-usability issue in deep learning frameworks is to enforce principles of separation of concerns (SoC) and modularity in Python. 
% The implementation of modularity and SoC principles results in the separation of models, data-related components, and trainers. 
% This approach facilitates the creation of a modular design, where the code is organized into distinct, reusable components (\cite{sanner1999python,pressman2010software}). This separation enhances the reusability of the code, making it easier to experiment with different models and datasets within the same framework. This approach has the advantage of limiting the learning overhead to the principles of software engineering, rather than requiring a transition to a unified deep learning framework.



% A limitation of this approach is that it requires manual translation of AI components, such as data loaders and models, into different frameworks when necessary. While this approach may require additional effort, it allows for greater flexibility and customizability in implementing deep learning models. Additionally, it promotes a deeper understanding of the underlying software engineering principles and their application in the field of AI.




\section{Results}




% \subsection{Python Modularity Structure}
% The Structure prioritizes modularity, separation of concerns, and consistent naming Structures to promote maintainability and reusability of code. It places a high priority on creating independent modules that are standalone and reusable components, which can be used across different projects. 

% This Structure for organizing Python research projects is designed to promote modularity and separation of concerns, making research more maintainable and reusable. 

\subsection{Python Modular Project Structure}
% The Structure involves organizing the project directory in a hierarchical tree structure, with an arbitrary name given to the root project directory. The project is then broken down into separate concern modules, such as models, experiments, trainers, data, analyzers, and simulators, each with its own subdirectory. If the project does not use a framework, it can directly use this level of tree to define an independent module. Independent modules can be defined in various heights.

% Within each module, a framework can be specified, such as "jax", "torch", or "keras", followed by the name of the specific model, experiment, or trainer being used. 
% Independent submodules can be created for individual components such as data loaders, data augmentations, or loss functions. These can be organized in the following manner: \texttt{"data/torch/imagenet"} or \texttt{"data/torch/augmentations/random\_crop"} or\texttt{"models/jax/gpt2"} or \texttt{"models/torch/big\_transfer"}.
% This Structure emphasizes the importance of creating independent modules that can be used across different projects, promoting code reuse and making research more efficient. By following this Structure, researchers can make their work more accessible, reproducible, and transparent, thereby promoting open science in the field of AI.


 The modular structure of the framework involves organizing the project directory in a hierarchical tree structure, with an arbitrary name given to the root project directory by the user. The project is then broken down into distinct concerns such as models, data, trainers, experiments, analyzers, simulators, each with its own subdirectory. Within each concern, modules can be defined with their own subdirectories, such as models, trainers, data loaders, data augmentations, or loss functions. Independent modules can be defined in various heights in the tree.

The framework prioritizes the organization of the project into independent modules when applicable, however there are situations where a combination of independent modules may be necessary for a particular concern. An example of this is the experiment concern, which imports and combines models, data, and trainers to define and create a specific experiment. In such cases, the module is not independent and is designed to combine the previously defined independent modules. 
% In the case of non-independent modules. 
% Yerbamate creates a dependency list of independent modules that can be used to install the code and Python dependencies. This ensures that the necessary modules are installed and that the code can be reused. 

When implementing a project using multiple frameworks, it can become challenging to specify the exact framework used for each module. To address this issue, Yerbamaté provides a naming convention that specifies the framework used before the name of the module. For instance, the subdirectory for a Jax-based model named \texttt{"my\_model"} would be \texttt{"models/jax/my\_model"}, while the subdirectory for a PyTorch-based model with the same name would be \texttt{"models/torch/my\_model"}. This naming convention makes it easy to identify which framework was used for a specific module and ensures consistency throughout the project.

% While the prioritization of independency is encouraged, sometimes it may not be desired, and a combination of independent modules may be necessary for a particular concern. An example of this is the experiment concern, which imports and combines models, data, and trainers to create a specific experiment. In such cases, the module is not independent and is designed to combine the previously defined independent modules.



% The framework involves organizing the project directory in a hierarchical tree structure, with an arbitrary name given to the root project directory by user. The project is then broken down into separate concern modules, such as models, experiments, trainers, data, analyzers, and simulators, each with its own subdirectory. If the project does not use a framework, it can directly use this level of tree to define an independent module. Independent modules can be defined in various heights.

% Within each module, a framework can be specified, such as "jax", "torch", or "keras", followed by the name of the specific model, experiment, or trainer being used. Independent submodules can be created for individual components such as data loaders, data augmentations, or loss functions. 
% These can be organized in the following manner: \texttt{"data/torch/imagenet"} or \texttt{"data/torch/augmentations"} or\texttt{"optimizers/jax/gpt2"} or \texttt{"models/torch/big\_transfer"}.



\begin{figure}
\centering
\framebox[\0.5\textwidth]{%
\begin{minipage}{0.45\textwidth}
\dirtree{%
.1 project.
.2 data.
.3 torch.
.4 imagenet.
.4 \texttt{bit}.
.2 models.
.3 jax.
.4 transformer.
.4 gpt2.
.3 torch.
.4 \texttt{big\_transfer}.
.4 transformer.
.2 experiments.
.3 torch.
.4 bit.
.3 jax.
.4 gpt2.
.2 trainers.
.3 jax.
.4 gpt2.
.3 torch.
.4 \textt{bit}.
.2 ....
}
% For modular Python projects, organizing files into a hierarchical directory tree structure is a best practice.
\end{minipage}}
\caption{
This figure illustrates a directory tree example of a deep learning modular project. The organization of files into separated concerns in the tree is a common best practice for developing modular Python projects. By adhering to principle of independency, the submodules, such as Transformer and GPT2 can be easily shared and reused across different projects. Yerbamaté simplifies this process by providing a toolkit that allows for the injection of dependencies and the installation of modules from open-source projects.
}
\end{figure}


\subsection{Independent Python Modules}

Python independent modules are stand-alone modules that only depend on Python dependencies, such as NumPy, PyTorch, TensorFlow, or Hugging Face. The code inside the module uses relative imports to import within the module, making it an independent module that can be reused once the necessary Python dependencies are installed. This approach enhances modularity, reusability, and shareability while promoting good software engineering practices.

% Separation of concerns and organizing files into a modular hierarchical tree structure are among best software engineering practices for developing Python projects. 
% \subsection{Independent Python Modules}
% Python independent modules only depend on Python dependencies (such as NumPy, PyTorch, TensorFlow, or Hugging Face), and the code inside the module uses relative imports to import within the module, hence making it an in dependent module that can be re-used when python dependencies are installed.
% All independent python modules c
% This framework places a high priority on creating independent modules that are standalone and reusable components, which can be used across different projects. 
% This design allows the reuse of code and its components under different projects. This framework priotizes independency of modules over seperated concerns.


\subsection{No-Loop Python Experiment Definition}

This framework uses a Python experiment configuration definition that disallows the use of loops as per convention, promoting a hyperparameter-focused approach while maintaining separation of concerns. This format offers flexibility and customization, allowing the configuration format to be adapted to various AI tasks, custom use cases, frameworks, and libraries.


% \subsection{No-Loop Python Experiment Configuration Structure}

% This Structure adopts an experiment configuration with the Python language that only disallows the use of loops. The experiment configuration design choice is intended to maintain a hyperparameter-focused experiment format while promoting separation of concerns. The resulting approach provides several benefits, including increased flexibility and customization, as the configuration format can be adapted to various AI tasks, custom use cases, frameworks, and libraries. 



% Yerbamate follows a directory tree structure that is organized into distinct modules, namely "models", "experiments", "trainers", and "data", which have been identified as major concerns in AI projects across a variety of tasks. This naming convention is designed to enhance the understandability and readability of the project structure. 

\subsection{Modular Project Structure for AI}

The structure for AI projects follows a modular pattern that is organized into distinct modules, namely "models", "experiments", "trainers", and "data", which have been identified as major concerns in AI projects across a variety of tasks. This modular approach creates standalone and independent modules for models, trainers, and data, which can operate independently to enhance modularity and reusability. The "experiments" module harmonizes the three independent modules and provides a unified experiment. This naming convention is used to enhance the understandability of the project and facilitate the adoption of software engineering best practices, such as modularity and separation of concerns.

% \subsection{AI Project Structure}
% The structure for AI projects often involves four modules: "models", "experiments", "trainers", and "data". The modular approach creates arbitrary standalone modules for models, trainers, and data, each with the ability to operate independently. The experiments module harmonizes the three modules and provides a unified experiment. 
% This modular structure enables the sharing of independent models, trainers, and data modules across various projects, promoting code reusability.

% The Structure for structuring AI/ML projects involves four distinct modules: "models", "experiments", "trainers", and "data". This modular approach leads to the creation of three standalone modules for models, trainers, and data, each with the ability to operate independently. The fourth module, experiments, serves to harmonize the three modules and provide a unified and comprehensive experiment. The modular structure of independence enables the sharing of independent models, trainers, and data loaders across various projects and experiments, promoting the reusability of code.



% \begin{figure}
% \centering
% \framebox[\0.45\textwidth]{%
% \begin{minipage}{0.45\textwidth}
% \dirtree{%
% .1 /.
% .2 models.
% .3 \texttt{\_\_init\_\_.py}.
% .2 experiments.
% .3 \texttt{\_\_init\_\_.py}.
% .2 trainers.
% .3 \texttt{\_\_init\_\_.py}.
% .2 data.
% .3 \texttt{\_\_init\_\_.py}.
% .2 \texttt{\_\_init\_\_.py}.
% }
% \end{minipage}
% }
% \caption{The modular architecture of the Structure for AI is characterized by the structured folder system, separating the key components into four distinct modules: models, experiments, trainers, and data. This enhances code legibility and comprehensibility, enabling other researchers to quickly access the desired code components.}
% \end{figure}




% \subsubsection{}

% The implementation of software engineering principles of as separation of concerns (SoC) and modularity in deep learning projects results in a well-structured illustrated below. 

% \vspace{0.7em}


% The modular architecture of a deep learning project, characterized by the structured folder system, separates the key components into four distinct modules: models, experiments, trainers, and data. This enhances code legibility and comprehensibility, enabling other researchers to quickly access the desired code components. This approach leads to the formation of three standalone modules for models, trainers, and data, each possessing the ability to operate independently. Meanwhile, the fourth module, experiments, functions to harmonize the three modules and provide a unified and comprehensive experiment. The modular structure of independence further enables the sharing of models, trainers, and data loaders across various projects and experiments, thereby enhancing the code's reusability and scalability. 


% \begin{figure}
% \centering
% \framebox[\0.45\textwidth]{%
% \begin{minipage}{0.45\textwidth}


% \dirtree{%
% .1 /.
% .2 data.
% .3 cifar10.
% .4 \texttt{data\_loader.py}.
% .4 \texttt{\_\_init\_\_.py}.
% .3 cifar100.
% .4 \texttt{data\_loader.py}.
% .4 \texttt{\_\_init\_\_.py}.
% .3 \texttt{\_\_init\_\_.py}.
% .2 models.
% .3 resnet.
% .4 \texttt{fine\_tune.py}.
% .4 \texttt{resnet.py}.
% .4 \texttt{\_\_init\_\_.py}.
% .3 efficientnet.
% .4 \texttt{efficientnet.py}.
% .4 \texttt{\_\_init\_\_.py}.
% .3 \texttt{\_\_init\_\_.py}.
% .2 trainers.
% .3 classification.
% .4 \texttt{pl\_classification\_module.py}.
% .4 \texttt{\_\_init\_\_.py}.
% .3 \texttt{\_\_init\_\_.py}.
% .2 experiments.
% .3 torch.
% .4 \texttt{resnet\_cifar10.py}.
% .4 \texttt{resnet\_cifar100.py}.
% .4 \texttt{fine\_tune\_resnet\_cifar.py}.
% .4 \texttt{\_\_init\_\_.py}.
% .3 keras.
% .4 \texttt{keras\_efficient\_net\_cifar.py}.
% .4 \texttt{\_\_init\_\_.py}.
% .3 \texttt{\_\_init\_\_.py}.
% .2 \texttt{\_\_init\_\_.py}.
% }
% \end{minipage}
% }
% \caption{This figure illustration a sample folder structure for a deep learning vision project\footnote{ \url{https://github.com/oalee/deep-vision}}. Each folder represents a modular code component with a specific concern, which is demonstrated by the presence of \texttt{\_\_init\_\_.py} within the folder.}
% \end{figure}



% \subsection{Example Project Structure}


% \vspace{2em}
% Each top-level module can be divided into multiple sub-modules, allowing for the separation of individual components within each module. For instance, the models module can comprise sub-modules such as ResNet (\cite{resnet}), ViT (\cite{dosovitskiy2020vit}), CvT (\cite{wu2021cvt}), and others, implemented as independent components. Likewise, the data module can encompass sub-modules such as data loaders, preprocessing pipelines, and post-processing routines. This modular design promotes increased organization and manageability of the codebase, contributing to the efficiency and effectiveness of the deep learning project.


\subsection{Yerbamaté: An Open Science Python Framework}
% Yerbamaté\footnote{\url{https://github.com/oalee/yerbamate}} is an open-source, open-science Python framework that aims to streamline and simplify the development and management of AI projects. It promotes modular design principles and encourages quality coding, collaboration, sharing of models, trainers, data loaders, and knowledge, while also prioritizing reproducibility, customization, and flexibility. The framework provides a command-line interface (CLI) toolkit that conforms to the software engineering Structure of modularity and independence, allowing the creation and injection of dependencies for greater reproducibility and sharing capabilities. Moreover, Yerbamate is compatible with Linux systems and Jupyter notebooks, providing researchers with the ability to run experiments on Colab. By adhering to the Yerbamate Structure, any module in a project is automatically sharable.
% is a result of the study on reproducibility and the importance of modularity in AI and machine learning projects. Yerbamaté is an open source open science framework designed to streamline and simplify the development and management of artificial intelligence and machine learning projects. Built around the principles of modularity and separation of concerns, Yerbamaté provides a convenient and efficient means of adding source code and dependencies to projects.

% Modularity allows for a clean and organized codebase, making it easier to maintain, scale, and reuse the code in the future. In addition, Yerbamaté provides an easy-to-use interface for sharing the source code of models, trainers, data loaders, and experiments between modular projects, enabling greater flexibility and collaboration. Furthermore, adopting the Yerbamaté framework ensures that all projects adhering to its principles are readily compatible with Colab\footnote{\url{https://colab.research.google.com/github/oalee/yerbamate/blob/main/deep_learning.ipynb}}, a cloud-based platform widely used for machine learning experimentation and development.

% Yerbamaté also provides support for full customizability and reproducibility of results through the inclusion of dependencies in your project. The tool supports pip and conda for dependency management, making it easy to manage and install the necessary dependencies for your project.
% Additionally, Yerbamaté is fully compatible with python, and can be used with popular libraries such as PyTorch/Lightning, TensorFlow/Keras, JAX/Flax, Huggingface/transformers. Another feature of Yerbamaté is its convenient environment management through the Yerbamaté Environment API. This API allows for the creation and management of virtual environments, ensuring that each experiment is run in an isolated environment with the necessary dependencies. 
% For a comprehensive understanding of the Yerbamaté, see the documentation\footnote{\url{https://oalee.github.io/yerbamate/}}.


Yerbamate is an open-source open-science Python framework designed to streamline and simplify the development and management of AI projects using power of modular design principles and open source. The framework encourages quality coding, collaboration, and sharing of models, trainers, data loaders, and knowledge, while also promoting reproducibility, customization, and flexibility. 
Yerbamate is designed as a command-line interface (CLI) toolkit that works when the software engineering Structure of modularity and independence are adhered. The CLI can be used to create and inject dependencies, providing greater reproducibility and sharing capabilities. 
% The framework also offers an Environment API, which, along with the CLI, is presented in the Appendix and subject to updates.
Users can refer to the documentation\footnote{\url{https://oalee.github.io/yerbamate/}} for the most up-to-date material. Any module in a project adhering to the modular structure is automatically sharable out of the box, leveraging the power of open source to enhance open science and collaboration in the AI community.

\subsubsection{Flexibility and Customization}

One of the key challenges in the development of the Yerbamaté framework was ensuring flexibility and customization to meet the varying needs of researchers and practitioners in the field of AI. To address this challenge, Yerbamaté provides increased flexibility by not imposing any restrictions on additional module names, enabling researchers to utilize their preferred module names for custom tasks. 
% For example, researchers can use module names such as "simulations" and "analyzers" for their specific use cases, enabling greater customization and adaptability of the framework. 
Additionally, the framework is designed to be compatible with Python, providing greater flexibility as Python can directly be used to execute experiments and Python files.


\subsubsection{Yerbamaté Command Line}
The Yerbamaté command line provides useful utility functions and support for modular projects, which can facilitate the development of machine learning models.
Here is a list of the key Yerbamaté command line options:

\begin{itemize}
\item \textbf{mate init module\_name}: Initializes a new empty modular project skeleton with the given module name. 

\item \textbf{mate install url -y\textbar n\textbar o pm}: Installs a module from a git repository. Supports multiple formats for the repository URL. The flags -y, -n, and -o specify whether to skip confirmation, skip installing python dependencies, and overwrite existing code modules, respectively. The pm argument specifies the package manager to use.
\item \textbf{mate list}: Lists all available modules in the project. 
\item \textbf{mate exports}: Generates dependencies for reproduciblity and sharing.
\item \textbf{mate test exp\_module exp}: Runs the experiment specified by exp in the module exp\_module. Equivalent to \texttt{python -m root\_module.exp\_module.exp test}
\item \textbf{mate train exp\_module exp}: Runs the experiment specified by exp in the module exp\_module. Equivalent to \texttt{python -m root\_module.exp\_module.exp train}
\end{itemize}


\subsubsection{Yerbamaté install}
Yerbamaté's install command is a crucial component of the tool, allowing for effortless installation of modularized projects that adhere to the principles of separation of concerns. With just one command, you can install a model along with its Python dependencies, making it easier to share and export modules. Additionally, Yerbamaté also supports the installation of coupled modules as whole modules. For example, you can install the source code of over 100 torch image models \footnote{\url{https://github.com/rwightman/pytorch-image-models/tree/main/timm/}} and over 30 implementation PyTorch vision in transformers \footnote{\url{https://github.com/lucidrains/vit-pytorch/tree/main/vit\_pytorch/}} directly into your project. However, the limitation of installing non-Yerbamaté projects is that Yerbamaté cannot install Python dependencies out of the box and this modules can only be installed as a whole, and sub-modules, such as models, are not installable as a standalone module since they are not independent.

\subsection{Yerbamaté Environment API}

The Yerbamaté Environment API is a tool designed to manage environment variables within a experiment. It prioritizes the use of an \texttt{env.json} file for storing environment variables, but if it is not found, it falls back to the operating system's environment variables. The API offers a convenient way to set, retrieve, and manage these variables in a centralized and organized manner. This can be particularly useful in storing and accessing environment-specific information, such as local results and data paths, API keys, database URLs, and other sensitive data, thus ensuring that the application operates optimally regardless of the environment in which it is executed. The Yerbamaté Environment API can be easily accessed within experiments, providing a seamless and efficient method for managing environment variables within a project. 




% Additionally, the Yerbamaté framework is designed to be compatible with pure Python, providing greater flexibility as Python can directly be used to execute experiments and Python files. The Structure can be used without the toolkit, allowing for even greater customization and flexibility. This flexibility and customization offered by Yerbamaté can facilitate the implementation of diverse AI applications, supporting the development of a more accessible and collaborative AI research environment.

% \subsubsection{Flexibility and Customization}
% One of the key challenges in the development of the Yerbamaté framework was ensuring flexibility and customization to meet the varying needs of researchers and practitioners in the field of AI. To address this challenge, Yerbamaté provides increased flexibility by not imposing any restrictions on additional module names, enabling researchers to utilize their preferred module names for custom tasks. For example, researchers can use module names such as "simulations" and "analyzers" for their specific use cases, enabling greater customization and adaptability of the framework. Additionally, the Yerbamaté framework is designed to be compatible with pure Python, providing greater flexibility as python can directly be used to execute experiments and python files, and the Structure can be used without the toolkit. This flexibility and customization offered by Yerbamaté can facilitate the implementation of diverse AI applications, supporting the development of a more accessible and efficient AI research environment.
% framework-agnostic, meaning it can work out of the box with any machine learning framework or library.  Moreover, Yerbamaté offers compatibility with Python and Python can be used for running experiments. 


% Additionally, Yerbamate is compatible with Linux systems and Jupyter notebooks, enabling researchers to run their experiments on Colab out of the box. 


\section{Discussion}

% Yerbamaté is an open science Python framework designed to promote software engineering principle of modularity, and enhance reproducibility, collaboration and openness in AI research. The framework offers a promising approach to addressing the reproducibility crisis in AI, by standardizing the development and management of AI projects and facilitating the sharing of research findings. Yerbamaté's software engineering conventions and accompanying toolkit provides researchers with the tools to re-use and share code. 
The decision to use GitHub, a closed-source platform, for an open science project raises concerns regarding the alignment of this platform with the principles of open science. While open source alternatives, such as GitLab and Bitbucket, may be more compatible with open science principles, the authors chose GitHub as a means of facilitating community-based discussions over git. Despite its closed-source nature, the widespread use of GitHub within the developer community, coupled with its popularity, make it a practical choice for experimenting with contemporary open science methodologies. The authors acknowledge the need for open-source alternatives to GitHub. 
% Using git for scientific work can enhance transparency and collaboration among researchers by providing a clear history of project development and contributions by authors.
% The Yerbamate framework and associated repositories are continuously updated to address limitations, bugs, and improve the developer experience, ensuring that it remains relevant and useful to the community.

Future areas of research and improvement for the Yerbamaté framework include the creation of additional artifacts and tutorials with varying formats and complexity levels, further enhancing knowledge dissemination. The framework's modularity also allows for the addition of more searchable and queryable models, expanding its functionality and providing further utility for researchers. Other potential areas of research and improvement include refining the design, testing, and updating of the framework, as well as addressing identified areas for improvement and refactoring the code.


\section{Social Impact}
Machine learning models and artificial intelligence systems have the potential to impact society, both positively and negatively significantly (\cite{mittelstadt2019principles, jobin2019global,arrieta2020explainable, floridi2018ai4people}). Consequently, AI's ethical and societal implications have been the subject of much discussion and research in recent years (\cite{floridi2018ai4people, goodman2017european, floridi2019establishing, mittelstadt2016ethics}).
Open science can significantly impact society by promoting collaboration, transparency, and accessibility in research, enabling broader participation in the scientific process and sharing of knowledge and tools, thus accelerating safer progress and (\cite{kocak2022transparency, wachter2017transparent, coro2020open, braun2018open, paton2019open, goodman2017european}). In artificial intelligence, open science can help democratize access to machine learning and facilitate the development of ethical and accountable AI systems (\cite{goodman2017european, batarseh2020data}). By promoting the principles of open science, researchers and practitioners can work together to build more robust, trustworthy, and fair AI solutions (\cite{accountabilityInAi, kocak2022transparency,wachter2017transparent, coro2020open, braun2018open, hicks2021open, goodman2017european}.
The development of open source, open science, accessible and user-friendly tools for training machine learning models has the potential to democratize access to artificial intelligence and facilitate more involvement in research and innovation. The Yerbamaté toolkit contributes in this regard, providing a simple and effective means for researchers and practitioners to share, develop and evaluate machine learning models. Moreover, it makes training AI accessible to a broader audience as anyone can run an experiment on accessible science tools such as Colab\footnote{\url{https://colab.research.google.com/github/oalee/yerbamate/blob/main/deep_learning.ipynb}} and reproduce a scientific experiment and conduct their own experiments.
The wide accessibility of AI through Yerbamaté and other similar open science tools have the potential to accelerate research in various fields (\cite{olson2018system, wolf2020designing,morris2020ai,ong2021guide, li2018can}) including healthcare (\cite{haristiani2020combining}), law (\cite{ashley2017legal}), and education (\cite{goel2020ai}). For example, machine learning models can improve patient outcomes (\cite{hamet2017medicine}), detect fraud in financial transactions (\cite{bao2022fraudartificial}, and enhance personalized learning in education (\cite{haristiani2020combining, goel2020ai}). At the same time, it is essential to ensure that the development and application of AI adhere to ethical principles, including inclusion, transparency, accountability, and fairness (\cite{ accountabilityInAi,  jobin2019global, floridi2018ai4people, wachter2017transparent, arrieta2020explainable, mittelstadt2019principles, goodman2017european, mittelstadt2016ethics, floridi2019establishing, o2017weapons}).



\section{Conclusion}



    % This research internship aimed to address the reproducibility crisis in artificial intelligence (AI) research by investigating software engineering best practices, open science and accessible AI. To achieve this goal, Yerbamaté framework was developed around the software engineering principles of modularity and separation of concerns. Yerbamaté is an open science modular Python framework designed to streamline and simplify the development and management of machine learning projects. The framework encourages quality coding, collaboration, and the sharing of models, trainers, data loaders, and knowledge, while also promoting reproducibility, customization, and flexibility. The modular design and separation of concerns simplify the development and maintenance of machine learning models, leading to an improved developer experience. The straightforward installation, sharing, and training process makes it accessible to researchers and practitioners with varying technical expertise, enhancing collaboration and knowledge sharing. The findings suggest that the adoption of modular design principles and open science tools can contribute significantly to addressing the reproducibility crisis in AI, leading to more accessible, transparent, and trustworthy AI.
    

The Yerbamaté framework could provide a valuable contribution to the field of artificial intelligence by promoting open science and accessible AI. The software engineering principles and modular design approach utilized by Yerbamaté simplify the development and maintenance of machine learning models, enhancing developer experience, and enabling collaboration and knowledge sharing among researchers and practitioners with varying technical expertise. The flexibility and customization offered by Yerbamaté make it accessible to researchers and practitioners with varying technical backgrounds and enhance the adaptability of the framework to meet the diverse needs of AI projects. The potential for Yerbamaté to address the reproducibility crisis in AI, along with its potential to facilitate collaborations and sharing, makes it a promising tool for future AI research. The promotion of open science frameworks has the potential to revolutionize the development and implementation of trustworthy and transparent AI, fostering innovation and progress across a variety of fields.
% In conclusion, the development of the Yerbamaté framework could offer a significant contribution to the field of artificial intelligence by promoting open science and accessible AI. By providing a modular framework that encourages standardized software engineering practices, Yerbamaté simplifies the development and maintenance of machine learning models, enhancing collaboration and knowledge sharing among researchers and practitioners with varying technical expertise. The framework's ease of use and flexibility enables wider accessibility to artificial intelligence, making it possible for individuals and organizations to develop and train machine learning models with greater ease and efficiency. Overall, the adoption of modular design principles and open science tools has the potential to revolutionize the development and implementation of trustworthy and transparent AI, facilitating innovation and progress in various fields. 
% Nevertheless, the widespread adoption of the Yerbamaté toolkit and other similar tools may pose challenges, including a potential learning curve, implementation overhead, and a need for further refinement and evaluation.




\clearpage
\newpage


\printbibliography




\onecolumn

\section{Appendix}\label{appendix}



\subsection{Yerbamaté CLI}\label{CLIApp}

Yerbamaté CLI provides utility functions and supports modular projects, facilitating machine learning model development. Key CLI options include:

\begin{itemize}
\item \texttt{mate init module\_name}: Initializes a new empty modular project skeleton with the given module name. 

\item \texttt{mate install url -y\textbar n\textbar o pm}: Installs a module from a git repository. Supports multiple formats for the repository URL. The flags -y, -n, and -o specify whether to skip confirmation, skip installing python dependencies, and overwrite existing code modules, respectively. The pm argument specifies the Python package manager to use.
\item \texttt{mate list}: Lists all available modules in the project. 
\item \texttt{mate exports}: Generates dependencies for reproduciblity and sharing.
\item \texttt{mate test exp\_module exp}: Runs the experiment specified by exp in the module exp\_module. Equivalent to \texttt{python -m root\_module.exp\_module.exp test}
\item \texttt{mate train exp\_module exp}: Runs the experiment specified by exp in the module exp\_module. Equivalent to \texttt{python -m root\_module.exp\_module.exp train}
\end{itemize}

\subsubsection{Example Commands}

\begin{itemize}
    \item \textbf{Installing GAN experiment from a modular project}: 
    
    \texttt{mate install oalee/lightweight-gan/lgan/experiments/lgan -yo pip}
    \item \textbf{Training the GAN experiment}: 
    
    \texttt{mate train lgan cars}
    
    \item \textbf{Installing transfer learning experiment}

    \texttt{mate install oalee/big\_transfer/experiments/bit}
    
    \item \textbf{Installing +100 models and code from non modular project code}

    \texttt{mate install https://github.com/rwightman/pytorch-image-models/tree/main/timm/}

    \item \textbf{Installing +30 Pytorch ViT model source code}
    
    \texttt{mate install https://github.com/lucidrains/vit-pytorch/tree/main/vit\_pytorch/}

\end{itemize}


\subsection{Examples}


\subsubsection{Exported Modules}

Yerbamate's mate export command generates a markdown table of reusable components, showcasing metadata such as the module name, type, short URL for installation, exact versions of dependencies for reproducibility, and URLs of code module dependencies. This metadata enhances transparency and reproducibility of research by providing detailed information about each component used in a project. For example, the experiment module may have a dependency on a code module, which is included in the metadata table with its associated information.

% {p{0.35\linewidth} | p{0.6\linewidth}}

\begin{figure}[H]
    \centering

\begin{tabular}{|l|l|l|p{0.35\linewidth}|}
\hline
 name           & type        & short\_url                                          & dependencies                                                                                                                                                                                                                                                                                                                             \\
\hline
 cifar10        & data        & oalee/deep-vision/deepnet/data/cifar10             & ['pytorch\_lightning\textasciitilde{}=1.7.5', 'ipdb\textasciitilde{}=0.13.9', 'torch\textasciitilde{}=1.12.1', 'torchvision\textasciitilde{}=0.13.1']                                                                                                                                                                                                                                                         \\ 
  \hline 
 
 cifar100       & data        & oalee/deep-vision/deepnet/data/cifar100            & ['pytorch\_lightning\textasciitilde{}=1.7.5', 'ipdb\textasciitilde{}=0.13.9', 'torch\textasciitilde{}=1.12.1', 'torchvision\textasciitilde{}=0.13.1']                                                                                                                                                                                                                                                         \\
  \hline 
 keras          & data        & oalee/deep-vision/deepnet/data/keras               & ['tensorflow\_gpu\textasciitilde{}=2.10.0']                                                                                                                                                                                                                                                                                                                   \\
  \hline 
 timm\_aug       & data        & oalee/deep-vision/deepnet/data/timm\_aug            & ['numpy\textasciitilde{}=1.24.2']                                                                                                                                                                                                                                                                                                                            \\
  \hline 
 resnet         & experiments & oalee/deep-vision/deepnet/experiments/resnet       & ['pytorch\_lightning\textasciitilde{}=1.7.5', 'timm\textasciitilde{}=0.6.12', 'torch\textasciitilde{}=1.12.1', 'tensorboard\textasciitilde{}=2.10.0', 'torchvision\textasciitilde{}=0.13.1', 'https://github.com/oalee/deep-vision/tree/main/deepnet/data/cifar10', 'https://github.com/oalee/deep-vision/tree/main/deepnet/trainers/classification', 'https://github.com/oalee/deep-vision/tree/main/deepnet/models/resnet'] \\
 
  \hline 
 
 resnet\_keras   & experiments & oalee/deep-vision/deepnet/experiments/resnet\_keras & ['keras\textasciitilde{}=2.10.0', 'https://github.com/oalee/deep-vision/tree/main/deepnet/models/keras/resnet']                                                                                                                                                                                                                                              \\
  \hline 
 kerasnet       & experiments & oalee/deep-vision/deepnet/experiments/kerasnet     & ['ipdb\textasciitilde{}=0.13.9', 'tensorflow\_gpu\textasciitilde{}=2.10.0', 'keras\textasciitilde{}=2.10.0']                                                                                                                                                                                                                                                                                  \\
  \hline 
 timm           & experiments & oalee/deep-vision/deepnet/experiments/timm         & ['pytorch\_lightning\textasciitilde{}=1.7.5', 'torch\textasciitilde{}=1.12.1', 'https://github.com/oalee/deep-vision/tree/main/timm/', 'https://github.com/oalee/deep-vision/tree/main/deepnet/data/cifar10', 'https://github.com/oalee/deep-vision/tree/main/deepnet/trainers/classification', 'https://github.com/oalee/deep-vision/tree/main/deepnet/models/resnet']       \\
 resnet         & models      & oalee/deep-vision/deepnet/models/resnet            & ['torch\textasciitilde{}=1.12.1']                                                                                                                                                                                                                                                                                                                            \\
  \hline 
 keras          & models      & oalee/deep-vision/deepnet/models/keras             & ['torch\textasciitilde{}=1.12.1', 'tensorflow\_gpu\textasciitilde{}=2.10.0']                                                                                                                                                                                                                                                                                                  \\
  \hline 
 vit\_pytorch    & models      & oalee/deep-vision/deepnet/models/vit\_pytorch       & ['einops\textasciitilde{}=0.4.1', 'torch\textasciitilde{}=1.12.1', 'torchvision\textasciitilde{}=0.13.1']                                                                                                                                                                                                                                                                                    \\
  \hline 
 torch\_vit      & models      & oalee/deep-vision/deepnet/models/torch\_vit         & ['einops\textasciitilde{}=0.4.1', 'torch\textasciitilde{}=1.12.1', 'torchvision\textasciitilde{}=0.13.1']                                                                                                                                                                                                                                                                                    \\
  \hline 
 classification & trainers    & oalee/deep-vision/deepnet/trainers/classification  & ['pytorch\_lightning\textasciitilde{}=1.7.5', 'torchmetrics\textasciitilde{}=0.9.3', 'torch\textasciitilde{}=1.12.1', 'ipdb\textasciitilde{}=0.13.9']                                                                                                                                                                                                                                                         \\
\hline

\end{tabular}
   
\caption{Exported module metadata generated from a modular project. The following example is generated from this repository: \url{https://github.com/oalee/deep-vision} }
    \label{fig:my_label}
\end{figure}


\subsubsection{Example Custom Data Preprocessing}
The modular structure of the Yerbamaté toolkit, coupled with its compatibility with pure Python, allows for the integration of custom data preprocessing pipelines with ease. By utilizing the Yerbamaté environment API, developers and researchers can readily access the data paths and results path for the destination of their processed data. For instance, the following project structure can utilize the command \texttt{python -m deepnet.data.my\_data.preprocessing} to execute a custom preprocessing pipeline. The flexibility offered by the python modularity enables users to efficiently tailor their preprocessing procedures to the specific requirements of their research or application, and the Yerbamaté toolkit can be used to share these pipelines effortlessly.

% \captionsetup[figure]{position=bottom,justification=centering,width=.4\textwidth,labelfont=bf,font=small}

\begin{figure}[H]
\centering
\framebox[\0.4\textwidth]{%
\begin{minipage}{0.4\textwidth}
\dirtree{%
.1 deepnet.
.2 data.
.3 \texttt{\_\_init\_\_.py}.
.3 my\_data.
.4 \texttt{\_\_init\_\_.py}.
.4 preprocessing.
.5 \texttt{\_\_init\_\_.py}.
.5 preprocess.py.
.4 data\_loader.
.2 models.
.2 trainers.
.2 experiments.
}
\end{minipage}
}
\caption{
Custom data preprocessing modular structure example
}
\label{customdata}
\end{figure}

\subsubsection{Example Keras Fine Tuning}

The following exmaple illustrated an experiment definition with Keras framework.

\begin{minted}{python}

from tensorflow import keras
from keras.applications.resnet import ResNet50
import ipdb 

resnet: keras.Model = ResNet50(
    include_top=False,
    input_tensor=keras.Input(shape=(32, 32, 3)),
    classes=10,
    classifier_activation="softmax",
)



resnet.compile(
    optimizer=keras.optimizers.Adam(learning_rate=0.0004, beta_1=0.9, beta_2=0.999),
    loss="binary_crossentropy",
    metrics=["accuracy", "loss"],
)


(x_train, y_train), (x_test, y_test) = keras.datasets.cifar10.load_data()

resnet.fit(
    x=x_train,
    y=y_train,
    validation_data=(x_test, y_test),
    batch_size=64,
    epochs=10,
)

\end{minted}


% \subsubsection{Example JAX Experiment}
% The following experiment showcases the use of the experiment format with JAX framework

\subsubsection{Example GAN Experiment}

The following example illustrates the experiment definition of a Lightweight Generative Adversarial Networks (\cite{lgan,goodfellow2020generative}) implemented with Pytorch Lightning. The use of Python enables the customization of model hyperparameters, loggers, model savers, learning rate schedulers, and optimization algorithms through argument specqification in functions or classes. The source code for the complete project is accessible on Github\footnote{\url{https://github.com/oalee/lightweight-gan}} and all its modules can be installed and the experiment can be trained using Yerbamaté command line on Colab or local machines. The experiment integrates and imports independent trainers, models, and data modules, and defines the experiment's hyperparameters.

% [
% frame=lines,
% framesep=2mm,
% baselinestretch=1.2,
% % bgcolor=LightGray,
% fontsize=\footnotesize,
% linenos
% ]


\begin{minted}{python}
from ...data.cars import CarsLightningDataModule, AugWrapper
from ...trainers.lgan import LightningGanModule
from ...models.lgan import Generator, Discriminator
from torch import nn
import yerbamate, torch, pytorch_lightning as pl, pytorch_lightning.callbacks as pl_callbacks, os
# Managing environment variables
env = yerbamate.Environment()

data_module = CarsLightningDataModule(
    image_size=128,
    aug_prob=0.5,
    in_channels=3,
    data_dir=env["data_dir"],
    batch_size=8,
)

generator = Generator(
    image_size=128,
    latent_dim=128,
    fmap_max=256,
    fmap_inverse_coef=12,
    transparent=False,
    greyscale=False,
    attn_res_layers=[],
    freq_chan_attn=False,
    norm_class=nn.BatchNorm2d,
)

discriminator = Discriminator(
    image_size=128,
    fmap_max=256,
    fmap_inverse_coef=12,
    transparent=False,
    greyscale=False,
    disc_output_size=5,
    attn_res_layers=[],  # Try [16, 32, 64, 128, 256] if your hardware allows
)

g_optimizer = torch.optim.Adam(generator.parameters(), lr=0.0002, betas=(0.5, 0.999))
d_optimizer = torch.optim.Adam(
    discriminator.parameters(), lr=0.0002, betas=(0.5, 0.999)
)

model = LightningGanModule(
    save_dir=env["results"],
    sample_interval=100,
    generator=generator,
    discriminator=AugWrapper(discriminator),
    optimizer=[
        {
            "optimizer": g_optimizer,
            "lr_scheduler": {
                "scheduler": torch.optim.lr_scheduler.StepLR(
                    g_optimizer, step_size=100, gamma=0.5
                ),
                "monitor": "fid",
            },
        },
        {
            "optimizer": d_optimizer,
            "lr_scheduler": {
                "scheduler": torch.optim.lr_scheduler.ReduceLROnPlateau(
                    d_optimizer, mode="min", factor=0.5, patience=5, verbose=True
                ),
                "monitor": "fid",
            },
        },
    ],
    aug_types=["translation", "cutout", "color", "offset"],
    aug_prob=0.5,
)

logger = pl.loggers.TensorBoardLogger(env["results"], name=env.name)
callbacks = [
    pl_callbacks.ModelCheckpoint(
        monitor="fid",
        dirpath=env["results"],
        save_top_k=1,
        mode="min",
        save_last=True,
    ),
    pl_callbacks.LearningRateMonitor(logging_interval="step"),
]
trainer = pl.Trainer(
    logger=logger,
    accelerator="gpu",
    precision=16,
    gradient_clip_val=0.5,
    callbacks=callbacks,
    max_epochs=100,
)

if env.train:
    trainer.fit(model, data_module)
if env.test:
    trainer.test(model, data_module)
if env.restart:
    trainer.fit(model, data_module, ckpt_path=os.path.join(env["results"], "last.ckpt"))

\end{minted}





\section{Transfer Learning Case Study}\label{transfer-study}


In this section, we showcase the application of modularity and separation of concerns on the official implementation of "Big Transfer (BiT): General Visual Representation Learning"\cite{transferlearning}. The source code from the official repository \footnote{\url{https://github.com/google-research/big_transfer}} has been refactored \footnote{\url{https://github.com/oalee/big_transfer}} into a modular, decoupled structure.  The original folder structure of the repository is as follows:


\begin{figure}[H]
\centering
\framebox[\0.4\textwidth]{%
\begin{minipage}{0.4\textwidth}
\dirtree{%
.1 /.
.2 \texttt{bit\_common.py}.
.2 \texttt{bit\_hyperrule.py}.
.2 \texttt{bit\_pytorch}.
.3 \texttt{fewshot.py}.
.3 \texttt{\_\_init\_\_.py}.
.3 \texttt{lbtoolbox.py}.
.3 \texttt{models.py}.
.3 \texttt{requirements.txt}.
.3 \texttt{train.py}.
.2 \texttt{\_\_init\_\_.py}.
}
\end{minipage}
}
\caption{
Official Repository of BiT Project Structure for Pytorch
}
\end{figure}



\vspace{0.4em}
In this examination, we delve into the components of the official Big Transfer repository to better understand the purpose of each one. The components are as follows:

\begin{itemize}
    \item \texttt{bit\_common.py} serves as the central point for defining an argument parser and setting up a logger for experiments. Currently, the project only supports a limited set of hyperparameter selections, which include the initial learning rate, batch size, batch split, and five datasets, namely CIFAR10, CIFAR100, Oxford\_iiit\_pet, Oxford\_flowers102, and ImageNet2012. The name of this file, however, does not accurately reflect its purpose.
    \item \texttt{bit\_hyperrule.py} is responsible for defining the learning rate scheduler and a utility function for computing the resolution of the model based on the dataset. The name of this file, once again, does not accurately reflect its purpose and separates the concern of data-related functions from the learning rate scheduling.
    \item \texttt{few\_shot.py} is specifically designed to find few-shot learning samples for the model, and its name accurately reflects its purpose.
    \item \texttt{lbtoolbox.py} handles interruptions in training and provides a chronometer interface for profiling. This component is independent and does not couple with any other part of the system.
    \item \texttt{models.py} defines the models and is also an independent component that does not couple with any other part.
    \item \texttt{train.py} is utilized for training the model. This component includes the implementation for batch splitting and can only be executed with the pre-defined hyperparameter selection.
\end{itemize}


The following code illustrates the execution of the training procedure in the original repository. The training process is initiated by the \texttt{main} function located at the end of the \texttt{train.py} file. 

\begin{minted}{python}
if __name__ == "__main__":
  parser = bit_common.argparser(models.KNOWN_MODELS.keys())
  parser.add_argument("--datadir", required=True,
                      help="Path to the ImageNet data folder, preprocessed for torchvision.")
  parser.add_argument("--workers", type=int, default=8,
                      help="Number of background threads used to load data.")
  parser.add_argument("--no-save", dest="save", action="store_false")
  main(parser.parse_args())
\end{minted}

This implementation exhibits limited adaptability as the function is dependent solely on the arguments provided through the command-line interface. The following tree structure and experiment definition showcases modularity and separation of concerns applied on this task .




\begin{minted}{python}
from ...trainers.bit_torch.trainer import test, train
from ...models.bit_torch.models import load_trained_model, get_model_list
from ...data.bit import get_transforms, mini_batch_fewshot
import torchvision as tv, yerbamate, os, tensorboard
from torch.utils.tensorboard import SummaryWriter


# BigTransfer Medium ResNet50 Width 1
model_name = "BiT-M-R50x1"
# Choose a model form get_model_list that can fit in to your memoery
# Try "BiT-S-R50x1" if this doesn't works for you

env = yerbamate.Environment()

train_transform, val_transform = get_transforms(img_size=[32, 32])
data_set = tv.datasets.CIFAR10(
    env["datadir"], train=True, download=True, transform=train_transform
)
val_set = tv.datasets.CIFAR10(env["datadir"], train=False, transform=val_transform)

train_set, val_set, train_loader, val_loader = mini_batch_fewshot(
    train_set=data_set,
    valid_set=val_set,
    examples_per_class=None,  # Fewshot disabled
    batch=128,
    batch_split=2,
    workers=os.cpu_count(),  # Auto-val to cpu count
)

imagenet_weight_path = os.path.join(env["weights_path"], f"{model_name}.npz")
model = load_trained_model(
    weight_path=imagenet_weight_path, model_name=model_name, num_classes=10
)
logger = SummaryWriter(log_dir=env["results"], comment=env.name)

if env.train:
    train(
        model=model,
        train_loader=train_loader,
        valid_loader=val_loader,
        train_set_size=len(train_set),
        save=True,
        save_path=os.path.join(env["results"], f"trained_{model_name}.pt"),
        batch_split=2,
        base_lr=0.001,
        eval_every=100,
        log_path=os.path.join(env["results"], "log.txt"),
        tensorboardlogger=logger,
    )

if env.test:
    test(
        model=model,
        val_loader=val_loader,
        save_path=os.path.join(env["results"], f"trained_{model_name}.pt"),
        log_path=os.path.join(env["results"], "log.txt"),
        tensorboardlogger=logger,
    )

\end{minted}



\begin{figure}[H]
\centering
\framebox[\0.4\textwidth]{%
\begin{minipage}{0.4\textwidth}
\dirtree{%
.1 /\texttt{big\_transfer}.
.2 data.
.3 bit.
.4 \texttt{fewshot.py}.
.4 \texttt{\_\_init\_\_.py}.
.4 \texttt{minibatch\_fewshot.py}.
.4 \texttt{requirements.txt}.
.4 \texttt{transforms.py}.
.3 \texttt{\_\_init\_\_.py}.
.2 experiments.
.3 bit.
.4 \texttt{dependencies.json}.
.4 \texttt{few\_shot.py}.
.4 \texttt{\_\_init\_\_.py}.
.4 \texttt{learn.py}.
.4 \texttt{requirements.txt}.
.3 \texttt{\_\_init\_\_.py}.
.2 \texttt{\_\_init\_\_.py}.
.2 models.
.3 \texttt{bit\_torch}.
.4 downloader.
.5 \texttt{downloader.py}.
.5 \texttt{\_\_init\_\_.py}.
.5 \texttt{requirements.txt}.
.5 \texttt{utils.py}.
.4 \texttt{\_\_init\_\_.py}.
.4 \texttt{models.py}.
.4 \texttt{requirements.txt}.
.3 \texttt{\_\_init\_\_.py}.
.2 trainers.
.3 \textt{bit\_torch}.
.4 \texttt{\_\_init\_\_.py}.
.4 \texttt{utils.py}.
.4 \texttt{logger.py}.
.4 \texttt{lr\_schduler.py}.
.4 \texttt{requirements.txt}.
.4 \texttt{trainer.py}.
.3 \texttt{\_\_init\_\_.py}.
}
\end{minipage}
}\caption{Refactored Repository of BiT Project Structure}
\end{figure}


The new structure in the refactored repository is designed to address the limitations of the original implementation. By separating concerns and adopting modularization, the refactored repository provides a more flexible and scalable solution for training models. The different components in the new structure, such as the trainer, model, and data loading modules, are designed to be independent and reusable, making it easier to manage and maintain the codebase. Moreover, the experimentation module provides a unified interface for combining and executing the various components, making it easier to experiment with different configurations and hyperparameters. Overall, the new structure in the refactored repository represents a significant improvement over the original implementation, offering a better and more organized approach to training machine learning models.


\end{document}