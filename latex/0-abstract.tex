
\begin{abstract}

This report presentes
Yerbamate\footnote{\url{https://github.com/oalee/yerbamate}}, an open science
% \footnote{
% % Yerbamate strives to promote the principles of open science, transparency and collaboration. In support of this, the work's GitHub repository, \url{https://github.com/oalee/os-yerbamate}, provides an open QA platform where the history of the LaTeX files for this work are available for transparency. 
% To promote the principles of open science a non academic simple to understand tutorial for this framework has been published on medium to make the knowledge gained from this research accessible to a wider audience \url{https://medium.com/ai-in-plain-english/the-ultimate-deep-learning-project-structure-a-software-engineers-guide-into-the-land-of-ai-c383f234fd2f}
% .
% } 
framework designed for Python-based research projects, with a focus on Artificial Intelligence (AI) and deep learning. The framework prioritizes modularity and separation of concerns to enhance the developer experience, promote reproducibility, reusability, shareability, maintainability, and code quality. Yerbamate supports all Python frameworks and libraries, including Jax, Keras, PyTorch, Hugging Face, and Darts, and can be applied to a wide range of tasks, such as natural language processing, computer vision, deep learning, machine learning, genetic algorithm and optimizations. Yerbamate simplifies the experimentation process, providing a user-friendly installation procedure, intuitive commands for running experiments, and streamlined execution of code. The framework fosters open science by enabling the sharing of independent code modules, such as models and trainers, among practitioners. Adoption of Yerbamate can facilitate collaboration, enhance the developer experience, and promote open science while contributing to addressing the reproducibility crisis in the field of AI.


% This paper introduces Yerbamate\footnote{\url{https://github.com/oalee/yerbamate}}, an open science\footnote{To promote the principles of open science, the progress history of this paper is available on GitHub at the following URL: \url{https://github.com/oalee/os-yerbamate}} framework and software engineering convention for Python-based AI research projects. Yerbamate prioritizes modularity and separation of concerns to promote reproducibility, reusability, shareability, maintainability, and code quality. The convention and framework can be applied to a variety of Python-based tasks, including experimentation with machine learning, deep learning, genetic algorithms, optimizations, and simulations and analysis. Yerbamate simplifies the implementation of the convention with straightforward commands for installing, experimenting, and running code. The framework is compatible with all Python libraries and AI/ML frameworks, and it fosters open science by this convention enabling sharing of code modules such as models and trainers between practitioners. The adoption of this convention and framework can enhance collaboration, help address the reproducibility crisis in AI and enhance the developer experience while promoting open science and accessible AI.

% This reports presents Yerbamate\footnote{\url{https://github.com/oalee/yerbamate}}, an open science
% % \footnote{
% % As an open science work, Yerbamate strives to promote the principles of transparency and collaboration. To this end, the history of the LaTeX files for work are available on GitHub: \url{https://github.com/oalee/os-yerbamate}. These open science repositories are open to collaboration and encourage participation from the community to enhance the validity, reproducibility, accessibility, and quality of this work.
% % } 
% framework designed for Python-based projects including Artificial Intelligence (AI). Yerbamate emphasizes modularity and separation of concerns to promote reproducibility, reusability, shareability, maintainability, and code quality. Yerbamate is applicable to a wide range of tasks, including deep learning, machine learning, optimization, and genetic algorithms. It supports all Python frameworks and libraries such as Jax, Keras, PyTorch, Hugging Face, and Darts, making it a versatile tool for AI development. Yerbamate simplifies the experimentation process with a user-friendly installation procedure, intuitive commands for running experiments, and streamlined execution of code. The framework can fosters open science by enabling the sharing of code modules such as models and trainers between practitioners. The adoption of Yerbamate can facilitate collaboration, enhance the developer experience, and promote open science while contributing to addressing the reproducibility crisis in the field of AI.
\end{abstract}





% This paper presents a software engineering convention, accompanied by a framework named Yerbamate, for Python-based artificial intelligence (AI) projects. The convention emphasizes modularity and separation of concerns, with the goal of promoting reproducibility, reusability, shareability, maintainability, and code quality. The use of modular code structures enables flexibility and scalability, facilitating the reuse of code and its components. The Yerbamate framework simplifies the implementation of the convention by providing a set of simple and consistent commands for installing, experimenting, and running code. The convention and framework are compatible with all Python libraries, AI and ML frameworks, and can be used for a variety of Python-based experimentation, analysis, ML, AI, genetic algorithms, particle swarm optimization, and simulations. The use of this convention and framework can address the reproducibility crisis in AI and promote open science by enabling the sharing of code modules among researchers and practitioners



%   In recent years, deep learning has emerged as a powerful tool for solving complex problems in various fields, such as image and speech recognition, natural language processing, and computer vision (\cite{lecun2015deep}). 
% Deep learning projects, whilst still considered software engineering, differ from traditional software engineering in that they revolve around the creation of models that can extract knowledge from data and make predictions or decisions, as opposed to relying on a pre-defined set of instructions (\cite{lecun2015deep,amershi2019software,wan2019does,se4dl}). These disparities make some phases such as the testing phase of deep learning projects distinct from traditional software (\cite{wan2019does}). Still, principles and design patterns commonly used in software engineering can still aid in developing and maintaining deep learning software (\cite{amershi2019software,wan2019does,se4dl}). 

% Artificial intelligence (AI) and machine learning (ML) have become increasingly powerful in recent years, with applications in fields ranging from natural language, art, music, healthcare, finance to education. While these technologies have the potential to revolutionize many areas of human life, their successful development and deployment depend heavily on quality code and sound software engineering practices. Good software engineering practices can increase the reliability, maintainability, and scalability of AI and ML systems, enabling their widespread adoption and use.