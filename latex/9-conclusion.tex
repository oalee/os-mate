

\section{Acknowledgments and Contribution Alt\-metric}

The author acknowledges the invaluable contributions of the open source and open science community, including the developers of tools, libraries, and frameworks such as Jax, Keras, PyTorch, HuggingFace, and many others, as well as researchers and developers who contribute to open source projects. The author expresses gratitude to all the open source code that has been forked and utilized in the development of Yerbamaté. Without these contributions, this work would not be possible.


The author acknowledges the contributors of open source framework Maté\footnote{\url{https://github.com/ilex-paraguariensis/yerbamate}}, an AI experimentation framework that underwent various design developments and testing from June 2022 onwards. Due to laws at Maastricht University that prohibit collaborations in research internships, the project was forked into Yerbamaté as an individual effort to enhance flexibility. The author acknowledges the contributions of Giulio Zani, the developer of the Maté project.
The author also expresses gratitude to their supervisors, Iris Groen of the University of Amsterdam and Mirela Popa of Maastricht University, for their guidance and support. 

The progress and contributions of the Yerbamaté framework development can be tracked using the git commit history as an alt-metric. The contributions and progress can be analyzed using various tools, including the contribution analyses tool provided on the Yerbamaté repositories\footnote{\url{https://github.com/oalee/yerbamate/graphs/contributors}}\footnote{\url{https://github.com/oalee/os-yerbamate/graphs/contributors}}. Future work could include exploring and refining alternative metrics to more accurately capture community engagement and collaboration, beyond just code contributions. This could involve developing new tools and methods for tracking and analyzing contributions and engagement within the Yerbamate framework and extending it to other open-source projects. Additionally, identifying ways to encourage and incentivize community engagement, such as providing rewards or recognition for contributions, could further promote collaboration and improve the development of the framework.

% The contributions to the Yerbamaté framework extend beyond just code commits and can also include community engagement, such as issue tracking, responding to questions, and participating in discussions. These contributions can be tracked and analyzed using alternative metrics, such as karma points, which can provide insight into the level of community involvement and collaboration around the framework.

% The progress and contributions of the Yerbamaté framework development can be tracked using the git commit history as an alt-metric, providing a measure of community engagement and collaboration. The contributions and progress can be analyzed using various tools, including the contribution analyses tool provided on the Yerbamaté repository.\footnote{\url{https://github.com/oalee/yerbamate/graphs/contributors}}
% \footnote{\url{https://github.com/oalee/os-yerbamate/graphs/contributors}}
% The contributions to the Yerbamaté framework extend beyond just code commits and can also include community engagement, such as issue tracking, responding to questions, and participating in discussions. These contributions can be tracked and analyzed using alternative metrics, such as karma points, which can provide insight into the level of community involvement and collaboration around the framework.
\section{Conclusion}





The Yerbamaté framework could provide a valuable contribution to the field of artificial intelligence by promoting open science and accessible AI. The software engineering principles and modular design approach utilized by Yerbamaté simplify the development and maintenance of machine learning models, enhancing developer experience, and enabling collaboration and knowledge sharing among researchers and practitioners with varying technical expertise. The flexibility and customization offered by Yerbamaté make it accessible to researchers and practitioners with varying technical backgrounds and enhance the adaptability of the framework to meet the diverse needs of AI projects. The potential for Yerbamaté to address the reproducibility crisis in AI, along with its potential to facilitate collaborations and sharing, makes it a promising tool for future AI research. The promotion of open science frameworks has the potential to revolutionize the development and implementation of trustworthy and transparent AI, fostering innovation and progress across a variety of fields. Moreover the Yerbamate framework has the potential for further expansion and refinement, providing opportunities to augment its functionality with more dependable, searchable and queryable models to boost its utility for researchers.

% In conclusion, the development of the Yerbamaté framework could offer a significant contribution to the field of artificial intelligence by promoting open science and accessible AI. By providing a modular framework that encourages standardized software engineering practices, Yerbamaté simplifies the development and maintenance of machine learning models, enhancing collaboration and knowledge sharing among researchers and practitioners with varying technical expertise. The framework's ease of use and flexibility enables wider accessibility to artificial intelligence, making it possible for individuals and organizations to develop and train machine learning models with greater ease and efficiency. Overall, the adoption of modular design principles and open science tools has the potential to revolutionize the development and implementation of trustworthy and transparent AI, facilitating innovation and progress in various fields. 
% Nevertheless, the widespread adoption of the Yerbamaté toolkit and other similar tools may pose challenges, including a potential learning curve, implementation overhead, and a need for further refinement and evaluation.