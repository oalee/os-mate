

\section{Acknowledgments}

The author acknowledges the open source and open science community, tools, libraries, and frameworks, as well as the various open source and open science AI projects that made the development of the Yerbamaté framework possible. Without these contributions, this work would not be possible.

The development of the Maté\footnote{\url{https://github.com/ilex-paraguariensis/yerbamate}} framework, an open source tool for AI experimentation, began in June 2022
 and was further developed and tested various designs in the following months. Due to laws at Maastricht University that prohibits collaborations in research internships, the project was forked into Yerbamaté as an individual effort to enhance flexibly and open science.

The author also expresses gratitude to their supervisors, Iris Groen of the University of Amsterdam and Mirela Popa of Maastricht University, for their guidance and support. 
The author acknowledges Giulio Zani's contributions to the open source project, Maté. and many sample projects in the appendix, which are forked from Giulio Zani's repositories. The author's contribution to the development of the Yerbamaté framework is presented as an alt-metric of commits history, and the progress of the work can be tracked using GitHub's analyzing contributions tool\footnote{\url{https://github.com/oalee/yerbamate/graphs/contributors}}.
% The contributions of Giulio Zani to the open source project called Maté are also acknowledged. The author would also like to express gratitude to the supervisors, Iris Groen of University of Amsterdam and Mirela Popa of Maastricht University, for their guidance and support. Finally, the author acknowledges the many sample projects in the appendix, which are forked from Giulio Zani's repositories.
% The author's contribution to the implementation of the Yerbamaté framework is presented as an alt-metric of commits, and the progress of the work is accessible from GitHub's\footnote{\url{https://github.com/oalee/yerbamate/graphs/contributors}} analyzing contributions tool.
% Moreover, the contribution of code to this project is presented as an altmetric in Figure X.

% \section{Acknowledgments}

% The author acknowledges the open source community, as this work builds on open source projects, libraries, and frameworks, without which this work would not have been possible. The Yerbamaté framework was started as an open source project and further developed during a research internship at UVA for a master's thesis in AI. Due to prohibitive laws at Maastricht University that restrict collaborations in research internships, the project was forked as an individual effort. The contributions of Giulio Zani to the open source project called Maté are also acknowledged. The Yerbamaté framework was designed to be flexible and extensible, with the goal of promoting open science and improving the developer experience. The development of the framework was made possible by the guidance of the supervisors, Iris Goren of UVA and Mirela Popa of Maastricht University. Additionally, the author would like to acknowledge the support of the open source community, which provided valuable feedback, contributions, and insights, without which this project would not have been possible.

% The Yerbamate framework acknowledges the open source community and the principles of open science. This work started as an open source project and was further developed during a research internship at UVA for a master's thesis in AI. Due to prohibitive laws at Maastricht University that restrict collaborations in research internships, the project was forked as an individual effort. The contributions of Giulio Zani to the open source project called Maté are also acknowledged. The Yerbamate framework was designed to be flexible and extensible, with the goal of promoting open science and improving the developer experience. The development of the framework was made possible by the guidance of the supervisors, Iris Goren of UVA and Mirela Popa of Maastricht University.

\section{Conclusion}



    % This research internship aimed to address the reproducibility crisis in artificial intelligence (AI) research by investigating software engineering best practices, open science and accessible AI. To achieve this goal, Yerbamaté framework was developed around the software engineering principles of modularity and separation of concerns. Yerbamaté is an open science modular Python framework designed to streamline and simplify the development and management of machine learning projects. The framework encourages quality coding, collaboration, and the sharing of models, trainers, data loaders, and knowledge, while also promoting reproducibility, customization, and flexibility. The modular design and separation of concerns simplify the development and maintenance of machine learning models, leading to an improved developer experience. The straightforward installation, sharing, and training process makes it accessible to researchers and practitioners with varying technical expertise, enhancing collaboration and knowledge sharing. The findings suggest that the adoption of modular design principles and open science tools can contribute significantly to addressing the reproducibility crisis in AI, leading to more accessible, transparent, and trustworthy AI.
    

The Yerbamaté framework could provide a valuable contribution to the field of artificial intelligence by promoting open science and accessible AI. The software engineering principles and modular design approach utilized by Yerbamaté simplify the development and maintenance of machine learning models, enhancing developer experience, and enabling collaboration and knowledge sharing among researchers and practitioners with varying technical expertise. The flexibility and customization offered by Yerbamaté make it accessible to researchers and practitioners with varying technical backgrounds and enhance the adaptability of the framework to meet the diverse needs of AI projects. The potential for Yerbamaté to address the reproducibility crisis in AI, along with its potential to facilitate collaborations and sharing, makes it a promising tool for future AI research. The promotion of open science frameworks has the potential to revolutionize the development and implementation of trustworthy and transparent AI, fostering innovation and progress across a variety of fields.
% In conclusion, the development of the Yerbamaté framework could offer a significant contribution to the field of artificial intelligence by promoting open science and accessible AI. By providing a modular framework that encourages standardized software engineering practices, Yerbamaté simplifies the development and maintenance of machine learning models, enhancing collaboration and knowledge sharing among researchers and practitioners with varying technical expertise. The framework's ease of use and flexibility enables wider accessibility to artificial intelligence, making it possible for individuals and organizations to develop and train machine learning models with greater ease and efficiency. Overall, the adoption of modular design principles and open science tools has the potential to revolutionize the development and implementation of trustworthy and transparent AI, facilitating innovation and progress in various fields. 
% Nevertheless, the widespread adoption of the Yerbamaté toolkit and other similar tools may pose challenges, including a potential learning curve, implementation overhead, and a need for further refinement and evaluation.