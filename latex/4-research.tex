
\section{Research Questions}

% The objective of this internship at the start was to address the following research questions.

% \subsection{
% What should be the recommended conventions for software engineering in AI and machine learning to promote reproducibility, reusability, shareability, maintainability, and code quality?
% }
\subsection{
    What are the most effective software engineering practices for promoting reproducibility and open science in AI research?}
    Effective software engineering practices for promoting reproducibility and open science in AI and ML research include modularity, separation of concerns, documentation, version control, and testing. Modularity enables the reuse of code and components, while separation of concerns separates code into distinct modules based on their functionality, making it easier to maintain and modify. Documentation helps other researchers understand how to use and reproduce the code, while version control enables tracking of changes to the code over time. Testing helps ensure that the code works as intended and can be used by others. These practices can help promote open science by enabling the sharing of code and promoting transparency in research.
    
    
    
\subsection{
How can software engineering principles be utilized for AI to improve the developer experience?
}

Software engineering best practices in AI/ML can be utilized to improve the developer experience by prioritizing modularity, separation of concerns, and standardized naming conventions for independent and interchangeable modules such as models, data, and trainers. This approach allows for a standardized approach to project organization, code structure, and modularity, which can facilitate collaboration and code sharing among researchers and practitioners. Additionally, providing clear documentation and support resources can help users understand and adopt the convention, leading to an improved developer experience.
% Furthermore, the use of open-source technologies and practices can promote collaboration and sharing of code and models within the scientific community, leading to the development of more efficient and effective AI/ML research practices.
% Software engineering conventions in AI/ML can be designed to improve the developer experience and promote open science practices by prioritizing modularity, separation of concerns, and consistent naming conventions for independent and interchangeable modules such as models, data, and trainers. To ensure compatibility and customizability, the convention should be flexible enough to accommodate a wide range of use cases and project types. Clear documentation, training materials, and support resources can facilitate adoption of the convention, while open-source technologies and practices can foster collaboration and sharing of code and models within the scientific community. Additionally, the ability to upgrade the convention with input from the community can ensure that the convention remains relevant and adaptable to new technologies and emerging research practices. By adopting a well-designed software engineering convention, AI/ML developers can improve the efficiency and effectiveness of their research, while promoting open science practices and facilitating collaboration within the research community.


\subsection{What are the key tools and technologies needed to support open science, and how can they be made more accessible to researchers and the wider scientific community?}
The key tools and technologies needed to support open science include open source frameworks, open-access journals, repositories, data sharing platforms, collaborative tools for writing, documentation and version control, and open source software and hardware. These tools can help to promote transparency, reproducibility, and collaboration in scientific research, and can contribute to the development of a more open and accessible research environment. To make these tools more accessible to researchers and the wider scientific community, it is important to provide clear documentation, training materials, and support resources, as well as promoting interdisciplinary collaborations and knowledge sharing. Additionally, adopting open science practices can require a cultural shift in the scientific community, and it is important to engage stakeholders and the public in discussions around the benefits and challenges of AI and open science. By promoting open science and making key tools and technologies more accessible, we can help to foster a more collaborative and innovative research environment.
    
% \subsection{
% In what ways does the Yerbamate framework align with the principles of open science, how does it support collaboration, and how can its features contribute to the development of more transparent and reproducible AI and ML research?
% }
% The Yerbamate framework aligns with the principles of open science in several ways. It promotes transparency and reproducibility by providing a set of consistent tools and practices that allow for the creation, sharing, and experimentation of machine learning and artificial intelligence models. The framework's modular design allows for easy customization and extension, making it simple for researchers to build upon existing work and collaborate with others. Additionally, the framework's collaboration features, such as out-of-the-box sharing of models via GitHub URLs, facilitate easy sharing and reuse of code, enabling researchers to build on each other's work and contribute to a more open and collaborative scientific community. By combining these features, the Yerbamate framework provides a powerful tool for advancing open science practices in the field of AI and ML research.
% \subsection{
%     How can we establish a set of conventions for software engineering in AI and ML that are widely accepted and easy to adopt?}
%     Establishing widely accepted and easy to adopt software engineering conventions for AI and ML is a complex task that requires input and collaboration from multiple stakeholders, including researchers, practitioners, and software engineers. One approach is to focus on the principles of modularity, separation of concerns, and documentation, which can help to promote flexibility, reusability, and maintainability of code. The use of independent modules that only depend on python dependencies and the code inside the module uses relative imports is a simple and effective way to promote modularity. This convention can be supplemented with best practices for documentation, including standardized naming conventions and clear documentation of code functionality, assumptions, and limitations.
    
    % \subsection{
    % How can modularity be utilized to promote open science practices in the field of artificial intelligence and machine learning?
    % }
    
    % \subsection{
    % Can conventions in AI and ML improve the learning experience for developers and enhance the efficiency and effectiveness of AI development?
    % }
    % Establishing conventions for AI and ML development can lead to an improved developer experience, resulting in greater efficiency and effectiveness in the development process. Conventions that prioritize modularity, separation of concerns, and consistent naming conventions can help to make the code more maintainable and reusable. This can save developers time in the long run, as they can easily incorporate existing code and components into new projects without having to start from scratch. Additionally, established conventions can make it easier for developers to collaborate and share code with others, ultimately contributing to a more open and collaborative scientific community. Overall, the adoption of conventions in AI and ML can lead to a more efficient and effective development process, ultimately enhancing the developer experience.
    
    % \subsection{
    % How can the adoption of software engineering conventions in AI and ML improve the developer experience and enhance the efficiency and effectiveness of AI development?
    % }
    % The establishment of conventions for software engineering in AI and ML can improve the developer experience by providing a standardized approach to project organization, code structure, and modularity. Such conventions can facilitate collaboration and enable code sharing among researchers, practitioners, and the wider scientific community. 
    % In the case of the Yerbamate framework, the use of independent modules and a non-independent experimentation module provides a clear separation of concerns and enables ease of use for experimentation, training, and validation of models. The Yerbamate framework provides a set of simple and consistent commands for installing, experimenting, and running code, which can lead to an improved developer experience for AI. Additionally, the framework includes an out-of-the-box sharing feature that allows users to share their models via GitHub URLs. By making it easy for users to share and reuse existing code and models, the framework promotes collaboration and open science practices in AI development.
    
%     \subsection{
%   What are the benefits and challenges of adopting a software engineering convention for AI and ML research, and how can we ensure that the benefits outweigh the costs while promoting open science and fostering a culture of collaboration and sharing within the research community?
%     }
%     Adopting a software engineering convention for AI and ML research can lead to significant benefits, including improved code quality, better maintainability, and enhanced reproducibility. However, there can be a learning curve associated with adopting a new convention, particularly for researchers who may not have extensive software engineering backgrounds. The challenge is to develop a convention that is easy to learn and widely accepted. Modularity can be an effective approach, as it allows for the gradual refactoring of existing code into independent, reusable modules, which can lead to an improved developer experience over time. To ensure that the benefits of adopting a convention outweigh the costs, it is essential to provide clear documentation, training materials, and support resources to facilitate the learning process. Additionally, it is important to foster a culture of collaboration and sharing within the research community, which can help to promote the widespread adoption of best practices and ensure that the benefits of improved software engineering practices are shared across the field.



% \subsection{How can a software engineering convention for AI/ML be designed to ensure compatibility and customizability, and what are the benefits of such a convention for open science and collaborative research?}
% The convention should prioritize modularity, separation of concerns, and consistent naming conventions for independent and interchangeable modules such as models, data, and trainers. The convention should also be flexible enough to accommodate a wide range of use cases and project types. Clear documentation, training materials, and support resources can help users understand and adopt the convention, and open-source technologies and practices can facilitate collaboration and sharing of code and models within the scientific community.

% \subsection{How can modular software engineering conventions promote the reusability and scalability of AI and ML systems?}

% \subsection{Can a standard format be designed for data loaders and preprocessing pipelines?}
% {
% The development of a universal standard format for data loaders and preprocessing pipelines in deep learning is challenging due to the framework-specific nature of the task. Preprocessing pipeline design is highly dependent on the specific problem and its data, making it difficult to create a general solution. However, Separation of concerns (SoC) and modularity principles can enhance code reusability by creating the data loading as an independent python module, and separating preprocessing pipeline from data loading. This project focused on making data loading and code components, such as data loaders and the preprocessing pipelines, sharable within a python machine learning framework.
% }

% \subsection{Can this be put in a high-quality, easily accessible database for local access?} {
% The creation of a high-quality, easily accessible database storing preprocessed datasets and related components, such as data loaders and preprocessing pipelines, is complex and involves challenges in terms of data processing, storage, and retrieval, which were beyond the scope of the project. Limitations such as credibility, scalability, compatibility with different deep learning frameworks, privacy, and storage pose additional challenges. This project focused on designing a standard for sharing code-related components, instead of making data available, for feasibility of the task.
% }

% \subsection{Can the database be made easily expandable?} {
% Expanding the database of deep learning code components, such as data loaders, relies on adopting software engineering principles of separation of concerns (SoC) and modularity. Adhering to these principles makes code components shareable across projects within the same framework, which can get added to a database and scale in time.
% }