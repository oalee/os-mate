\section{Discussion}

% Yerbamaté is an open science Python framework designed to promote software engineering principle of modularity, and enhance reproducibility, collaboration and openness in AI research. The framework offers a promising approach to addressing the reproducibility crisis in AI, by standardizing the development and management of AI projects and facilitating the sharing of research findings. Yerbamaté's software engineering conventions and accompanying toolkit provides researchers with the tools to re-use and share code. 
The decision to use GitHub, a closed-source platform, for an open science project raises concerns regarding the alignment of this platform with the principles of open science. While open source alternatives, such as GitLab and Bitbucket, may be more compatible with open science principles, the authors chose GitHub as a means of facilitating community-based discussions over Git. Despite its closed-source nature, the widespread use of GitHub within the developer community, coupled with its popularity, make it a practical choice for experimenting with contemporary open science methodologies. The authors acknowledge the need for open-source alternatives to GitHub. 
% The Yerbamate framework and associated repositories are continuously updated to address limitations, bugs, and improve the developer experience, ensuring that it remains relevant and useful to the community.

Future areas of research and improvement for the Yerbamaté framework include the creation of additional tutorials with varying formats and complexity levels, further enhancing knowledge dissemination. The framework's modularity also allows for the addition of more searchable and queryable models, expanding its functionality and providing further utility for researchers. Other potential areas of research and improvement include refining the design, testing, and updating of the framework, as well as addressing identified areas for improvement and refactoring the code.
