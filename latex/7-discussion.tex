\section{Discussion}

% Yerbamaté is an open science Python framework designed to promote software engineering principle of modularity, and enhance reproducibility, collaboration and openness in AI research. The framework offers a promising approach to addressing the reproducibility crisis in AI, by standardizing the development and management of AI projects and facilitating the sharing of research findings. Yerbamaté's software engineering conventions and accompanying toolkit provides researchers with the tools to re-use and share code. 
The decision to use GitHub, a closed-source platform, for an open science project raises concerns regarding the alignment of this platform with the principles of open science. While open source alternatives, such as GitLab and Bitbucket, may be more compatible with open science principles, the authors chose GitHub as a means of facilitating community-based discussions over Git. Despite its closed-source nature, the widespread use of GitHub within the developer community, coupled with its popularity, make it a practical choice for experimenting with contemporary open science methodologies.

However, the authors acknowledge the need for open-source alternatives to GitHub and the potential for the open-source community to make future improvements. The Yerbamate framework and associated repositories are continuously updated to address limitations, bugs, and improve the developer experience, ensuring that it remains relevant and useful to the community.


The development of AI systems is a complex and challenging task that involves multiple stages, including data preprocessing, model selection and training, and performance evaluation (\cite{lecun2015deep}). To navigate this complexity, researchers and practitioners require tools and frameworks that are not only efficient and effective but also user-friendly, customizable, and provide a seamless development experience (\cite{cardoso2022monai,pathml,thingsvision, olson2018system}). Thus, the development of user-friendly and customizable tools, such as the Yerbamaté framework, is critical to promoting ease and accessibility in AI research and development (\cite{olson2018system}). The continued development of such tools can enhance the developer experience in the field of artificial intelligence and drive innovation and progress in the field.
% and components, and ultimately increasing reproducibility and the sharing of research findings.


The flexibility and customization offered by Yerbamaté facilitate the adaptation of the framework to meet researchers' diverse needs, contributing to the development of more accessible and efficient AI research environments. Researchers with varying technical backgrounds can share coding under software engineering guidelines, fostering collaboration and interdisciplinary learning.

% Additionally, the Yerbamaté framework is designed to be compatible with pure Python, providing greater flexibility as Python can directly be used to execute experiments and Python files. The Structure can be used without the toolkit, allowing for even greater customization and flexibility. This flexibility and customization offered by Yerbamaté can facilitate the implementation of diverse AI applications, supporting the development of a more accessible and collaborative AI research environment.

The impact of open science frameworks in AI is a crucial area for future research. Further investigations can focus on the usability of the framework, its potential limitations, and identifying areas for improvement. Further studies could investigate the user experience of open source frameworks in the field of AI to identify areas for improvement and potential limitations.
% Future studies can also explore the impact of Yerbamaté on facilitating interdisciplinary collaborations and knowledge sharing in the field of AI.

%  Another area for future work could be the creation of naming conventions for modules and functions, which could further enhance the modularity and readability of the code. Additionally, comprehensive evaluation and testing of the framework could help to identify any potential issues or areas for improvement, ensuring the continued effectiveness and usefulness of the Yerbamaté framework.
 
 % The modular design principles employed in Yerbamaté can enhance transparency, reproducibility, and the sharing of knowledge and research findings. However, it is worth noting that the adoption of these principles may require additional time and effort for researchers who are not familiar with software engineering principles. Additionally, compatibility with pure Python may pose some challenges to researchers who prefer other programming languages.
