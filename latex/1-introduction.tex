
\section{Introduction}


 In recent years, AI has made remarkable progress, finding numerous applications across various fields such as natural language (\cite{gpt}), general intelligence(\cite{gato}), art (\cite{diffusion}), music (\cite{musiclm}), healthcare (\cite{aihealthcare}), finance (\cite{bao2022fraudartificial}), education (\cite{aieducation}), and weather forecasting (\cite{weather}). The potential benefits of AI are immense (\cite{beneficialai,potencialaibenefit}), their development and deployment require quality data (\cite{lecun2015deep}), code, and sound software engineering practices (\cite{se4dl,amershi2019software}). Poor software engineering practices can lead to a reproducibility crisis, undermining the reliability and credibility of AI systems (\cite{leakage-recrisis}). On the other hand, good software engineering practices can enhance the developer experience, reliability, maintainability, and replicability of AI systems (\cite{se4dl,amershi2019software, wan2019does}). In order to realize the full potential of AI, researchers and practitioners require open source tools and frameworks that are user-friendly, customizable, and promote reproducibility in the development of AI projects (\cite{lu2022softwareAIReponse,li2018can,wolf2020designing,olson2018system,ong2021guide,gundersen2018reproducible}). This is particularly important for ensuring that the benefits of AI are accessible to all and that the development of AI aligns with open science and ethical principles, enabling the sharing of research findings and supporting the progress and implementation of diverse AI applications for societal benefits (\cite{coro2020open,braun2018open, mittelstadt2016ethics,floridi2018ai4people,ong2021guide}).

The Yerbamate framework embodies a novel approach to open science and python development, showcasing a modular design that emphasizes reproducibility, reusability, transparency, and collaboration in the research community. The framework incorporates contemporary open science methodologies, blending open-source and open science principles by leveraging version control (git) and community-based discussion platforms for open question answering, feedback, and issue tracking. The framework is subject to continuous updates and refinements to address limitations, enhance the developer experience, and ensure its dependability and functionality. 
% The Yerbamate framework has the potential for further expansion and refinement, providing opportunities to augment its functionality with more dependable, searchable and queryable models to boost its utility for researchers.

% The Yerbamate framework represents an innovative and contemporary approach to open science and AI development, featuring a modular design that promotes reproducibility, reusability, transparency, and collaboration in the research community. The framework adopts a contemporary open science methodology, that combines open source and open science by using version control (git) and community-based discussion platforms around git for open question answering, feedback, and issue tracking. The framework is continuously updated to address limitations, improve developer experience, and ensure its functionality and reliability. 
% The Yerbamate framework has the potential for continuous improvement and expansion, allowing for the addition of more searchable and queryable models to provide further utility for researchers.
% The Yerbamate framework presents a contemporary approach to open science and AI development by integrating best software engineering practices with open source principles. The framework's modular design enhances reproducibility, transparency, and collaboration within the research community, and the use of a community-based discussion platform enables open question answering, feedback, and issue tracking. These features contribute to the promotion of open science principles and encourage community involvement in the development of AI research. Version 1.0.0 of the framework is available for use, and with sufficient resources, it has the potential for ongoing improvement and growth. The framework can be updated and expanded to address limitations, enhance functionality and utility, and improve the developer and user experience. Furthermore, additional resources and tutorials can be created to further disseminate knowledge on software engineering in the context of AI.

% The Yerbamate framework represents an innovative approach to open science and AI development, featuring a modular design that promotes reproducibility, transparency, and collaboration in the research community. Additionally, the framework leverages a collaborative, community-based discussion platform for open question answering, feedback, and issue tracking, further encouraging community involvement and promoting open science principles. The framework is continually updated to address limitations and improve developer experience, and version 1.0.0 is currently available for use. Detailed documentation of the API and usage can be found on the project's website, with additional tutorials planned for the future.
% The Yerbamate framework presents a novel approach to open science and AI development by combining software engineering best practices with open source principles to create a general-purpose framework that enhances reproducibility, transparency, and collaboration. The framework's modular design allows for reusable and configurable modules for various components of the development process, including models, trainers, and data loading and prepossessing. This approach addresses the challenges of reproducibility in AI research by standardizing the structure of AI projects, promoting code reuse, and facilitating the sharing of research findings.

