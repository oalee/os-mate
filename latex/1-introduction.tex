
\section{Introduction}


 In recent years, AI has made remarkable progress, finding numerous applications across various fields such as natural language (\cite{gpt}), general intelligence(\cite{gato}), art (\cite{diffusion}), music (\cite{musiclm}), healthcare (\cite{aihealthcare}), finance (\cite{bao2022fraudartificial}), education (\cite{aieducation}), and weather forecasting (\cite{weather}). The potential benefits of AI are immense (\cite{beneficialai,potencialaibenefit}), their development and deployment require quality data (\cite{lecun2015deep}), code, and sound software engineering practices (\cite{se4dl,amershi2019software}). Poor software engineering practices can lead to a reproducibility crisis, undermining the reliability and credibility of AI systems (\cite{leakage-recrisis}). On the other hand, good software engineering practices can enhance the developer experience, reliability, maintainability, and replicability of AI systems (\cite{se4dl,amershi2019software, wan2019does}). In order to realize the full potential of AI, researchers and practitioners require open source tools and frameworks that are user-friendly, customizable, and promote reproducibility in the development of AI projects (\cite{lu2022softwareAIReponse,li2018can,wolf2020designing,olson2018system,ong2021guide,gundersen2018reproducible}). This is particularly important for ensuring that the benefits of AI are accessible to all and that the development of AI aligns with open science and ethical principles, enabling the sharing of research findings and supporting the progress and implementation of diverse AI applications for societal benefits (\cite{coro2020open,braun2018open, mittelstadt2016ethics,floridi2018ai4people,ong2021guide}).

The Yerbamate framework presents a novel approach to open science and AI development by combining software engineering best practices with open source principles to create a general-purpose framework that enhances reproducibility, transparency, and collaboration. The framework's modular design allows for reusable and configurable modules for various components of the development process, including models, trainers, and data loading and prepossessing. This approach addresses the challenges of reproducibility in AI research by standardizing the development and management of AI projects, promoting code reuse, and facilitating the sharing of research findings.

