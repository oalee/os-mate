


\section{Methodology}

The methodology section of this study includes the framework design and open science methodology employed in the development of the Yerbamate framework.
This study aimed to address the knowledge gap in software engineering for AI by developing a framework that prioritizes open science and improves the developer experience. 

\subsection{Open Source Open Science Methodology}

This study introduces and adopts an open source open science methodology that combines both principles to promote transparency, reproducibility, and collaboration in research. The open science methodology employed in this study involves the development of the open source git repository\footnote{\url{https://github.com/oalee/yerbamate}} for the Python framework Yerbamate and the creation of an open source open science git repository\footnote{\url{https://github.com/oalee/os-yerbamate}} for Yerbamate. The progress history of both repositories is publicly available, enhancing transparency in the research process and enabling open access to contributions, questions, feedbacks, ideas, and discussions.
In addition, the methodology involves disseminating knowledge at different levels of complexity. The Yerbamate framework's documentation and associated tutorial serve as an example of such knowledge dissemination. Furthermore, a Medium article titled "The Ultimate Deep Learning Project Structure: A Software Engineer’s Guide into the Land of AI"\footnote{\url{https://medium.com/@alee.rmi/c383f234fd2f}} has been published to disseminate knowledge of software engineering in a more accessible manner.

The Yerbamate framework and associated repositories are continuously updated to address limitations, bugs, and improve the developer experience, ensuring that it remains relevant and useful to the research community. 
% \subsection{Open Science}

% The methodology of open science employed in this study involves the development of the open-source git repository and framework Yerbamate\footnote{\url{https://github.com/oalee/yerbamate}} and the creation of an open source open science git repository\footnote{\url{https://github.com/oalee/os-yerbamate}} for Yerbamate. The progress history of both repositories is publicly available, and the repositories provide open access to contributions, question answers, ideas, and discussions. Moreover the methodology involves disseminating knowledge at different complexities. The Yerbamate framework's documentation and associated tutorial serve as an example of such knowledge dissemination, additionally
%   a Medium article titled "The Ultimate Deep Learning Project Structure: A Software Engineer’s Guide into the Land of AI"\footnote{\url{https://medium.com/@alee.rmi/c383f234fd2f}} has been published to disseminate knowledge of software engineering in a more accessible manner.
%  Additional tutorials with varying formats and complexity levels can be created in the future.
%  The Yerbamate framework and associated repositories are continuously updated to address limitations, bugs, and improve the developer experience, ensuring that it remains relevant and useful to the community.


% The article serves as an example of a tutorial that offers a simple and easy-to-understand introduction to software engineering for AI, explaining modularity and its application in the context of AI. Detailed documentation of the API and usage can be found on the Yerbamate repository\footnote{\url{https://oalee.github.io/yerbamate}}. With sufficient resources, additional tutorials can be created at various complexity levels to promote open science principles and disseminate knowledge on software engineering in the context of AI. These tutorials can be in different formats and provide more in-depth information on various aspects of the framework
% % The methodology of open science in this work encompasses the development of the open-source framework Yerbamate\footnote{\url{https://github.com/oalee/yerbamate}} and the creation of an open science report\footnote{\url{https://github.com/oalee/os-yerbamate}} for Yerbamate. Both repositories' progress history is publicly available, and the repositories provide open access to contributions, question answers, ideas, and discussions using a platform to enhance open science and encourage collaboration. A Medium article titled "The Ultimate Deep Learning Project Structure: A Software Engineer’s Guide into the Land of AI"\footnote{\url{https://medium.com/@alee.rmi/c383f234fd2f}} has been published to disseminate knowledge of software engineering in a more accessible manner, incorporating humor and memes with the aim of enhancing learning (\cite{powell1985humour}).
% This tutorial provides a simple and easy-to-understand introduction to software engineering for AI, explaining modularity and its application in the context of AI. Detailed documentation of the API and usage can be found on our repository\footnote{\url{https://oalee.github.io/yerbamate}}. 
% The Yerbamate framework and its associated open-source repositories are continuously updated to address limitations, bugs, and improve developer experience. These updates ensure that the framework remains current and reliable, while also incorporating feedback from the AI research community to further enhance its functionality and utility. Additionally, more tutorials will be added in the future with enough resources to promote open science.


% The methodology of open science in this work encompasses the development of the open-source framework Yerbamate\footnote{\url{https://github.com/oalee/yerbamate}} and the creation of an open science report\footnote{ \url{https://github.com/oalee/os-yerbamate}} for Yerbamate. Both repositories' progress history is publicly available, and the repositories provide open access to contributions, question answers, ideas, and discussions using a platform to enhance open science and encourage collaboration. A Medium article titled "The Ultimate Deep Learning Project Structure: A Software Engineer’s Guide into the Land of AI"\footnote{\url{https://medium.com/@alee.rmi/c383f234fd2f}} has been published to disseminate knowledge of software engineering in a more accessible manner, incorporating humor and memes with the aim of enhance learning (\cite{powell1985humour}). This tutorial provides a simple and easy-to-understand introduction to software engineering for AI, and explaining modularity and its application in the context AI. Detailed documentation of the API and usage can be found on our repository\footnote{\url{https://oalee.github.io/yerbamate}}. The Yerbamate framework and its associated open-source repositories are continuously updated to address limitations, bugs, and improve developer experience. These updates ensure that the framework remains current and reliable, while also incorporating feedback from the AI research community to further enhance its functionality and utility.



\subsection{Framework Design}

The methodology for developing this framework involved analyzing existing open-source python AI projects, reviewing the literature on software engineering practices, and studying modularity, separation of concerns, and open science in AI. The resulting framework prioritizes the creation of independent modules that are standalone and reusable components, which can be used across different projects.

\subsubsection{Experiment Configuration}

Defining hyperparameters and experiments plays a crucial role in AI development and optimization \cite{wu2019hyperparameter}. After evaluating various formats, it was decided that Python is the most suitable format for defining experiments in AI python projects due to its expressiveness, versatility, and straightforward syntax. The integration of existing code and libraries in Python reduces overhead and enhances the experimentation process. Python's powerful libraries and modules enable the exploration of a wider range of hyperparameters and models, leading to a more comprehensive understanding of the problem at hand. The use of well-documented Python code improves the readability of the configuration file, making it more accessible. Additionally, the format is executable directly with Python, which makes it Turing complete, capable of expressing any algorithm, enhancing its flexibility and power.


\subsubsection{Flexibility and Customization}

One of the key challenges in the development of the Yerbamaté framework was ensuring flexibility and customization to meet the varying needs of researchers and practitioners in the field of AI. To address this challenge, Yerbamaté provides increased flexibility by not imposing any restrictions on additional module names, enabling researchers to utilize their preferred module names for custom tasks. Additionally, the framework is designed to be compatible with Python, providing greater flexibility as Python can directly be used to execute experiments and Python files.

\subsection{Evaluation}

The Yerbamaté framework was tested on various open-source AI projects across different frameworks and libraries, including Jax, Flax, Pytorch, Pytorch\-Lightning, Tensorflow, Keras, and Huggingface, to evaluate its effectiveness. The framework's flexibility was demonstrated through various use cases, such as particle swarm optimization (\cite{kennedy1995particle}) and a computer vision experiment with RANSAC (\cite{lowe2004distinctive}). Furthermore, the framework's effectiveness was evaluated through a case study involving the refactoring of the official implementation of Big Transfer (BiT) to conform to modularity. Code samples, installable modules, examples, and the results of the case study are available in the Appendix.
