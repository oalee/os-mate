\documentclass{IEEEtran}
\usepackage[utf8]{inputenc}
\usepackage[sorting = none]{biblatex}
\usepackage{amsmath,amssymb,amsfonts}
\usepackage{algorithm}
\usepackage{algpseudocode}
\usepackage{graphicx}
\usepackage{textcomp}
\usepackage[table]{xcolor}
\usepackage{array}
\usepackage{hyperref}           % page numbers and '\ref's become clickable
\usepackage{todonotes}
\usepackage{float}          % for forcing images to stay where they should
\usepackage{multirow}       % for using multirow in tables
\usepackage{tikz}
\usepackage{tabularx}
% \usepackage{adjustbox}
\usepackage{longtable}
\usepackage[edges]{forest}
\usepackage[framemethod=TikZ]{mdframed}
\usepackage{amsmath}
\usepackage{caption}
\usepackage{subcaption}
\usepackage{MnSymbol}
\usepackage{tipa}
\renewcommand{\algorithmicrequire}{\textbf{Input:}}
\renewcommand{\algorithmicensure}{\textbf{Output:}}
\usepackage{dirtree}

\usepackage[export]{adjustbox}
% \usepackage{caption}
\usepackage{xurl}

\mdfdefinestyle{Frame}{
    linecolor=black,
    outerlinewidth=0.3pt,
    roundcorner=2pt,
    innertopmargin=\baselineskip,
    innerbottommargin=\baselineskip,
    leftmargin =1cm,
    rightmargin =1cm,
    backgroundcolor=white}
    \mdfdefinestyle{Frame_NoMargin}{%
    linecolor=black,
    outerlinewidth=0.3pt,
    roundcorner=2pt,
    innertopmargin=\baselineskip,
    innerbottommargin=\baselineskip,
    backgroundcolor=white}
    \mdfdefinestyle{InnerFrame_NoMargin}{%
    linecolor=black,
    outerlinewidth=0.1pt,
    roundcorner=0.5pt,
    % innertopmargin=\baselineskip,
    % innerbottommargin=\baselineskip,
    leftmargin =-0.1cm,
    innerleftmargin = 0.05cm,
    rightmargin =0cm,
    backgroundcolor=white}  
    \mdfdefinestyle{InnerFrameBig_NoMargin}{%
    linecolor=black,
    outerlinewidth=0.1pt,
    roundcorner=0.5pt,
    % innertopmargin=\baselineskip,
    % innerbottommargin=\baselineskip,
    leftmargin =-0.22cm,
    innerleftmargin = 0.05cm,
    rightmargin =-0.22cm,
    backgroundcolor=white}
      \mdfdefinestyle{Frame_NoInnerMargin}{%
    linecolor=black,
    outerlinewidth=0.3pt,
    roundcorner=2pt,
    innerleftmargin = 0.02cm,
    innertopmargin=\baselineskip,
    innerbottommargin=\baselineskip,
    backgroundcolor=white}
      \mdfdefinestyle{Frame_Blank}{%
    linecolor=white,
    outerlinewidth=0.0pt,
    roundcorner=0pt,
    innerleftmargin = 0.3cm,
        innertopmargin=0cm,
    innerbottommargin=0.2cm,
    backgroundcolor=white}    

%%%%%%%%%%%%%%%%%%%%%%%
% Some commands       %
%%%%%%%%%%%%%%%%%%%%%%%
% Notes
\newcommand{\note}[1]{\todo[inline]{#1}}
\newcommand{\urgent}[1]{\todo[inline, color=red]{#1}}
\newcommand{\revision}[1]{\todo[inline, color=green]{#1}}
\newcommand{\fix}[1]{\todo[inline, color=yellow]{#1}}
% Tables
\newcolumntype{L}[1]{>{\raggedright\let\newline\\\arraybackslash\hspace{0pt}}m{#1}}
\newcolumntype{C}[1]{>{\centering\let\newline\\\arraybackslash\hspace{0pt}}m{#1}}
\newcolumntype{R}[1]{>{\raggedleft\let\newline\\\arraybackslash\hspace{0pt}}m{#1}}
\newcommand{\adaptcell}[1]{\parbox{0.95\columnwidth}{\vspace{0.5em}#1\vspace{0.5em}}}
% Math
\newcommand{\rarrow}{$\rightarrow\ $}
\newcommand{\larrow}{$\leftarrow\ $}
\newcommand{\lif}{\rightarrow}
\newcommand{\liff}{\leftrightarrow}
\newcommand{\la}{\forall}
\renewcommand{\le}{\exists}
\newcommand{\lxor}{\dot\vee}
\renewcommand{\ll}{\llbracket}
\renewcommand{\phi}{\varphi}
\newcommand{\rr}{\rrbracket}
\newcommand{\lk}{\square}
\newcommand{\lp}{\lozenge}
\newcommand{\argmin}[1]{\underset{#1}{\mathrm{argmin}}}
\newcommand{\argmax}[1]{\underset{#1}{\mathrm{argmax}}}
\newcommand{\Sum}[1]{\underset{#1}{\sum}}

% Others
\newcommand{\newcolumn}{\vfill\pagebreak}
\renewcommand{\b}[1]{\textbf{#1}}
\renewcommand{\it}[1]{\textit{#1}}
\renewcommand{\u}[1]{\underline{#1}}

%%%%%%%%%%% STUFF TO PRINT SEMANTIC TABLEAUX %%%%%%%%%%
% \usepackage{tikz}
% \usetikzlibrary{positioning,arrows,calc, trees, automata}

% \usepackage{minted}
\usepackage[outputdir=../]{minted}

% \usepackage[T1]{fontenc}

\usepackage[zerostyle=b,scaled=.75]{newtxtt}
\usemintedstyle{colorful}

\setminted[python]{breaklines, framesep=2mm, fontsize=\footnotesize, numbersep=5pt}

\usepackage{tabulary}

% \usepackage{xcolor}
% \definecolor{Text}{HTML}{000000}
% \AtBeginEnvironment{minted}{\color{Text}}


\tikzset{
	modal/.style={>=stealth',shorten >=1pt,shorten <=1pt,auto,node distance=1.5cm,semithick},world/.style={circle,draw,minimum size=0.5cm,fill=gray!15},point/.style={circle,draw,inner sep=0.5mm,fill=black},reflexive above/.style={->,loop,looseness=7,in=120,out=60},reflexive below/.style={->,loop,looseness=7,in=240,out=300},reflexive left/.style={->,loop,looseness=7,in=150,out=210},reflexive right/.style={->,loop,looseness=7,in=30,out=330},none/.style={}
}

\forestset{%
	declare toks={T}{},
	declare toks={F}{},
	my label/.style={%
		tikz+={%
			\path[late options={%
				name=\forestoption{name},label={#1}}
			];
		}
	},
	tableaux/.style={%
		forked edges,
		for tree={
			math content,
			parent anchor=children,
			child anchor=parent,
		},
		where level=0{%
			for children={no edge},
			phantom,
		}{%
			before typesetting nodes={%
				content/.wrap value={\circ},
			},
			delay={%
				my label/.wrap pgfmath arg={{[inner sep=0pt, xshift=-3.5pt, yshift=3.5pt, anchor=north west, font=\scriptsize]-45:$##1$}}{content()},
				insert before/.wrap pgfmath arg={%
					[{##1}, no edge, math content, before drawing tree={x'+=7.5pt}]
				}{T()},
				insert after/.wrap pgfmath arg={%
					[{##1}, no edge, math content, before drawing tree={x'-=7.5pt}]
				}{F()},
			},
			if={n_children("!u")==1}{%
				before packing={calign with current edge},
			}{}
		},
	}
}

\forestset{
  smullyan tableaux/.style={
    for tree={
      math content
    },
    where n children=1{
      !1.before computing xy={l=\baselineskip},
      !1.no edge
    }{},
    closed/.style={
      label=below:$\times$
    },
  },
}
%%%%%%%%%%%%%%%%%%%%%%%%%%%%%%%%%%%%%%%%%%%%%%%%%%%%%%%



%%%%%%%%%%%%%%%%%%%%%%%%%%%%%%%%%%%%

\addbibresource{ref.bib}
\def\BibTeX{{\rm B\kern-.05em{\sc i\kern-.025em b}\kern-.08em
    T\kern-.1667em\lower.7ex\hbox{E}\kern-.125emX}}





% \title{ \Huge \textbf{Yerbamaté: A Modular and Open Science Convention for Python-based AI Projects} \\[0.5cm]}

\title{ \Huge \textbf{Yerbamaté: An Open Science Python Framework} \\[0.3cm]
% \small{v0.9}

}

\author{
\begin{tabular}{ll}
Ali Rahimi
\end{tabular} \bigskip \\

\textit{Maastricht University} \\
\textit{Department of Advanced Computing Sciences}\\
\textit{Maastricht, The Netherlands}\\

}
\date{January 2023}
\addbibresource{ref.bib}


\begin{document}

\maketitle



\begin{abstract}
% This paper introduces Yerbamate\footnote{\url{https://github.com/oalee/yerbamate}}, an open science\footnote{To promote the principles of open science, the progress history of this paper is available on GitHub at the following URL: \url{https://github.com/oalee/os-yerbamate}} framework and software engineering convention for Python-based AI research projects. Yerbamate prioritizes modularity and separation of concerns to promote reproducibility, reusability, shareability, maintainability, and code quality. The convention and framework can be applied to a variety of Python-based tasks, including experimentation with machine learning, deep learning, genetic algorithms, optimizations, and simulations and analysis. Yerbamate simplifies the implementation of the convention with straightforward commands for installing, experimenting, and running code. The framework is compatible with all Python libraries and AI/ML frameworks, and it fosters open science by this convention enabling sharing of code modules such as models and trainers between practitioners. The adoption of this convention and framework can enhance collaboration, help address the reproducibility crisis in AI and enhance the developer experience while promoting open science and accessible AI.

This paper presents Yerbamate\footnote{\url{https://github.com/oalee/yerbamate}}, an open science\footnote{
As an open science paper, Yerbamate strives to promote the principles of transparency and collaboration. To this end, the history of the LaTeX files for this paper are available on GitHub: \url{https://github.com/oalee/os-yerbamate}. These open science repositories are open to collaboration and encourage participation from the community to enhance the validity, reproducibility, accessibility, and quality of this work.
} framework and software engineering convention designed for Python-based AI research projects. Yerbamate emphasizes modularity and separation of concerns to promote reproducibility, reusability, shareability, maintainability, and code quality. The framework is applicable to a range of Python-based tasks, including experimentation with machine learning, deep learning, genetic algorithms, optimizations, and simulations and analysis. Yerbamate simplifies the implementation of the convention with straightforward commands for installing, experimenting, and running code. The framework is compatible with all Python libraries and frameworks, and it fosters open science by enabling the sharing of code modules such as models and trainers between practitioners. The adoption of Yerbamate can facilitate collaboration, enhance the developer experience, and promote open science while contributing to addressing the reproducibility crisis in the field of AI.
\end{abstract}





% This paper presents a software engineering convention, accompanied by a framework named Yerbamate, for Python-based artificial intelligence (AI) projects. The convention emphasizes modularity and separation of concerns, with the goal of promoting reproducibility, reusability, shareability, maintainability, and code quality. The use of modular code structures enables flexibility and scalability, facilitating the reuse of code and its components. The Yerbamate framework simplifies the implementation of the convention by providing a set of simple and consistent commands for installing, experimenting, and running code. The convention and framework are compatible with all Python libraries, AI and ML frameworks, and can be used for a variety of Python-based experimentation, analysis, ML, AI, genetic algorithms, particle swarm optimization, and simulations. The use of this convention and framework can address the reproducibility crisis in AI and promote open science by enabling the sharing of code modules among researchers and practitioners



%   In recent years, deep learning has emerged as a powerful tool for solving complex problems in various fields, such as image and speech recognition, natural language processing, and computer vision (\cite{lecun2015deep}). 
% Deep learning projects, whilst still considered software engineering, differ from traditional software engineering in that they revolve around the creation of models that can extract knowledge from data and make predictions or decisions, as opposed to relying on a pre-defined set of instructions (\cite{lecun2015deep,amershi2019software,wan2019does,se4dl}). These disparities make some phases such as the testing phase of deep learning projects distinct from traditional software (\cite{wan2019does}). Still, principles and design patterns commonly used in software engineering can still aid in developing and maintaining deep learning software (\cite{amershi2019software,wan2019does,se4dl}). 

% Artificial intelligence (AI) and machine learning (ML) have become increasingly powerful in recent years, with applications in fields ranging from natural language, art, music, healthcare, finance to education. While these technologies have the potential to revolutionize many areas of human life, their successful development and deployment depend heavily on quality code and sound software engineering practices. Good software engineering practices can increase the reliability, maintainability, and scalability of AI and ML systems, enabling their widespread adoption and use.

\section{Introduction}


 In recent years, AI has made remarkable progress, finding numerous applications across various fields such as natural language (\cite{gpt,gato}), art (\cite{diffusion}), music (\cite{musiclm}), healthcare (\cite{aihealthcare}), finance (\cite{bao2022fraudartificial}), education (\cite{aieducation}), and weather forecasting (\cite{weather}). The potential benefits of AI are immense (\cite{beneficialai,potencialaibenefit}), their development and deployment require quality data (\cite{lecun2015deep}), code, and sound software engineering practices (\cite{se4dl,amershi2019software}). Poor software engineering practices can lead to a reproducibility crisis, undermining the reliability and credibility of AI systems (\cite{leakage-recrisis}). On the other hand, good software engineering practices can enhance the developer experience, reliability, maintainability, and replicability of AI systems (\cite{se4dl,amershi2019software, wan2019does}). In order to realize the full potential of AI, researchers and practitioners require tools and frameworks that are user-friendly, customizable, and promote reproducibility and openness in the development of AI projects (\cite{lu2022softwareAIReponse,li2018can,wolf2020designing,olson2018system,ong2021guide,gundersen2018reproducible}). This is particularly important for ensuring that the benefits of AI are accessible to all and that the development of AI aligns with open science and ethical principles, enabling the sharing of research findings and supporting the progress and implementation of diverse AI applications for societal benefits (\cite{coro2020open,braun2018open, mittelstadt2016ethics,floridi2018ai4people,ong2021guide}).

% In recent years, AI has made remarkable progress, finding numerous applications across a wide range of fields, including natural language processing, art, music, healthcare, finance, education and weather forecasting. These technologies have the potential to revolutionize many areas of human life, leading to significant societal and economic benefits. However, their successful development and deployment depend heavily on quality data, code, and sound software engineering practices. Poor software engineering practices can result in the reproducibility crisis in AI research, negatively affecting the credibility and reliability of AI systems. Good software engineering practices, on the other hand, can increase the reliability, maintainability, and replicability of AI systems, enhancing the developer experience. Developers and researchers in the field of AI require tools and frameworks that are user-friendly, customizable, and provide a seamless development experience. These tools and frameworks should promote, reproducibility, and openness in the development of AI projects, facilitating the sharing of research findings and supporting the implementation of diverse AI applications.

% In recent years, artificial intelligence (AI) and machine learning (ML) have made remarkable progress and found numerous applications across a wide range of fields, including natural language processing, art, music, healthcare, finance, and education. 
% While these technologies have the potential to revolutionize many areas of human life, their successful development and deployment depend heavily on quality data, code and sound software engineering practices. Good software engineering practices can increase the reliability, maintainability, and replicability of AI systems, and enhance the developer experience. 

% In this paper, we propose a software engineering convention for AI and ML that emphasizes modularity, and separation of concerns, promoting reproducibility, maintainability, and code quality. This convention enhances open science by enabling the sharing of standalone code modules, such as models, trainers, and data loaders, among researchers and practitioners. By promoting modular design, this convention allows for increased flexibility and reusability of code, making it easier to reproduce and build upon previous research.

% In this paper, we present a software engineering convention of modularity that enhances reproduciblility, maintainability, code quality, and enhances open science through the sharing of standalone code modules such as models, trainers, and data loaders between researchers and practitioners. 

% The proposed convention addresses the challenges associated with the current state of software engineering practices in the AI and ML fields and has the potential to facilitate the growth and adoption of new technologies.
% However, the current state of software engineering in the AI and ML fields leaves much to be desired. Many AI and ML systems are developed in an ad hoc manner, with limited attention to testing, documentation, version control, and other fundamental software engineering practices. This lack of attention to good software engineering can result in buggy, hard-to-maintain systems that invalidate research and impede progress. Furthermore, the lack of standardization in the AI and ML fields can make it difficult for researchers to collaborate and share their work effectively.


\section{Motivation}

Despite the significant progress in artificial intelligence (AI) and machine learning (ML) in recent years, the lack of standardized software engineering practices for these fields remains a significant challenge. The absence of a software engineering convention for AI  has led to several issues, including the presence of bugs and invalidation of research, making maintenance difficult, and hindering the wider adoption of AI systems. Open source research projects typically create their command line tools to experiment with different models and hyperparameters, which can lead to comprehensive options but often result in limitations. Moreover, AI  researchers typically implement their hyperparameter selection and experiment execution which leads to reinvention of the wheel and a potential learning curve.

This heterogeneity in software engineering practices poses a significant obstacle to code reuse and collaboration among researchers. In some cases, AI codes are of low quality, resembling spaghetti code that is hard to maintain. In contrast, others exhibit a more modular design, which can enhance code quality and maintainability. However, the problem arises when attempting to use another researcher's model, data augmentation, or a specific approach, as it requires learning a new framework for hyperparameter configuration and execution. The lack of standardized software engineering practices increases the difficulty of reproducing and building upon previous research.
% To address these issues, we present a software engineering convention of modularity that enhances maintainability, code quality, and enables the sharing of models, trainers, data loaders, and code components between researchers and practitioners.

% Our research suggests that establishing standard software engineering practices for AI  can enhance collaboration and sharing between researchers and practitioners. This will make it easier to replicate and build upon previous work, leading to faster innovation and progress in the field. By adopting good software engineering practices, AI  systems can become more reliable, maintainable, and scalable, enabling their widespread adoption and use.


% The widespread adoption of the Python programming language in AI programming has been noted in recent studies (\cite{mihajlovic2020use,raschka2020machine}). As such, this project focuses on the examination of software engineering practices in the context of Python and its machine learning frameworks. Many interdisciplinary researchers may lack the software engineering expertise necessary to manage their code bases to ensure correctness, understandability, extendability, and reusability (\cite{amershi2019software,scully-debt-ml,leakage-recrisis,accountabilityInAi}).
% To address these challenges, we investigated the application of software engineering principles of separation of concerns (SoC) and modular design in Python to AI projects. The goal of this study was to improve the reproducibility and reusability of python AI components through the adoption of best software engineering practices. Software engineering is the process of designing, developing, and maintaining software systems efficiently and reliably (\cite{pressman2010software}).
% In this internship, I explored software engineering practices that could help researchers share and collaborate AI code.


\section{Background Information}


\subsection{Crisis of Reproducibility in AI}
The crisis of reproducibility in AI refers to the difficulty in reproducing the results of AI research (\cite{gundersen2018reproducible}). The lack of transparency in data collection and research has greatly contributed to the crisis of reproducibility in AI (\cite{gundersen2018reproducible,hutson2018artificial,leakage-recrisis}). Many AI models are developed close sourced using proprietary data and methods, making it difficult for others to replicate the research and understand the inner workings of the models (\cite{gundersen2018reproducible,accountabilityInAi}). Additionally, the lack of transparency in the data collection process can lead to issues such as biased or unreliable data, which can further undermine the credibility and reproducibility of the research, and it can decrease the trust in the field as the results of the research are not independently verifiable  (\cite{accountabilityInAi,leakage-recrisis,scully-debt-ml}). The pressure to publish results and the lack of incentives to share data and code can discourage researchers from making their work easily reproducible. (\cite{psychology-reproducibility-crisis, friesike2015open,kwon2021incentive, ali2017motivating,o2017evaluation})


\subsection{Open Science}

Open science is a research methodology that prioritizes transparency, collaboration, and reproducibility (\cite{nielsen2011reinventing}). The promotion of open science in the field of AI has garnered considerable attention in recent years (\cite{accountabilityInAi,gundersen2018reproducible,leakage-recrisis,scully-debt-ml,stodden-towardreprodicibleresearch,coro2020open,braun2018open,hicks2021open,burgelman2019open}). In the field of AI, open science practices can help to address concerns about biased or unreliable data, as well as provide a way for researchers to collaborate and build reproducible research and enhance accountability in AI (\cite{accountabilityInAi,stodden-towardreprodicibleresearch}).  Open science encourages researchers to share their knowledge, data, code, and detailed documentation of their methods (\cite{hutson2018artificial,accountabilityInAi}). 
Researchers can also use open-source frameworks and standard evaluation metrics (\cite{gundersen2018reproducible}) to facilitate reproducibility. Furthermore, the scientific community can encourage reproducibility by valuing it in the peer-review process (\cite{scully-debt-ml}), and by giving credit to researchers who share their data and code (\cite{scully-debt-ml,credit-datasharing,stodden-towardreprodicibleresearch}).




\subsection{The Significance of Data for AI}

Data is a crucial factor in the success of deep learning models (\cite{lecun2015deep}). The quality, pre-processing, and augmentations applied to the data can significantly impact the model's ability to extract knowledge and make accurate predictions (\cite{shorten2019survey}). Therefore, it is essential for researchers to consistently use the same data split, pre-processing and augmentations when comparing models to ensure fair comparisons (\cite{caton2020fairness,mehrabi2021survey, leakage-recrisis}). Furthermore, if the data collection process is not transparent and well-documented, it can lead to issues such as biased or unreliable data, which can harm the credibility and reproducibility of the research (\cite{accountabilityInAi}). 
 This is highlighted in the seminal publication "Towards Accountability for Machine Learning Datasets: Practices from Software Engineering and Infrastructure," which emphasizes the importance of incorporating open science principles into the development life cycle of AI datasets and the documentation process for researcher collaboration (\cite{accountabilityInAi}).



% Additionally, bugs in the data-related code or leakage between the test and train sets can also invalidate the results of a study (\cite{leakage-recrisis}).




% \subsection{Open Science}

\subsection{The Significance of Open Science for AI}

 
Open science represents a crucial component in the pursuit of responsible and trustworthy AI (\cite{floridi2019establishing,coro2020open,braun2018open,hicks2021open}). By prioritizing transparency and reproducibility, researchers in the field can advance its development in a safer and trustworthy manner (\cite{coro2020open,floridi2018ai4people,kocak2022transparency,stodden-towardreprodicibleresearch}).
Open science practices serve to mitigate the risks associated with closed-source AI and big data bias, which have raised significant concerns among stakeholders (\cite{batarseh2020data, o2017weapons}). Adoption of open science and open source AI can increase the fairness and impartiality of AI models (\cite{stodden-towardreprodicibleresearch,accountabilityInAi,gundersen2018reproducible}) and enhance the credibility and trustworthiness of their outputs among stakeholders and decision-makers (\cite{goodman2017european,hsiao2018vtaiwan,praprotnikevaluation}). % 
% Beyond that, close-sourced AI and bias in big data can have detrimental effects and increases inequality, and threaten democracy (\cite{o2017weapons}).  
\subsection{Software Engineering}
% Software engineering is the process of designing, developing, and maintaining software systems efficiently and reliably (\cite{pressman2010software}).
Software engineering is a well-established discipline that encompasses the process of designing, developing, testing, and maintaining software systems with a focus on quality, reliability, and efficiency \cite{pressman2010software}. While the specific activities and methodologies involved in software engineering can vary depending on the type of software, the principles of good software engineering practices are generally applicable across all types of software (\cite{pressman2010software}), including those in the field of artificial intelligence (\cite{se4dl,wan2019does,martinez2022softwareAI,davis2011understandingmodularity}). The software engineering practices employed in the development of AI systems include, but are not limited to, testing, debugging, documentation, version control, and code review. Additionally, given the complex and evolving nature of AI systems, specific attention must be given to software requirements and their evolution over time (\cite{heyn2021requirement,belani2019requirements}). However, unlike traditional software engineering, the requirements of AI systems may not always be well-defined, and software engineering practices may need to be adapted to the rapidly changing needs of these systems(\cite{heyn2021requirement,belani2019requirements}). 

\subsubsection{Developer Experience}
The concept of developer experience (DX) is a multidimensional construct that refers to developers' perceptions of various aspects of the development process, including the usability and effectiveness of the tools, frameworks, and platforms used. DX is a critical factor in software development as it has the potential to impact productivity, motivation, and satisfaction (\cite{fagerholm2012developer}).

\subsubsection{Developer Experience in AI}
 The development of AI systems is a complex and challenging task that involves a range of tasks, including data preprocessing, model selection and training, and performance evaluation (\cite{lecun2015deep}). Given the complexity of the development process, researchers and practitioners in the field of AI require tools and frameworks that are user-friendly, customizable, and provide a seamless development experience (\cite{wolf2020designing,li2018can,olson2018system}.


\subsubsection{Separation of Concerns}
 The separation of concerns (SoC) is a software engineering principle that suggests that different aspects of a system should be separated into distinct components, allowing for increased clarity, maintainability, and scalability of code (\cite{pressman2010software, de2002importance}). In the context of AI, this principle can be applied by separating the different stages of a machine learning pipeline into distinct, reusable components (\cite{mo2016decoupling,mo2016decoupling,pressman2010software, de2002importance}). By adhering to SoC, researchers can improve the clarity of their code and reduce the risk of introducing bugs and errors (\cite{mo2016decoupling,mo2016decoupling,pressman2010software, de2002importance}).

\subsubsection{Modularity}
Modularity, or the practice of creating reusable components, is a fundamental aspect of software engineering (\cite{pressman2010software}). By breaking down complex systems into smaller, reusable components, researchers can improve the understandability and extendability of their code. Additionally, modular code is more easily testable and maintainable, leading to increased reproducibility and reliability of results (\cite{amershi2019software,pressman2010software}). 
\subsubsection{Modualirty in Python}
In Python, modular design can be achieved through the use of functions, modules, and libraries (\cite{sanner1999python}). 
A python module contains definitions, functions, classes, and variables (\cite{raschka2015python}). By convention, modules are stored in separate directories, and a directory containing one or more modules is called a package. The presence of \verb|__init__.py| file in a package directory indicates that it is a package, and all files in the directory are considered modules of that package. In other words, the \verb|__init__.py| file makes the directory it's in a Python package, and any code in that file is executed when the package is imported.


\subsection{Seperation of Concerns for AI}

% In the field of software engineering, including artificial intelligence and machine learning, decoupling concerns, also known as separation of concerns, is a crucial design principle that helps to improve the maintainability, scalability, and reusability of code (\cite{mo2016decoupling,qian2006decoupling, pressman2010software}. The goal of decoupling concerns is to break down complex systems into smaller, modular components that can be independently developed, tested, and maintained (\cite{pressman2010software, mo2016decoupling, qian2006decoupling}). 

Decoupling concerns is a crucial design principle in the field of artificial intelligence and machine learning (\cite{mo2016decoupling,qian2006decoupling, pressman2010software}. For instance, in a typical deep learning experiment, the trainer component is responsible for training the model. The trainer component can be designed to receive either a string representing the dataset/model names or the actual dataset/model objects, with the latter approach providing greater flexibility and customization. Another example of separation of concerns in AI is the data loading and augmentation process, which can be hardcoded into the data loading module or passed as an object to a function. Similarly, a deep learning model can be implemented as a monolithic block of code or as a series of modular components, such as the encoder, decoder, and attention mechanism. The latter approach allows for greater flexibility and customization of the model, as each component can be modified or replaced without affecting the other components.




\subsection{Experiment Configuration}

In the field of artificial intelligence and machine learning, defining hyperparameters and experiments plays a crucial role in the development and optimization of models \cite{wu2019hyperparameter}. The choice of format for defining experiments and hyperparameters can greatly impact the efficiency and flexibility of the experimentation process. After evaluating various formats such as JSON and TOML, it has been determined that Python is the most suitable format for defining experiments in AI and ML python projects. Python provides a high level of expressiveness and versatility, allowing for easy modifications and adaptations of the experiment definition. In addition, Python offers a straightforward syntax for defining hyperparameters and allows for the integration of existing code and libraries. This greatly reduces the overhead associated with switching between different languages or systems, leading to a more streamlined and efficient experimentation process. Furthermore, the ability to use Python's powerful libraries and modules enhances the experimentation process by providing access to a vast range of tools and resources. This further enables the exploration of a wider range of hyperparameters and models, leading to a more comprehensive understanding of the problem at hand.

Additionally, by incorporating well-documented Python code, the readability of the configuration file is improved, making it more accessible to researchers and practitioners alike. Furthermore, this format is executable directly with Python, which makes it Turing complete. This property is particularly useful because it means that the format can include arbitrary computation and is capable of expressing any algorithm, enhancing its flexibility and power.


% \subsection{Deep Learning Frameworks in Python}

% The utilization of deep learning models in Python has seen the rise of several prominent frameworks, including Jax, Keras/TensorFlow, PyTorch, PyTorch Lightning, and Jax/Flax (\cite{mihajlovic2020use,raschka2020machine, nguyen2019machine, shatnawi2018comparative}). These frameworks provide a comprehensive set of tools and functionalities for the implementation and training of deep learning models (\cite{elshawi2021dlbench}). 
% However, choosing the right framework can be challenging as each framework has its own advantages and limitations (\cite{nguyen2019machine, elshawi2021dlbench,shatnawi2018comparative}).

% One approach to addressing this challenge is the development of unified deep learning frameworks such as Ivy (\cite{ivy}). Ivy supports multiple frameworks, enabling researchers to utilize the strengths of each framework without sacrificing compatibility and ease of use (\cite{ivy}). However, using such a unified framework also introduces overhead in learning, and computation, as well as limitations in terms of customizability. For instance, users are required to utilize Ivy tensors instead of PyTorch or Keras tensors, limiting access to low-level functionality (\cite{IvyDocs, ivy}).



\section{Research Questions}

% The objective of this internship at the start was to address the following research questions.

% \subsection{
% What should be the recommended conventions for software engineering in AI and machine learning to promote reproducibility, reusability, shareability, maintainability, and code quality?
% }
\subsection{
    What are the most effective software engineering practices for promoting reproducibility and open science in AI research?}
    Effective software engineering practices for promoting reproducibility and open science in AI and ML research include modularity, separation of concerns, documentation, version control, and testing. Modularity enables the reuse of code and components, while separation of concerns separates code into distinct modules based on their functionality, making it easier to maintain and modify. Documentation helps other researchers understand how to use and reproduce the code, while version control enables tracking of changes to the code over time. Testing helps ensure that the code works as intended and can be used by others. These practices can help promote open science by enabling the sharing of code and promoting transparency in research.
    
    
    
%     \subsection{How can software engineering principles be applied to AI development and experimentation in order to enhance the developer experience?}
    
%     The application of software engineering principles in AI development and experimentation can enhance the developer experience by promoting modularity, separation of concerns, and standardized naming conventions for modules such as models, data, and trainers. Standardized practices can result in more efficient, effective, and collaborative workflows, as well as easier sharing and reuse of code. Additionally, clear documentation, tutorials, and support resources can help users adopt and understand these practices, and open discussions, question-answer platforms can further improve the developer experience.
\subsection{
How can software engineering principles be utilized for AI to improve the developer experience?
}

Software engineering principles can be utilized to improve the developer experience in AI development and experimentation by emphasizing modularity, separation of concerns, and standardized naming conventions for modules, such as models, data, and trainers. Standardized project organization, code structure, and modularity enable easier collaboration and sharing among researchers and practitioners. Clear documentation, including contribution guides, tutorials, and support resources, can further facilitate understanding and collaboration, leading to an improved developer experience. Additionally, productivity tools, community driven discussions, open question-answer platforms, and issue tracking systems can be utilized to further enhance collaboration and improve the overall developer experience.

% Software engineering principles can enhance the developer experience in AI development and experimentation by prioritizing modularity, separation of concerns, and standardized naming conventions for independent and interchangeable modules such as models, data, and trainers. A standardized approach to project organization, code structure, and modularity facilitates collaboration and code sharing among researchers and practitioners. Clear documentation, including contribution guides, tutorials, and support resources can help users understand and collaborate, leading to an improved developer experience.

% Software engineering best practices in AI/ML can be utilized to improve the developer experience by prioritizing modularity, separation of concerns, and standardized naming conventions for independent and interchangeable modules such as models, data, and trainers. This approach allows for a standardized approach to project organization, code structure, and modularity, which can facilitate collaboration and code sharing among researchers and practitioners. Additionally, providing clear documentation and support resources can help users understand and adopt the convention, leading to an improved developer experience.


% Furthermore, the use of open-source technologies and practices can promote collaboration and sharing of code and models within the scientific community, leading to the development of more efficient and effective AI/ML research practices.
% Software engineering conventions in AI/ML can be designed to improve the developer experience and promote open science practices by prioritizing modularity, separation of concerns, and consistent naming conventions for independent and interchangeable modules such as models, data, and trainers. To ensure compatibility and customizability, the convention should be flexible enough to accommodate a wide range of use cases and project types. Clear documentation, training materials, and support resources can facilitate adoption of the convention, while open-source technologies and practices can foster collaboration and sharing of code and models within the scientific community. Additionally, the ability to upgrade the convention with input from the community can ensure that the convention remains relevant and adaptable to new technologies and emerging research practices. By adopting a well-designed software engineering convention, AI/ML developers can improve the efficiency and effectiveness of their research, while promoting open science practices and facilitating collaboration within the research community.

\subsection{How can we enhance open science to promote transparency, reproducibility, and collaboration in scientific research, and foster a more accessible and innovative research environment?}

Open science can be enhanced through the development of key tools and technologies that prioritize transparency, reproducibility, and collaboration in scientific research. These tools include open source frameworks, productivity tools for collaboration, open source software and hardware, open-access journals, repositories and data sharing platforms. To make these tools more accessible, it is important to provide clear documentation, training materials, and support resources, as well as promote interdisciplinary collaborations and knowledge sharing. A cultural shift towards open science practices can also be fostered by engaging stakeholders and the public in discussions around the benefits and challenges of AI and open science. By adopting these practices and making these tools more accessible, we can create a more open, transparent, and reproducible scientific community that promotes collaboration and innovation.

% Open science can be enhanced by developing key tools and technologies that promote transparency, reproducibility, and collaboration in scientific research. These tools include open source and open science frameworks, productivity tools for collaboration, open source software and hardware, open-access journals, repositories, data sharing, collaborative tools for writing, documentation, and version control. Making these tools more accessible to researchers and the wider scientific community can be done by providing clear documentation, training materials, and support resources, as well as promoting interdisciplinary collaborations and knowledge sharing. In addition, promoting a cultural shift towards open science practices by engaging stakeholders and the public in discussions around the benefits and challenges of open science can help to foster a more collaborative and innovative research environment. By adopting these practices and making these tools more accessible, we can create a more open, transparent, and reproducible scientific community.

% \subsection{What are the key tools and technologies needed to support open science, and how can they be made more accessible to researchers and the wider scientific community?}
% The key tools and technologies needed to support open science include open source frameworks, open-access journals, repositories, data sharing platforms, collaborative tools for writing, documentation and version control, and open source software and hardware. These tools can help to promote transparency, reproducibility, and collaboration in scientific research, and can contribute to the development of a more open and accessible research environment. To make these tools more accessible to researchers and the wider scientific community, it is important to provide clear documentation, training materials, and support resources, as well as promoting interdisciplinary collaborations and knowledge sharing. Additionally, adopting open science practices can require a cultural shift in the scientific community, and it is important to engage stakeholders and the public in discussions around the benefits and challenges of AI and open science. By promoting open science and making key tools and technologies more accessible, we can help to foster a more collaborative and innovative research environment.
    
% \subsection{
% In what ways does the Yerbamate framework align with the principles of open science, how does it support collaboration, and how can its features contribute to the development of more transparent and reproducible AI and ML research?
% }
% The Yerbamate framework aligns with the principles of open science in several ways. It promotes transparency and reproducibility by providing a set of consistent tools and practices that allow for the creation, sharing, and experimentation of machine learning and artificial intelligence models. The framework's modular design allows for easy customization and extension, making it simple for researchers to build upon existing work and collaborate with others. Additionally, the framework's collaboration features, such as out-of-the-box sharing of models via GitHub URLs, facilitate easy sharing and reuse of code, enabling researchers to build on each other's work and contribute to a more open and collaborative scientific community. By combining these features, the Yerbamate framework provides a powerful tool for advancing open science practices in the field of AI and ML research.
% \subsection{
%     How can we establish a set of conventions for software engineering in AI and ML that are widely accepted and easy to adopt?}
%     Establishing widely accepted and easy to adopt software engineering conventions for AI and ML is a complex task that requires input and collaboration from multiple stakeholders, including researchers, practitioners, and software engineers. One approach is to focus on the principles of modularity, separation of concerns, and documentation, which can help to promote flexibility, reusability, and maintainability of code. The use of independent modules that only depend on python dependencies and the code inside the module uses relative imports is a simple and effective way to promote modularity. This convention can be supplemented with best practices for documentation, including standardized naming conventions and clear documentation of code functionality, assumptions, and limitations.
    
    % \subsection{
    % How can modularity be utilized to promote open science practices in the field of artificial intelligence and machine learning?
    % }
    
    % \subsection{
    % Can conventions in AI and ML improve the learning experience for developers and enhance the efficiency and effectiveness of AI development?
    % }
    % Establishing conventions for AI and ML development can lead to an improved developer experience, resulting in greater efficiency and effectiveness in the development process. Conventions that prioritize modularity, separation of concerns, and consistent naming conventions can help to make the code more maintainable and reusable. This can save developers time in the long run, as they can easily incorporate existing code and components into new projects without having to start from scratch. Additionally, established conventions can make it easier for developers to collaborate and share code with others, ultimately contributing to a more open and collaborative scientific community. Overall, the adoption of conventions in AI and ML can lead to a more efficient and effective development process, ultimately enhancing the developer experience.
    
    % \subsection{
    % How can the adoption of software engineering conventions in AI and ML improve the developer experience and enhance the efficiency and effectiveness of AI development?
    % }
    % The establishment of conventions for software engineering in AI and ML can improve the developer experience by providing a standardized approach to project organization, code structure, and modularity. Such conventions can facilitate collaboration and enable code sharing among researchers, practitioners, and the wider scientific community. 
    % In the case of the Yerbamate framework, the use of independent modules and a non-independent experimentation module provides a clear separation of concerns and enables ease of use for experimentation, training, and validation of models. The Yerbamate framework provides a set of simple and consistent commands for installing, experimenting, and running code, which can lead to an improved developer experience for AI. Additionally, the framework includes an out-of-the-box sharing feature that allows users to share their models via GitHub URLs. By making it easy for users to share and reuse existing code and models, the framework promotes collaboration and open science practices in AI development.
    
%     \subsection{
%   What are the benefits and challenges of adopting a software engineering convention for AI and ML research, and how can we ensure that the benefits outweigh the costs while promoting open science and fostering a culture of collaboration and sharing within the research community?
%     }
%     Adopting a software engineering convention for AI and ML research can lead to significant benefits, including improved code quality, better maintainability, and enhanced reproducibility. However, there can be a learning curve associated with adopting a new convention, particularly for researchers who may not have extensive software engineering backgrounds. The challenge is to develop a convention that is easy to learn and widely accepted. Modularity can be an effective approach, as it allows for the gradual refactoring of existing code into independent, reusable modules, which can lead to an improved developer experience over time. To ensure that the benefits of adopting a convention outweigh the costs, it is essential to provide clear documentation, training materials, and support resources to facilitate the learning process. Additionally, it is important to foster a culture of collaboration and sharing within the research community, which can help to promote the widespread adoption of best practices and ensure that the benefits of improved software engineering practices are shared across the field.



% \subsection{How can a software engineering convention for AI/ML be designed to ensure compatibility and customizability, and what are the benefits of such a convention for open science and collaborative research?}
% The convention should prioritize modularity, separation of concerns, and consistent naming conventions for independent and interchangeable modules such as models, data, and trainers. The convention should also be flexible enough to accommodate a wide range of use cases and project types. Clear documentation, training materials, and support resources can help users understand and adopt the convention, and open-source technologies and practices can facilitate collaboration and sharing of code and models within the scientific community.

% \subsection{How can modular software engineering conventions promote the reusability and scalability of AI and ML systems?}

% \subsection{Can a standard format be designed for data loaders and preprocessing pipelines?}
% {
% The development of a universal standard format for data loaders and preprocessing pipelines in deep learning is challenging due to the framework-specific nature of the task. Preprocessing pipeline design is highly dependent on the specific problem and its data, making it difficult to create a general solution. However, Separation of concerns (SoC) and modularity principles can enhance code reusability by creating the data loading as an independent python module, and separating preprocessing pipeline from data loading. This project focused on making data loading and code components, such as data loaders and the preprocessing pipelines, sharable within a python machine learning framework.
% }

% \subsection{Can this be put in a high-quality, easily accessible database for local access?} {
% The creation of a high-quality, easily accessible database storing preprocessed datasets and related components, such as data loaders and preprocessing pipelines, is complex and involves challenges in terms of data processing, storage, and retrieval, which were beyond the scope of the project. Limitations such as credibility, scalability, compatibility with different deep learning frameworks, privacy, and storage pose additional challenges. This project focused on designing a standard for sharing code-related components, instead of making data available, for feasibility of the task.
% }

% \subsection{Can the database be made easily expandable?} {
% Expanding the database of deep learning code components, such as data loaders, relies on adopting software engineering principles of separation of concerns (SoC) and modularity. Adhering to these principles makes code components shareable across projects within the same framework, which can get added to a database and scale in time.
% }

\section{Methodology}

The widespread adoption of the Python programming language in AI programming has been noted in recent studies (\cite{mihajlovic2020use,raschka2020machine}). As such, this project focuses on the examination of software engineering practices in the context of Python and its machine learning frameworks. Many interdisciplinary researchers may lack the software engineering expertise necessary to manage their code bases to ensure correctness, understandability, extendability, and reusability (\cite{amershi2019software,scully-debt-ml,leakage-recrisis,accountabilityInAi}). To address these challenges, we investigated the application of software engineering principles of separation of concerns (SoC) and modular design in Python to AI projects. The goal of this study was to improve the reproducibility and reusability of python AI components through the adoption of best software engineering practices.

% \subsection{Software Engineering for Python AI Projects}
%  In the context of AI, the implementation of modularity and SoC principles results in the separation of models, data-related components, and trainers. 
% This approach facilitates the creation of a modular design, where the code is organized into distinct, reusable components (\cite{sanner1999python,pressman2010software}). This decoupling has the potential to greatly enhance the reusability of code greatly, enabling easier experimentation with models and datasets.


\subsection{A Different Approach to Deep Learning Framework}

An alternative approach to the compatibility and re-usability issue in deep learning frameworks is to enforce principles of separation of concerns (SoC) and modularity in Python. 
The implementation of modularity and SoC principles results in the separation of models, data-related components, and trainers. 
This approach facilitates the creation of a modular design, where the code is organized into distinct, reusable components (\cite{sanner1999python,pressman2010software}). This separation enhances the reusability of the code, making it easier to experiment with different models and datasets within the same framework. This approach has the advantage of limiting the learning overhead to the principles of software engineering, rather than requiring a transition to a unified deep learning framework.



A limitation of this approach is that it requires manual translation of AI components, such as data loaders and models, into different frameworks when necessary. While this approach may require additional effort, it allows for greater flexibility and customizability in implementing deep learning models. Additionally, it promotes a deeper understanding of the underlying software engineering principles and their application in the field of AI.



\section{Results}

This section presents the results of the study, which provide software engineering principles for shareable modular Python projects and an open science framework, Yerbamaté. The principles include modular project structuring, independent and interchangeable modules, and an experiment definition format.  Furthermore, the Yerbamaté CLI is introduced as a productivity tool for modular Python projects, promoting sharing and reuse of Python modules.


\subsection{Independent/Reusable Python Modules}

Independent/Reusable Python modules are self-contained modules that only depend on Python dependencies, such as NumPy, PyTorch, TensorFlow, or Hugging Face. The code inside the module uses relative imports to import within the module, making it independent and reusable once the necessary Python dependencies are installed locally. This approach enhances modularity, reusability, and shareability while promoting good software engineering practices.



\subsection{Separated Concern Python Modular Project Structure}


 The modular structure of the framework and software engineering best practice of SoC involves organizing the project directory in a hierarchical tree structure, with an arbitrary name given to the root project directory by the user. The project is then broken down into distinct concerns such as models, data, trainers, experiments, analyzers, simulators, each with its own subdirectory. Within each concern, arbitrary modules can be defined, including models, trainers, data loaders, data augmentations, and loss functions. Independent modules can be defined in various heights in the tree.

The framework prioritizes the organization of the project into independent modules when applicable, however there are situations where a combination of independent modules may be necessary for a particular concern. An example of this is the experiment concern, which imports and combines models, data, and trainers to define and create a specific experiment. In such cases, the module is not independent and is designed to combine the previously defined independent modules. 


Yerbamaté addresses the issue of specifying the exact framework used for each module in a project that involves multiple frameworks by providing a naming convention that specifies the framework used before the name of the module. This naming convention helps maintain consistency and make it easy to identify which framework was used for a specific module. For instance, the subdirectory for a Jax-based model named "my\_model" would be "models/jax/my\_model", while the subdirectory for a PyTorch-based model with the same name would be "models/torch/my\_model".




\begin{figure}
\centering
\framebox[\0.5\textwidth]{%
\begin{minipage}{0.45\textwidth}
\dirtree{%
.1 project.
.2 data.
.3 torch.
.4 imagenet.
.4 \texttt{bit}.
.2 models.
.3 jax.
.4 transformer.
.4 gpt2.
.3 torch.
.4 \texttt{big\_transfer}.
.4 transformer.
.2 experiments.
.3 torch.
.4 bit.
.3 jax.
.4 gpt2.
.2 trainers.
.3 jax.
.4 gpt2.
.3 torch.
.4 \textt{bit}.
.2 ....
}
\end{minipage}}
\caption{
This figure illustrates a directory tree example of a deep learning modular project. The organization of files into separated concerns in the tree is a common best practice for developing modular Python projects. By adhering to principle of independency, the submodules, such as Transformer and GPT2 can be easily shared and reused across different projects. 
}
\end{figure}



\subsection{No-Loop Python Experiment Definition}

This section presents a convention and guideline used in the Yerbamate framework for defining Python experiments that prohibits the use of loops. This approach aims to promote a hyperparameter-focused configuration and maintain separation of concerns, resulting in a more flexible and customizable format that can be adapted to various AI tasks, frameworks, and libraries. In practice, the configuration file should primarily consist of hyperparameters, object creations, and executions, while higher complexities such as training loops should generally be avoided.






% \subsection{Modular Project Structure for AI}

% The structure for AI projects results in a modular pattern that is organized into distinct  modules, namely "models", "experiments", "trainers", and "data", which have been identified as major concerns in AI projects across a variety of tasks. This modular approach creates standalone and independent modules for models, trainers, and data, which can operate independently to enhance modularity and reusability. The "experiments" module harmonizes the three independent modules and provides a unified experiment. This naming convention is used to enhance the understandability of the project and facilitate the adoption of software engineering best practices, such as modularity and separation of concerns.


\subsection{Yerbamaté: A Python Framework for Open Science}

Yerbamaté is an open science framework that prioritizes modularity, independent modules, and a separated concern project structure for Python projects. These principles enable Yerbamaté to provide a command line interface (CLI) for easy sharing and installation of code modules, dependency management, and execution of experiments. The framework supports any kind of experimentation with any python framework, with key features such as:

\subsubsection{Easy Installation with Yerbamaté Install}

Yerbamaté's install command enables the installation of modularized projects adhering to separation of concerns principles, along with their associated Python dependencies, with a single command. Additionally, Yerbamaté allows the installation of non-Yerbamaté projects as concrete modules, which can be installed alongside the root module of the project. While Yerbamaté cannot automatically install Python dependencies without a "requirements.txt" file in the root directory of the module, it can for example be used to install source code of over 100 Torch image models or 30 PyTorch vision in transformers directly into a project. See Appendix \ref{CLIApp}.

% \subsubsection{Metadata and Reproducibility}
% Yerbamate can be used to create and inject python version dependencies for reproduciblity by creating a requirements.txt file. Moreover metadata can be generated to create list of reusable modules that are in the project. THese metadata compily with FAIR pricniples

\subsubsection{Metadata and Reproducibility}

Yerbamaté enhances reproducibility by creating dependencies of a python module with a "requirements.txt" file that lists the python version dependencies. The framework can also be used to generate metadata for exported and reusable modules that complies with the FAIR principles. By adhering to these principles and guidelines, Yerbamaté promotes transparency, reproducibility, and open science practices in the research community.

\subsubsection{Environment API}

The Yerbamaté Environment API is a tool designed to manage environment variables within an experiment. It prioritizes the use of an \texttt{env.json} file for storing environment variables, falling back to the operating system's environment variables if the file is not found. The API offers a convenient way to set, retrieve, and manage these variables in a centralized and organized manner, which can be useful in storing and accessing environment-specific information, such as local results and data paths, API keys, database URLs, and other sensitive data.


% \subsection{Yerbamaté: A Python Framework for Open Science}

% Yerbamaté is an open science framework that prioritizes modularity, independent modules, and a separated concern project structure for Python projects. These principles enable Yerbamaté to provide a command line interface (CLI) for easy sharing and installation of code modules, dependency management, and execution of experiments. The framework also supports machine learning model development, with key features such as:

% \subsubsection{Easy Installation with Yerbamaté Install}

% Yerbamaté's install command enables the installation of modularized projects adhering to separation of concerns principles, along with their associated Python dependencies, with a single command. Additionally, Yerbamaté allows the installation of non-Yerbamaté projects as concrete modules, which can be installed alongside the root module of the project. While Yerbamaté cannot automatically install Python dependencies without a "requirements.txt" file in the root directory of the module, it allows the manual installation of over 100 Torch image models and 30 PyTorch vision implementations in transformers, for example.

% \subsubsection{Metadata and Reproducibility}

% \subsection{Yerbamaté: An Open Science Python Framework}




% Yerbamaté is an open science framework built around the aforementioned principles of modularity, independent modules, and a separated concern project structure for Python projects. When these principles and guidelines are adhered to, the resulting framework provides a command line interface (CLI) for sharing and installing code modules, dependency management, and execution of experiments. 



% Yerbamaté provides utility functions and supports modular projects, facilitating machine learning model development. Key Yerbamate features include:


% \subsubsection{Yerbamaté install}
% Yerbamaté's install command enables easy installation of modularized projects adhering to separation of concerns principles, including both source code and their associated Python dependencies, with a single command. In addition, Yerbamaté allows for the installation of non-Yerbamaté projects as concrete modules, which can be installed alongside the root module of the project. However, note that Yerbamaté cannot install Python dependencies automatically in the absence of a "requirements.txt" file in the root directory of the module.

% For example, one could install the source code of over 100 Torch image models\footnote{\url{https://github.com/rwightman/pytorch-image-models/tree/main/timm/}} and over 30 PyTorch vision implementations in transformers\footnote{\url{https://github.com/lucidrains/vit-pytorch/tree/main/vit_pytorch/}} directly into your project and then manually install the dependencies with pip.


% \subsubsection{Yerbamaté Environment API}

% The Yerbamaté Environment API is a tool designed to manage environment variables within a experiment. It prioritizes the use of an \texttt{env.json} file for storing environment variables, but if it is not found, it falls back to the operating system's environment variables. The API offers a convenient way to set, retrieve, and manage these variables in a centralized and organized manner. This can be particularly useful in storing and accessing environment-specific information, such as local results and data paths, API keys, database URLs, and other sensitive data.


\section{Discussion}

Yerbamaté is an open-source, open-science Python framework designed to promote software engineering principle of modularity, reproducibility, and openness in AI research. The framework offers a promising approach to addressing the reproducibility crisis in AI, by standardizing the development and management of AI projects and facilitating the sharing of research findings. Yerbamaté's software engineering conventions and accompanying toolkit provide researchers with the tools to create and inject dependencies, leading to the reuse of code and components, and ultimately increasing reproducibility and the sharing of research findings.

% The modular design principles employed in Yerbamaté can enhance transparency, reproducibility, and the sharing of knowledge and research findings. However, it is worth noting that the adoption of these principles may require additional time and effort for researchers who are not familiar with software engineering principles. Additionally, compatibility with pure Python may pose some challenges to researchers who prefer other programming languages.

The flexibility and customization offered by Yerbamaté facilitate the adaptation of the framework to meet researchers' diverse needs, contributing to the development of more accessible and efficient AI research environments. Researchers with varying technical backgrounds can share coding under software engineering guidelines, fostering collaboration and interdisciplinary learning.

The impact of Yerbamaté on the reproducibility and trustworthiness of AI research is a crucial area for future research. The usability of the framework and its potential limitations should also be further investigated. Future studies can also explore the impact of Yerbamaté on facilitating interdisciplinary collaborations and knowledge sharing in the field of AI.

%  Another area for future work could be the creation of naming conventions for modules and functions, which could further enhance the modularity and readability of the code. Additionally, comprehensive evaluation and testing of the framework could help to identify any potential issues or areas for improvement, ensuring the continued effectiveness and usefulness of the Yerbamaté framework.
 

\section{Social Impact}
Machine learning models and artificial intelligence systems have the potential to impact society, both positively and negatively significantly (\cite{mittelstadt2019principles, jobin2019global,arrieta2020explainable, floridi2018ai4people}). Consequently, AI's ethical and societal implications have been the subject of much discussion and research in recent years (\cite{floridi2018ai4people, goodman2017european, floridi2019establishing, mittelstadt2016ethics}).
Open science can significantly impact society by promoting collaboration, transparency, and accessibility in research, enabling broader participation in the scientific process and sharing of knowledge and tools, thus accelerating safer progress and (\cite{kocak2022transparency, wachter2017transparent, coro2020open, braun2018open, paton2019open, goodman2017european}). 

In artificial intelligence, open science can help democratize access to machine learning and facilitate the development of ethical and accountable AI systems (\cite{goodman2017european, batarseh2020data}). By promoting the principles of open science, researchers and practitioners can work together to build more robust, trustworthy, and fair AI solutions (\cite{accountabilityInAi, kocak2022transparency,wachter2017transparent, coro2020open, braun2018open, hicks2021open, goodman2017european}.

The development of open source, open science, accessible and user-friendly tools for training machine learning models has the potential to democratize access to artificial intelligence and facilitate more involvement in research and innovation. The Yerbamaté toolkit contributes in this regard, providing a simple and effective means for researchers and practitioners to share, develop and evaluate machine learning models. Moreover, it makes training AI accessible to a broader audience as anyone can run an experiment on accessible science tools such as Colab\footnote{\url{https://colab.research.google.com/github/oalee/yerbamate/blob/main/deep_learning.ipynb}} and reproduce a scientific experiment and conduct their own experiments.


The wide accessibility of AI through Yerbamaté and other similar open science tools have the potential to accelerate research in various fields (\cite{olson2018system, wolf2020designing,morris2020ai,ong2021guide, li2018can}) including healthcare (\cite{haristiani2020combining}), law (\cite{ashley2017legal}), and education (\cite{goel2020ai}). For example, machine learning models can improve patient outcomes (\cite{hamet2017medicine}), detect fraud in financial transactions (\cite{bao2022fraudartificial}, and enhance personalized learning in education (\cite{haristiani2020combining, goel2020ai}). At the same time, it is essential to ensure that the development and application of AI adhere to ethical principles, including inclusion, transparency, accountability, and fairness (\cite{ accountabilityInAi,  jobin2019global, floridi2018ai4people, wachter2017transparent, arrieta2020explainable, mittelstadt2019principles, goodman2017european, mittelstadt2016ethics, floridi2019establishing, o2017weapons}).




\section{Acknowledgments}

The author acknowledges the open source and open science community, tools, libraries, and frameworks, as well as the various open source and open science AI projects that made the development of the Yerbamaté framework possible. Without these contributions, this work would not be possible.

The development of the Maté\footnote{\url{https://github.com/ilex-paraguariensis/yerbamate}} framework, an open source tool for AI experimentation, began in June 2022
 and was further developed and tested various designs in the following months. Due to laws at Maastricht University that prohibits collaborations in research internships, the project was forked into Yerbamaté as an individual effort to enhance flexibly and open science.

The author also expresses gratitude to their supervisors, Iris Groen of the University of Amsterdam and Mirela Popa of Maastricht University, for their guidance and support. 
The author acknowledges Giulio Zani's contributions to the open source project, Maté. and many sample projects in the appendix, which are forked from Giulio Zani's repositories. The author's contribution to the development of the Yerbamaté framework is presented as an alt-metric of commits history, and the progress of the work can be tracked using GitHub's analyzing contributions tool\footnote{\url{https://github.com/oalee/yerbamate/graphs/contributors}}.
% The contributions of Giulio Zani to the open source project called Maté are also acknowledged. The author would also like to express gratitude to the supervisors, Iris Groen of University of Amsterdam and Mirela Popa of Maastricht University, for their guidance and support. Finally, the author acknowledges the many sample projects in the appendix, which are forked from Giulio Zani's repositories.
% The author's contribution to the implementation of the Yerbamaté framework is presented as an alt-metric of commits, and the progress of the work is accessible from GitHub's\footnote{\url{https://github.com/oalee/yerbamate/graphs/contributors}} analyzing contributions tool.
% Moreover, the contribution of code to this project is presented as an altmetric in Figure X.

% \section{Acknowledgments}

% The author acknowledges the open source community, as this work builds on open source projects, libraries, and frameworks, without which this work would not have been possible. The Yerbamaté framework was started as an open source project and further developed during a research internship at UVA for a master's thesis in AI. Due to prohibitive laws at Maastricht University that restrict collaborations in research internships, the project was forked as an individual effort. The contributions of Giulio Zani to the open source project called Maté are also acknowledged. The Yerbamaté framework was designed to be flexible and extensible, with the goal of promoting open science and improving the developer experience. The development of the framework was made possible by the guidance of the supervisors, Iris Goren of UVA and Mirela Popa of Maastricht University. Additionally, the author would like to acknowledge the support of the open source community, which provided valuable feedback, contributions, and insights, without which this project would not have been possible.

% The Yerbamate framework acknowledges the open source community and the principles of open science. This work started as an open source project and was further developed during a research internship at UVA for a master's thesis in AI. Due to prohibitive laws at Maastricht University that restrict collaborations in research internships, the project was forked as an individual effort. The contributions of Giulio Zani to the open source project called Maté are also acknowledged. The Yerbamate framework was designed to be flexible and extensible, with the goal of promoting open science and improving the developer experience. The development of the framework was made possible by the guidance of the supervisors, Iris Goren of UVA and Mirela Popa of Maastricht University.

\section{Conclusion}



    % This research internship aimed to address the reproducibility crisis in artificial intelligence (AI) research by investigating software engineering best practices, open science and accessible AI. To achieve this goal, Yerbamaté framework was developed around the software engineering principles of modularity and separation of concerns. Yerbamaté is an open science modular Python framework designed to streamline and simplify the development and management of machine learning projects. The framework encourages quality coding, collaboration, and the sharing of models, trainers, data loaders, and knowledge, while also promoting reproducibility, customization, and flexibility. The modular design and separation of concerns simplify the development and maintenance of machine learning models, leading to an improved developer experience. The straightforward installation, sharing, and training process makes it accessible to researchers and practitioners with varying technical expertise, enhancing collaboration and knowledge sharing. The findings suggest that the adoption of modular design principles and open science tools can contribute significantly to addressing the reproducibility crisis in AI, leading to more accessible, transparent, and trustworthy AI.
    

The Yerbamaté framework could provide a valuable contribution to the field of artificial intelligence by promoting open science and accessible AI. The software engineering principles and modular design approach utilized by Yerbamaté simplify the development and maintenance of machine learning models, enhancing developer experience, and enabling collaboration and knowledge sharing among researchers and practitioners with varying technical expertise. The flexibility and customization offered by Yerbamaté make it accessible to researchers and practitioners with varying technical backgrounds and enhance the adaptability of the framework to meet the diverse needs of AI projects. The potential for Yerbamaté to address the reproducibility crisis in AI, along with its potential to facilitate collaborations and sharing, makes it a promising tool for future AI research. The promotion of open science frameworks has the potential to revolutionize the development and implementation of trustworthy and transparent AI, fostering innovation and progress across a variety of fields.
% In conclusion, the development of the Yerbamaté framework could offer a significant contribution to the field of artificial intelligence by promoting open science and accessible AI. By providing a modular framework that encourages standardized software engineering practices, Yerbamaté simplifies the development and maintenance of machine learning models, enhancing collaboration and knowledge sharing among researchers and practitioners with varying technical expertise. The framework's ease of use and flexibility enables wider accessibility to artificial intelligence, making it possible for individuals and organizations to develop and train machine learning models with greater ease and efficiency. Overall, the adoption of modular design principles and open science tools has the potential to revolutionize the development and implementation of trustworthy and transparent AI, facilitating innovation and progress in various fields. 
% Nevertheless, the widespread adoption of the Yerbamaté toolkit and other similar tools may pose challenges, including a potential learning curve, implementation overhead, and a need for further refinement and evaluation.




\clearpage
\newpage


\printbibliography




\onecolumn

\section{Appendix}







\subsection{Yerbamaté CLI Examples}

\begin{itemize}
    \item \textbf{Installing GAN experiment from a modular project}: 
    
    \texttt{mate install oalee/lightweight-gan/lgan/experiments/lgan -yo pip}
    \item \textbf{Training the GAN experiment}: 
    
    \texttt{mate train lgan cars}
    
    \item \textbf{Installing transfer learning experiment}

    \texttt{mate install oalee/big\_transfer/experiments/bit}
    
    \item \textbf{Installing code from non modular project code}

    \texttt{mate install https://github.com/rwightman/pytorch-image-models/tree/main/timm/}

    \item \textbf{Installing pytorch ViT implementation source code}
    
    \texttt{mate install https://github.com/lucidrains/vit-pytorch/tree/main/vit\_pytorch/}

\end{itemize}


\subsection{Examples}

\subsubsection{Example Custom Data Preprocessing}
The modular structure of the Yerbamaté toolkit, coupled with its compatibility with pure Python, allows for the integration of custom data preprocessing pipelines with ease. By utilizing the Yerbamaté environment API, developers and researchers can readily access the data paths and results path for the destination of their processed data. For instance, the following project structure can utilize the command \texttt{python -m deepnet.data.my\_data.preprocessing} to execute a custom preprocessing pipeline. The flexibility offered by the python modularity enables users to efficiently tailor their preprocessing procedures to the specific requirements of their research or application, and the Yerbamaté toolkit can be used to share these pipelines effortlessly.

% \captionsetup[figure]{position=bottom,justification=centering,width=.4\textwidth,labelfont=bf,font=small}

\begin{figure}[h!]
\centering
\framebox[\0.4\textwidth]{%
\begin{minipage}{0.4\textwidth}
\dirtree{%
.1 deepnet.
.2 data.
.3 \texttt{\_\_init\_\_.py}.
.3 my\_data.
.4 \texttt{\_\_init\_\_.py}.
.4 preprocessing.
.5 \texttt{\_\_init\_\_.py}.
.5 preprocess.py.
.4 data\_loader.
.2 models.
.2 trainers.
.2 experiments.
}
\end{minipage}
}
\caption{
Custom data preprocessing modular structure example
}
\label{customdata}
\end{figure}


\subsubsection{Example GAN Experiment}

The following example illustrates the experiment definition of a Lightweight Generative Adversarial Networks (\cite{lgan,goodfellow2020generative}) implemented with Pytorch Lightning. The use of Python enables the customization of model hyperparameters, loggers, model savers, learning rate schedulers, and optimization algorithms through argument specqification in functions or classes. The source code for the complete project is accessible on Github\footnote{\url{https://github.com/oalee/lightweight-gan}} and all its modules can be installed and the experiment can be trained using Yerbamaté command line on Colab or local machines. The experiment integrates and imports independent trainers, models, and data modules, and defines the experiment's hyperparameters.

% [
% frame=lines,
% framesep=2mm,
% baselinestretch=1.2,
% % bgcolor=LightGray,
% fontsize=\footnotesize,
% linenos
% ]


\begin{minted}{python}
from ...data.cars import CarsLightningDataModule, AugWrapper
from ...trainers.lgan import LightningGanModule
from ...models.lgan import Generator, Discriminator
from torch import nn
import yerbamate, torch, pytorch_lightning as pl, pytorch_lightning.callbacks as pl_callbacks, os
# Managing environment variables
env = yerbamate.Environment()

data_module = CarsLightningDataModule(
    image_size=128,
    aug_prob=0.5,
    in_channels=3,
    data_dir=env["data_dir"],
    batch_size=8,
)

generator = Generator(
    image_size=128,
    latent_dim=128,
    fmap_max=256,
    fmap_inverse_coef=12,
    transparent=False,
    greyscale=False,
    attn_res_layers=[],
    freq_chan_attn=False,
    norm_class=nn.BatchNorm2d,
)

discriminator = Discriminator(
    image_size=128,
    fmap_max=256,
    fmap_inverse_coef=12,
    transparent=False,
    greyscale=False,
    disc_output_size=5,
    attn_res_layers=[],  # Try [16, 32, 64, 128, 256] if your hardware allows
)

g_optimizer = torch.optim.Adam(generator.parameters(), lr=0.0002, betas=(0.5, 0.999))
d_optimizer = torch.optim.Adam(
    discriminator.parameters(), lr=0.0002, betas=(0.5, 0.999)
)

model = LightningGanModule(
    save_dir=env["results"],
    sample_interval=100,
    generator=generator,
    discriminator=AugWrapper(discriminator),
    optimizer=[
        {
            "optimizer": g_optimizer,
            "lr_scheduler": {
                "scheduler": torch.optim.lr_scheduler.StepLR(
                    g_optimizer, step_size=100, gamma=0.5
                ),
                "monitor": "fid",
            },
        },
        {
            "optimizer": d_optimizer,
            "lr_scheduler": {
                "scheduler": torch.optim.lr_scheduler.ReduceLROnPlateau(
                    d_optimizer, mode="min", factor=0.5, patience=5, verbose=True
                ),
                "monitor": "fid",
            },
        },
    ],
    aug_types=["translation", "cutout", "color", "offset"],
    aug_prob=0.5,
)

logger = pl.loggers.TensorBoardLogger(env["results"], name=env.name)
callbacks = [
    pl_callbacks.ModelCheckpoint(
        monitor="fid",
        dirpath=env["results"],
        save_top_k=1,
        mode="min",
        save_last=True,
    ),
    pl_callbacks.LearningRateMonitor(logging_interval="step"),
]
trainer = pl.Trainer(
    logger=logger,
    accelerator="gpu",
    precision=16,
    gradient_clip_val=0.5,
    callbacks=callbacks,
    max_epochs=100,
)

if env.train:
    trainer.fit(model, data_module)
if env.test:
    trainer.test(model, data_module)
if env.restart:
    trainer.fit(model, data_module, ckpt_path=os.path.join(env["results"], "last.ckpt"))

\end{minted}





\section{Transfer Learning Case Study}\label{transfer-study}


In this section, we showcase the application of modularity and separation of concerns on the official implementation of "Big Transfer (BiT): General Visual Representation Learning"\cite{transferlearning}. The source code from the official repository \footnote{\url{https://github.com/google-research/big_transfer}} has been refactored \footnote{\url{https://github.com/oalee/big_transfer}} into a modular, decoupled structure.  The original folder structure of the repository is as follows:


\begin{figure}
\centering
\framebox[\0.4\textwidth]{%
\begin{minipage}{0.4\textwidth}
\dirtree{%
.1 /.
.2 \texttt{bit\_common.py}.
.2 \texttt{bit\_hyperrule.py}.
.2 \texttt{bit\_pytorch}.
.3 \texttt{fewshot.py}.
.3 \texttt{\_\_init\_\_.py}.
.3 \texttt{lbtoolbox.py}.
.3 \texttt{models.py}.
.3 \texttt{requirements.txt}.
.3 \texttt{train.py}.
.2 \texttt{\_\_init\_\_.py}.
}
\end{minipage}
}
\caption{
Official Repository of BiT Project Structure for Pytorch
}
\end{figure}



\vspace{0.4em}
In this examination, we delve into the components of the official Big Transfer repository to better understand the purpose of each one. The components are as follows:

\begin{itemize}
    \item \texttt{bit\_common.py} serves as the central point for defining an argument parser and setting up a logger for experiments. Currently, the project only supports a limited set of hyperparameter selections, which include the initial learning rate, batch size, batch split, and five datasets, namely CIFAR10, CIFAR100, Oxford\_iiit\_pet, Oxford\_flowers102, and ImageNet2012. The name of this file, however, does not accurately reflect its purpose.
    \item \texttt{bit\_hyperrule.py} is responsible for defining the learning rate scheduler and a utility function for computing the resolution of the model based on the dataset. The name of this file, once again, does not accurately reflect its purpose and separates the concern of data-related functions from the learning rate scheduling.
    \item \texttt{few\_shot.py} is specifically designed to find few-shot learning samples for the model, and its name accurately reflects its purpose.
    \item \texttt{lbtoolbox.py} handles interruptions in training and provides a chronometer interface for profiling. This component is independent and does not couple with any other part of the system.
    \item \texttt{models.py} defines the models and is also an independent component that does not couple with any other part.
    \item \texttt{train.py} is utilized for training the model. This component includes the implementation for batch splitting and can only be executed with the pre-defined hyperparameter selection.
\end{itemize}


The following code illustrates the execution of the training procedure in the original repository. The training process is initiated by the \texttt{main} function located at the end of the \texttt{train.py} file. 

\begin{minted}{python}
if __name__ == "__main__":
  parser = bit_common.argparser(models.KNOWN_MODELS.keys())
  parser.add_argument("--datadir", required=True,
                      help="Path to the ImageNet data folder, preprocessed for torchvision.")
  parser.add_argument("--workers", type=int, default=8,
                      help="Number of background threads used to load data.")
  parser.add_argument("--no-save", dest="save", action="store_false")
  main(parser.parse_args())
\end{minted}

This implementation exhibits limited adaptability as the function is dependent solely on the arguments provided through the command-line interface. The following tree structure and experiment definition showcases modularity and separation of concerns applied on this task .




\begin{minted}{python}
from ...trainers.bit_torch.trainer import test, train
from ...models.bit_torch.models import load_trained_model, get_model_list
from ...data.bit import get_transforms, mini_batch_fewshot
import torchvision as tv, yerbamate, os, tensorboard
from torch.utils.tensorboard import SummaryWriter


# BigTransfer Medium ResNet50 Width 1
model_name = "BiT-M-R50x1"
# Choose a model form get_model_list that can fit in to your memoery
# Try "BiT-S-R50x1" if this doesn't works for you

env = yerbamate.Environment()

train_transform, val_transform = get_transforms(img_size=[32, 32])
data_set = tv.datasets.CIFAR10(
    env["datadir"], train=True, download=True, transform=train_transform
)
val_set = tv.datasets.CIFAR10(env["datadir"], train=False, transform=val_transform)

train_set, val_set, train_loader, val_loader = mini_batch_fewshot(
    train_set=data_set,
    valid_set=val_set,
    examples_per_class=None,  # Fewshot disabled
    batch=128,
    batch_split=2,
    workers=os.cpu_count(),  # Auto-val to cpu count
)

imagenet_weight_path = os.path.join(env["weights_path"], f"{model_name}.npz")
model = load_trained_model(
    weight_path=imagenet_weight_path, model_name=model_name, num_classes=10
)
logger = SummaryWriter(log_dir=env["results"], comment=env.name)

if env.train:
    train(
        model=model,
        train_loader=train_loader,
        valid_loader=val_loader,
        train_set_size=len(train_set),
        save=True,
        save_path=os.path.join(env["results"], f"trained_{model_name}.pt"),
        batch_split=2,
        base_lr=0.001,
        eval_every=100,
        log_path=os.path.join(env["results"], "log.txt"),
        tensorboardlogger=logger,
    )

if env.test:
    test(
        model=model,
        val_loader=val_loader,
        save_path=os.path.join(env["results"], f"trained_{model_name}.pt"),
        log_path=os.path.join(env["results"], "log.txt"),
        tensorboardlogger=logger,
    )

\end{minted}



\begin{figure}[h]
\centering
\framebox[\0.4\textwidth]{%
\begin{minipage}{0.4\textwidth}
\dirtree{%
.1 /\texttt{big\_transfer}.
.2 data.
.3 bit.
.4 \texttt{fewshot.py}.
.4 \texttt{\_\_init\_\_.py}.
.4 \texttt{minibatch\_fewshot.py}.
.4 \texttt{requirements.txt}.
.4 \texttt{transforms.py}.
.3 \texttt{\_\_init\_\_.py}.
.2 experiments.
.3 bit.
.4 \texttt{dependencies.json}.
.4 \texttt{few\_shot.py}.
.4 \texttt{\_\_init\_\_.py}.
.4 \texttt{learn.py}.
.4 \texttt{requirements.txt}.
.3 \texttt{\_\_init\_\_.py}.
.2 \texttt{\_\_init\_\_.py}.
.2 models.
.3 \texttt{bit\_torch}.
.4 downloader.
.5 \texttt{downloader.py}.
.5 \texttt{\_\_init\_\_.py}.
.5 \texttt{requirements.txt}.
.5 \texttt{utils.py}.
.4 \texttt{\_\_init\_\_.py}.
.4 \texttt{models.py}.
.4 \texttt{requirements.txt}.
.3 \texttt{\_\_init\_\_.py}.
.2 trainers.
.3 \textt{bit\_torch}.
.4 \texttt{\_\_init\_\_.py}.
.4 \texttt{utils.py}.
.4 \texttt{logger.py}.
.4 \texttt{lr\_schduler.py}.
.4 \texttt{requirements.txt}.
.4 \texttt{trainer.py}.
.3 \texttt{\_\_init\_\_.py}.
}
\end{minipage}
}\caption{Refactored Repository of BiT Project Structure}
\end{figure}


The new structure in the refactored repository is designed to address the limitations of the original implementation. By separating concerns and adopting modularization, the refactored repository provides a more flexible and scalable solution for training models. The different components in the new structure, such as the trainer, model, and data loading modules, are designed to be independent and reusable, making it easier to manage and maintain the codebase. Moreover, the experimentation module provides a unified interface for combining and executing the various components, making it easier to experiment with different configurations and hyperparameters. Overall, the new structure in the refactored repository represents a significant improvement over the original implementation, offering a better and more organized approach to training machine learning models.





\end{document}