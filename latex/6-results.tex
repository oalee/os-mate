
\section{Results}




% \subsection{Python Modularity Structure}
% The Structure prioritizes modularity, separation of concerns, and consistent naming Structures to promote maintainability and reusability of code. It places a high priority on creating independent modules that are standalone and reusable components, which can be used across different projects. 

% This Structure for organizing Python research projects is designed to promote modularity and separation of concerns, making research more maintainable and reusable. 

\subsection{Python Modular Project Structure}
% The Structure involves organizing the project directory in a hierarchical tree structure, with an arbitrary name given to the root project directory. The project is then broken down into separate concern modules, such as models, experiments, trainers, data, analyzers, and simulators, each with its own subdirectory. If the project does not use a framework, it can directly use this level of tree to define an independent module. Independent modules can be defined in various heights.

% Within each module, a framework can be specified, such as "jax", "torch", or "keras", followed by the name of the specific model, experiment, or trainer being used. 
% Independent submodules can be created for individual components such as data loaders, data augmentations, or loss functions. These can be organized in the following manner: \texttt{"data/torch/imagenet"} or \texttt{"data/torch/augmentations/random\_crop"} or\texttt{"models/jax/gpt2"} or \texttt{"models/torch/big\_transfer"}.
% This Structure emphasizes the importance of creating independent modules that can be used across different projects, promoting code reuse and making research more efficient. By following this Structure, researchers can make their work more accessible, reproducible, and transparent, thereby promoting open science in the field of AI.


 The modular structure of the framework involves organizing the project directory in a hierarchical tree structure, with an arbitrary name given to the root project directory by the user. The project is then broken down into distinct concerns such as models, data, trainers, experiments, analyzers, simulators, each with its own subdirectory. Within each concern, modules can be defined with their own subdirectories, such as models, trainers, data loaders, data augmentations, or loss functions. Independent modules can be defined in various heights in the tree.

The framework prioritizes the organization of the project into independent modules when applicable, however there are situations where a combination of independent modules may be necessary for a particular concern. An example of this is the experiment concern, which imports and combines models, data, and trainers to define and create a specific experiment. In such cases, the module is not independent and is designed to combine the previously defined independent modules. In the case of non-independent modules. 
% Yerbamate creates a dependency list of independent modules that can be used to install the code and Python dependencies. This ensures that the necessary modules are installed and that the code can be reused. 

When implementing a project using multiple frameworks, it can become challenging to specify the exact framework used for each module. To address this issue, Yerbamaté provides a naming convention that specifies the framework used before the name of the module. For instance, the subdirectory for a Jax-based model named \texttt{"my\_model"} would be \texttt{"models/jax/my\_model"}, while the subdirectory for a PyTorch-based model with the same name would be \texttt{"models/torch/my\_model"}. This naming convention makes it easy to identify which framework was used for a specific module and ensures consistency throughout the project.

% While the prioritization of independency is encouraged, sometimes it may not be desired, and a combination of independent modules may be necessary for a particular concern. An example of this is the experiment concern, which imports and combines models, data, and trainers to create a specific experiment. In such cases, the module is not independent and is designed to combine the previously defined independent modules.



% The framework involves organizing the project directory in a hierarchical tree structure, with an arbitrary name given to the root project directory by user. The project is then broken down into separate concern modules, such as models, experiments, trainers, data, analyzers, and simulators, each with its own subdirectory. If the project does not use a framework, it can directly use this level of tree to define an independent module. Independent modules can be defined in various heights.

% Within each module, a framework can be specified, such as "jax", "torch", or "keras", followed by the name of the specific model, experiment, or trainer being used. Independent submodules can be created for individual components such as data loaders, data augmentations, or loss functions. 
% These can be organized in the following manner: \texttt{"data/torch/imagenet"} or \texttt{"data/torch/augmentations"} or\texttt{"optimizers/jax/gpt2"} or \texttt{"models/torch/big\_transfer"}.


\begin{figure}
\centering
\framebox[\0.5\textwidth]{%
\begin{minipage}{0.45\textwidth}
\dirtree{%
.1 project.
.2 data.
.3 torch.
.4 imagenet.
.4 \texttt{bit}.
.2 models.
.3 jax.
.4 transformer.
.4 gpt2.
.3 torch.
.4 \texttt{big\_transfer}.
.4 transformer.
.2 experiments.
.3 torch.
.4 bit.
.3 jax.
.4 gpt2.
.2 trainers.
.3 jax.
.4 gpt2.
.3 torch.
.4 \textt{bit}.
.2 ....
}
% For modular Python projects, organizing files into a hierarchical directory tree structure is a best practice.
\end{minipage}}
\caption{
This figure illustrates a directory tree example of a deep learning modular project. The organization of files into separated concerns in the tree is a common best practice for developing modular Python projects. By adhering to principle of independency, the submodules, such as Transformer and GPT2 can be easily shared and reused across different projects. Yerbamaté simplifies this process by providing a toolkit that allows for the injection of dependencies and the installation of modules from open-source projects.
}
\end{figure}


\subsection{Independent Python Modules}

Python independent modules are stand-alone modules that only depend on Python dependencies, such as NumPy, PyTorch, TensorFlow, or Hugging Face. The code inside the module uses relative imports to import within the module, making it an independent module that can be reused once the necessary Python dependencies are installed. This approach enhances modularity, reusability, and shareability while promoting good software engineering practices.

% Separation of concerns and organizing files into a modular hierarchical tree structure are among best software engineering practices for developing Python projects. 
% \subsection{Independent Python Modules}
% Python independent modules only depend on Python dependencies (such as NumPy, PyTorch, TensorFlow, or Hugging Face), and the code inside the module uses relative imports to import within the module, hence making it an in dependent module that can be re-used when python dependencies are installed.
% All independent python modules c
% This framework places a high priority on creating independent modules that are standalone and reusable components, which can be used across different projects. 
% This design allows the reuse of code and its components under different projects. This framework priotizes independency of modules over seperated concerns.


\subsection{No-Loop Python Experiment Definition}

This framework uses a Python experiment configuration definition that disallows the use of loops as per convention, promoting a hyperparameter-focused approach while maintaining separation of concerns. This format offers flexibility and customization, allowing the configuration format to be adapted to various AI tasks, custom use cases, frameworks, and libraries.


% \subsection{No-Loop Python Experiment Configuration Structure}

% This Structure adopts an experiment configuration with the Python language that only disallows the use of loops. The experiment configuration design choice is intended to maintain a hyperparameter-focused experiment format while promoting separation of concerns. The resulting approach provides several benefits, including increased flexibility and customization, as the configuration format can be adapted to various AI tasks, custom use cases, frameworks, and libraries. 



% Yerbamate follows a directory tree structure that is organized into distinct modules, namely "models", "experiments", "trainers", and "data", which have been identified as major concerns in AI projects across a variety of tasks. This naming convention is designed to enhance the understandability and readability of the project structure. 

\subsection{Modular Project Structure for AI}

The structure for AI projects follows a modular pattern that is organized into distinct modules, namely "models", "experiments", "trainers", and "data", which have been identified as major concerns in AI projects across a variety of tasks. This modular approach creates standalone and independent modules for models, trainers, and data, which can operate independently to enhance modularity and reusability. The "experiments" module harmonizes the three independent modules and provides a unified experiment. This naming convention is used to enhance the understandability of the project and facilitate the adoption of software engineering best practices, such as modularity and separation of concerns.

% \subsection{AI Project Structure}
% The structure for AI projects often involves four modules: "models", "experiments", "trainers", and "data". The modular approach creates arbitrary standalone modules for models, trainers, and data, each with the ability to operate independently. The experiments module harmonizes the three modules and provides a unified experiment. 
% This modular structure enables the sharing of independent models, trainers, and data modules across various projects, promoting code reusability.

% The Structure for structuring AI/ML projects involves four distinct modules: "models", "experiments", "trainers", and "data". This modular approach leads to the creation of three standalone modules for models, trainers, and data, each with the ability to operate independently. The fourth module, experiments, serves to harmonize the three modules and provide a unified and comprehensive experiment. The modular structure of independence enables the sharing of independent models, trainers, and data loaders across various projects and experiments, promoting the reusability of code.



% \begin{figure}
% \centering
% \framebox[\0.45\textwidth]{%
% \begin{minipage}{0.45\textwidth}
% \dirtree{%
% .1 /.
% .2 models.
% .3 \texttt{\_\_init\_\_.py}.
% .2 experiments.
% .3 \texttt{\_\_init\_\_.py}.
% .2 trainers.
% .3 \texttt{\_\_init\_\_.py}.
% .2 data.
% .3 \texttt{\_\_init\_\_.py}.
% .2 \texttt{\_\_init\_\_.py}.
% }
% \end{minipage}
% }
% \caption{The modular architecture of the Structure for AI is characterized by the structured folder system, separating the key components into four distinct modules: models, experiments, trainers, and data. This enhances code legibility and comprehensibility, enabling other researchers to quickly access the desired code components.}
% \end{figure}




% \subsubsection{}

% The implementation of software engineering principles of as separation of concerns (SoC) and modularity in deep learning projects results in a well-structured illustrated below. 

% \vspace{0.7em}


% The modular architecture of a deep learning project, characterized by the structured folder system, separates the key components into four distinct modules: models, experiments, trainers, and data. This enhances code legibility and comprehensibility, enabling other researchers to quickly access the desired code components. This approach leads to the formation of three standalone modules for models, trainers, and data, each possessing the ability to operate independently. Meanwhile, the fourth module, experiments, functions to harmonize the three modules and provide a unified and comprehensive experiment. The modular structure of independence further enables the sharing of models, trainers, and data loaders across various projects and experiments, thereby enhancing the code's reusability and scalability. 


% \begin{figure}
% \centering
% \framebox[\0.45\textwidth]{%
% \begin{minipage}{0.45\textwidth}


% \dirtree{%
% .1 /.
% .2 data.
% .3 cifar10.
% .4 \texttt{data\_loader.py}.
% .4 \texttt{\_\_init\_\_.py}.
% .3 cifar100.
% .4 \texttt{data\_loader.py}.
% .4 \texttt{\_\_init\_\_.py}.
% .3 \texttt{\_\_init\_\_.py}.
% .2 models.
% .3 resnet.
% .4 \texttt{fine\_tune.py}.
% .4 \texttt{resnet.py}.
% .4 \texttt{\_\_init\_\_.py}.
% .3 efficientnet.
% .4 \texttt{efficientnet.py}.
% .4 \texttt{\_\_init\_\_.py}.
% .3 \texttt{\_\_init\_\_.py}.
% .2 trainers.
% .3 classification.
% .4 \texttt{pl\_classification\_module.py}.
% .4 \texttt{\_\_init\_\_.py}.
% .3 \texttt{\_\_init\_\_.py}.
% .2 experiments.
% .3 torch.
% .4 \texttt{resnet\_cifar10.py}.
% .4 \texttt{resnet\_cifar100.py}.
% .4 \texttt{fine\_tune\_resnet\_cifar.py}.
% .4 \texttt{\_\_init\_\_.py}.
% .3 keras.
% .4 \texttt{keras\_efficient\_net\_cifar.py}.
% .4 \texttt{\_\_init\_\_.py}.
% .3 \texttt{\_\_init\_\_.py}.
% .2 \texttt{\_\_init\_\_.py}.
% }
% \end{minipage}
% }
% \caption{This figure illustration a sample folder structure for a deep learning vision project\footnote{ \url{https://github.com/oalee/deep-vision}}. Each folder represents a modular code component with a specific concern, which is demonstrated by the presence of \texttt{\_\_init\_\_.py} within the folder.}
% \end{figure}



% \subsection{Example Project Structure}


% \vspace{2em}
% Each top-level module can be divided into multiple sub-modules, allowing for the separation of individual components within each module. For instance, the models module can comprise sub-modules such as ResNet (\cite{resnet}), ViT (\cite{dosovitskiy2020vit}), CvT (\cite{wu2021cvt}), and others, implemented as independent components. Likewise, the data module can encompass sub-modules such as data loaders, preprocessing pipelines, and post-processing routines. This modular design promotes increased organization and manageability of the codebase, contributing to the efficiency and effectiveness of the deep learning project.


\subsection{Yerbamaté: An Open Science Python Framework}
% Yerbamaté\footnote{\url{https://github.com/oalee/yerbamate}} is an open-source, open-science Python framework that aims to streamline and simplify the development and management of AI projects. It promotes modular design principles and encourages quality coding, collaboration, sharing of models, trainers, data loaders, and knowledge, while also prioritizing reproducibility, customization, and flexibility. The framework provides a command-line interface (CLI) toolkit that conforms to the software engineering Structure of modularity and independence, allowing the creation and injection of dependencies for greater reproducibility and sharing capabilities. Moreover, Yerbamate is compatible with Linux systems and Jupyter notebooks, providing researchers with the ability to run experiments on Colab. By adhering to the Yerbamate Structure, any module in a project is automatically sharable.
% is a result of the study on reproducibility and the importance of modularity in AI and machine learning projects. Yerbamaté is an open source open science framework designed to streamline and simplify the development and management of artificial intelligence and machine learning projects. Built around the principles of modularity and separation of concerns, Yerbamaté provides a convenient and efficient means of adding source code and dependencies to projects.

% Modularity allows for a clean and organized codebase, making it easier to maintain, scale, and reuse the code in the future. In addition, Yerbamaté provides an easy-to-use interface for sharing the source code of models, trainers, data loaders, and experiments between modular projects, enabling greater flexibility and collaboration. Furthermore, adopting the Yerbamaté framework ensures that all projects adhering to its principles are readily compatible with Colab\footnote{\url{https://colab.research.google.com/github/oalee/yerbamate/blob/main/deep_learning.ipynb}}, a cloud-based platform widely used for machine learning experimentation and development.

% Yerbamaté also provides support for full customizability and reproducibility of results through the inclusion of dependencies in your project. The tool supports pip and conda for dependency management, making it easy to manage and install the necessary dependencies for your project.
% Additionally, Yerbamaté is fully compatible with python, and can be used with popular libraries such as PyTorch/Lightning, TensorFlow/Keras, JAX/Flax, Huggingface/transformers. Another feature of Yerbamaté is its convenient environment management through the Yerbamaté Environment API. This API allows for the creation and management of virtual environments, ensuring that each experiment is run in an isolated environment with the necessary dependencies. 
% For a comprehensive understanding of the Yerbamaté, see the documentation\footnote{\url{https://oalee.github.io/yerbamate/}}.


Yerbamate is an open-source open-science Python framework designed to streamline and simplify the development and management of AI projects using power of modular design principles and open source. The framework encourages quality coding, collaboration, and sharing of models, trainers, data loaders, and knowledge, while also promoting reproducibility, customization, and flexibility. 
Yerbamate is designed as a command-line interface (CLI) toolkit that works when the software engineering Structure of modularity and independence are adhered. The CLI can be used to create and inject dependencies, providing greater reproducibility and sharing capabilities. 
The framework also offers an Environment API, which, along with the CLI, is presented in the Appendix and subject to updates. Users can refer to the documentation for the most up-to-date material. Any module in a project adhering to the modular structure is automatically sharable out of the box, leveraging the power of open source to enhance open science and collaboration in the AI community.

\subsubsection{Flexibility and Customization}

One of the key challenges in the development of the Yerbamaté framework was ensuring flexibility and customization to meet the varying needs of researchers and practitioners in the field of AI. To address this challenge, Yerbamaté provides increased flexibility by not imposing any restrictions on additional module names, enabling researchers to utilize their preferred module names for custom tasks. 
% For example, researchers can use module names such as "simulations" and "analyzers" for their specific use cases, enabling greater customization and adaptability of the framework. 
Additionally, the framework is designed to be compatible with Python, providing greater flexibility as Python can directly be used to execute experiments and Python files.

% Additionally, the Yerbamaté framework is designed to be compatible with pure Python, providing greater flexibility as Python can directly be used to execute experiments and Python files. The Structure can be used without the toolkit, allowing for even greater customization and flexibility. This flexibility and customization offered by Yerbamaté can facilitate the implementation of diverse AI applications, supporting the development of a more accessible and collaborative AI research environment.

% \subsubsection{Flexibility and Customization}
% One of the key challenges in the development of the Yerbamaté framework was ensuring flexibility and customization to meet the varying needs of researchers and practitioners in the field of AI. To address this challenge, Yerbamaté provides increased flexibility by not imposing any restrictions on additional module names, enabling researchers to utilize their preferred module names for custom tasks. For example, researchers can use module names such as "simulations" and "analyzers" for their specific use cases, enabling greater customization and adaptability of the framework. Additionally, the Yerbamaté framework is designed to be compatible with pure Python, providing greater flexibility as python can directly be used to execute experiments and python files, and the Structure can be used without the toolkit. This flexibility and customization offered by Yerbamaté can facilitate the implementation of diverse AI applications, supporting the development of a more accessible and efficient AI research environment.
% framework-agnostic, meaning it can work out of the box with any machine learning framework or library.  Moreover, Yerbamaté offers compatibility with Python and Python can be used for running experiments. 


% Additionally, Yerbamate is compatible with Linux systems and Jupyter notebooks, enabling researchers to run their experiments on Colab out of the box. 

