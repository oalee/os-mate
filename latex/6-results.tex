\section{Results}

This section presents the results of the study, which provide software engineering principles for shareable modular Python projects and an open science Framework, Yerbamaté. The principles include modular project structuring, independent and interchangeable modules, and an experiment definition format.  Furthermore, the Yerbamaté CLI is introduced as a productivity tool for modular Python projects, promoting sharing and reuse of Python modules.


\subsection{Independent/Reusable Python Modules}

Independent/Reusable Python modules are self-contained modules that only depend on Python dependencies, such as NumPy, PyTorch, TensorFlow, or Hugging Face. The code inside the module uses relative imports to import within the module, making it independent and reusable once the necessary Python dependencies are installed locally. This approach enhances modularity, reusability, and shareability while promoting good software engineering practices.



\subsection{Separated Concern Python Modular Project Structure}


 The modular structure of the framework and software engineering best practice of SoC involves organizing the project directory in a hierarchical tree structure, with an arbitrary name given to the root project directory by the user. The project is then broken down into distinct concerns such as models, data, trainers, experiments, analyzers, simulators, each with its own subdirectory. Within each concern, arbitrary modules can be defined, including models, trainers, data loaders, data augmentations, and loss functions. Independent modules can be defined in various heights in the tree.

The framework prioritizes the organization of the project into independent modules when applicable, however there are situations where a combination of independent modules may be necessary for a particular concern. An example of this is the experiment concern, which imports and combines models, data, and trainers to define and create a specific experiment. In such cases, the module is not independent and is designed to combine the previously defined independent modules. 


Yerbamaté addresses the issue of specifying the exact framework used for each module in a project that involves multiple frameworks by providing a naming convention that specifies the framework used before the name of the module. This naming convention helps maintain consistency and make it easy to identify which framework was used for a specific module. For instance, the subdirectory for a Jax-based model named "my\_model" would be "models/jax/my\_model", while the subdirectory for a PyTorch-based model with the same name would be "models/torch/my\_model".




\begin{figure}
\centering
\framebox[\0.5\textwidth]{%
\begin{minipage}{0.45\textwidth}
\dirtree{%
.1 project.
.2 data.
.3 torch.
.4 imagenet.
.4 \texttt{bit}.
.2 models.
.3 jax.
.4 transformer.
.4 gpt2.
.3 torch.
.4 \texttt{big\_transfer}.
.4 transformer.
.2 experiments.
.3 torch.
.4 bit.
.3 jax.
.4 gpt2.
.2 trainers.
.3 jax.
.4 gpt2.
.3 torch.
.4 \textt{bit}.
.2 ....
}
\end{minipage}}
\caption{
This figure illustrates a directory tree example of a deep learning modular project. The organization of files into separated concerns in the tree is a common best practice for developing modular Python projects. By adhering to principle of independency, the submodules, such as Transformer and GPT2 can be easily shared and reused across different projects. 
}
\end{figure}



\subsection{No-Loop Python Experiment Definition}

This section presents a convention and guideline used in the Yerbamate framework for defining Python experiments that prohibits the use of loops. This approach aims to promote a hyperparameter-focused configuration and maintain separation of concerns, resulting in a more flexible and customizable format that can be adapted to various AI tasks, frameworks, and libraries. In practice, the configuration file should primarily consist of hyperparameters, object creations, and executions, while higher complexities such as training loops should generally be avoided.






% \subsection{Modular Project Structure for AI}

% The structure for AI projects results in a modular pattern that is organized into distinct  modules, namely "models", "experiments", "trainers", and "data", which have been identified as major concerns in AI projects across a variety of tasks. This modular approach creates standalone and independent modules for models, trainers, and data, which can operate independently to enhance modularity and reusability. The "experiments" module harmonizes the three independent modules and provides a unified experiment. This naming convention is used to enhance the understandability of the project and facilitate the adoption of software engineering best practices, such as modularity and separation of concerns.


\subsection{Yerbamaté: A Python Framework for Open Science}

Yerbamaté is an open science framework that prioritizes modularity, independent modules, and a separated concern project structure for Python projects. These principles enable Yerbamaté to provide a command line interface (CLI) for easy sharing and installation of code modules, dependency management, and execution of experiments. The framework also supports machine learning model development, with key features such as:

\subsubsection{Easy Installation with Yerbamaté Install}

Yerbamaté's install command enables the installation of modularized projects adhering to separation of concerns principles, along with their associated Python dependencies, with a single command. Additionally, Yerbamaté allows the installation of non-Yerbamaté projects as concrete modules, which can be installed alongside the root module of the project. While Yerbamaté cannot automatically install Python dependencies without a "requirements.txt" file in the root directory of the module, it can for example be used to install source code of over 100 Torch image models or 30 PyTorch vision in transformers directly into a project. See Appendix \ref{CLIApp}.

% \subsubsection{Metadata and Reproducibility}
% Yerbamate can be used to create and inject python version dependencies for reproduciblity by creating a requirements.txt file. Moreover metadata can be generated to create list of reusable modules that are in the project. THese metadata compily with FAIR pricniples

\subsubsection{Metadata and Reproducibility}

Yerbamaté enhances reproducibility by creating dependencies of a python module with a requirements.txt file that lists the python version dependencies. The framework can also be used to generate metadata for exported and reusable modules that comply with the FAIR principles. By adhering to these principles and guidelines, Yerbamaté promotes transparency, reproducibility, and open science practices in the research community.

\subsubsection{Environment API}

The Yerbamaté Environment API is a tool designed to manage environment variables within an experiment. It prioritizes the use of an \texttt{env.json} file for storing environment variables, falling back to the operating system's environment variables if the file is not found. The API offers a convenient way to set, retrieve, and manage these variables in a centralized and organized manner, which can be useful in storing and accessing environment-specific information, such as local results and data paths, API keys, database URLs, and other sensitive data.


% \subsection{Yerbamaté: A Python Framework for Open Science}

% Yerbamaté is an open science framework that prioritizes modularity, independent modules, and a separated concern project structure for Python projects. These principles enable Yerbamaté to provide a command line interface (CLI) for easy sharing and installation of code modules, dependency management, and execution of experiments. The framework also supports machine learning model development, with key features such as:

% \subsubsection{Easy Installation with Yerbamaté Install}

% Yerbamaté's install command enables the installation of modularized projects adhering to separation of concerns principles, along with their associated Python dependencies, with a single command. Additionally, Yerbamaté allows the installation of non-Yerbamaté projects as concrete modules, which can be installed alongside the root module of the project. While Yerbamaté cannot automatically install Python dependencies without a "requirements.txt" file in the root directory of the module, it allows the manual installation of over 100 Torch image models and 30 PyTorch vision implementations in transformers, for example.

% \subsubsection{Metadata and Reproducibility}

% \subsection{Yerbamaté: An Open Science Python Framework}




% Yerbamaté is an open science framework built around the aforementioned principles of modularity, independent modules, and a separated concern project structure for Python projects. When these principles and guidelines are adhered to, the resulting framework provides a command line interface (CLI) for sharing and installing code modules, dependency management, and execution of experiments. 



% Yerbamaté provides utility functions and supports modular projects, facilitating machine learning model development. Key Yerbamate features include:


% \subsubsection{Yerbamaté install}
% Yerbamaté's install command enables easy installation of modularized projects adhering to separation of concerns principles, including both source code and their associated Python dependencies, with a single command. In addition, Yerbamaté allows for the installation of non-Yerbamaté projects as concrete modules, which can be installed alongside the root module of the project. However, note that Yerbamaté cannot install Python dependencies automatically in the absence of a "requirements.txt" file in the root directory of the module.

% For example, one could install the source code of over 100 Torch image models\footnote{\url{https://github.com/rwightman/pytorch-image-models/tree/main/timm/}} and over 30 PyTorch vision implementations in transformers\footnote{\url{https://github.com/lucidrains/vit-pytorch/tree/main/vit_pytorch/}} directly into your project and then manually install the dependencies with pip.


% \subsubsection{Yerbamaté Environment API}

% The Yerbamaté Environment API is a tool designed to manage environment variables within a experiment. It prioritizes the use of an \texttt{env.json} file for storing environment variables, but if it is not found, it falls back to the operating system's environment variables. The API offers a convenient way to set, retrieve, and manage these variables in a centralized and organized manner. This can be particularly useful in storing and accessing environment-specific information, such as local results and data paths, API keys, database URLs, and other sensitive data.

