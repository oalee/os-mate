
\section{Introduction}

In recent years, artificial intelligence (AI) and machine learning (ML) have made remarkable progress and found numerous applications across a wide range of fields, including natural language processing, art, music, healthcare, finance, and education. 
While these technologies have the potential to revolutionize many areas of human life, their successful development and deployment depend heavily on quality data, code and sound software engineering practices. Good software engineering practices can increase the reliability, maintainability, and replicability of AI systems, and enhance the developer experience. 

% In this paper, we propose a software engineering convention for AI and ML that emphasizes modularity, and separation of concerns, promoting reproducibility, maintainability, and code quality. This convention enhances open science by enabling the sharing of standalone code modules, such as models, trainers, and data loaders, among researchers and practitioners. By promoting modular design, this convention allows for increased flexibility and reusability of code, making it easier to reproduce and build upon previous research.

% In this paper, we present a software engineering convention of modularity that enhances reproduciblility, maintainability, code quality, and enhances open science through the sharing of standalone code modules such as models, trainers, and data loaders between researchers and practitioners. 

% The proposed convention addresses the challenges associated with the current state of software engineering practices in the AI and ML fields and has the potential to facilitate the growth and adoption of new technologies.
% However, the current state of software engineering in the AI and ML fields leaves much to be desired. Many AI and ML systems are developed in an ad hoc manner, with limited attention to testing, documentation, version control, and other fundamental software engineering practices. This lack of attention to good software engineering can result in buggy, hard-to-maintain systems that invalidate research and impede progress. Furthermore, the lack of standardization in the AI and ML fields can make it difficult for researchers to collaborate and share their work effectively.
